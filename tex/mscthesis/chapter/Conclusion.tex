%ब
\chapter{Conclusions and Future Work}
% % Cloud data storage has been progressing and evolving to match and scale the
% % advances made in cloud computing. 
% Cloud \acp{DBMS} adopt various data models in order to provide  efficient and
% easy ways of storing data on the cloud.
% % and many different kinds of such.
% % \acp{DBMS} have been developed and used on the cloud.
% % Such \acp{DBMS} have evolved and adapted to address different problems related
% % to the cloud environment.
% These \acp{DBMS} commonly adopt the key-value data model to meet 
% scalability, flexibility, robustness and other features that are essential in
% the cloud environment. 
% This key-value data model allows data to be stored in 
% key-value pairs without strict or rigid schemas as seen in traditional \acp{RDBMS}.
% However, this data model   compromises some
% useful \ac{DBMS} features, such as integrity constraints, efficient querying,
% amongst others, in order to provide cloud-suitable features.
% This thesis specifically addressed the limitation of lack of
% referential integrity  constraints in  cloud column-oriented key-value
% \acp{DBMS}. 

\acp{DBMS} on the cloud adopt various data models in order to provide  efficient
and simple ways of storing data on the cloud.
These \acp{DBMS} commonly adopt the key-value data model which stores data 
as key-value pairs without strict or rigid schemas (contrary to the traditional
relational model).
This key-value data model provides cloud \acp{DBMS} with features like
scalability, flexibility and robustness, which are essential in the
cloud environment.  However, in order to provide these, other features  such as
referential integrity  are often sacrificed.

In such cloud \acp{DBMS}, maintaining referential integrity incurs an additional
computational cost that is avoided in order to provide high data availability
and partition-tolerance, following the CAP theorem~\citep{Brewer,Stonebraker}.
However, when referential integrity is not maintained, data dependencies are not
necessarily correct as dangling references can be introduced. Hence, 
maintaining referential integrity is left to be dealt with by the application.


This thesis presents an \ac{API} that maintains referential integrity in Apache
Cassandra and serves as a middle layer between the applications and the
\ac{DBMS}. The \ac{API} prevents the execution of operations that violate
referential integrity constraints. It does so by triggering validations and
ensuring that constraints applied on data are always satisfied.
% maintains referential integrity by ensuring that the constraints on the data
% are always satisfied whenever operations are performed.
% Moreover,
More importantly, the \ac{API} presents four approaches to store the referential
integrity constraints as metadata. Also, the performance of each approach is
analyzed in terms of response time and throughput upon a set of experiments. The
results obtained show  clear differences in  performance across the solutions.
However, the final word on which one is better depends on the application as
each solution presents different trade-offs.


The first solution has the second-best performance as it embeds the metadata
within each entity, thus providing immediate access to the constraints once the
entity is retrieved. However, this solution requires a large amount of disk
space as it stores the metadata within every entity. Additionally, if changes
are to be performed on the metadata, these must be updated in all entities, thus
making its management difficult.
While this solution has a rather good performance due to its quick access to the
constraints, the trade-off  involves large space requirements
coupled with complex metadata management \textit{versus} good performance.


The second solution has a slightly worse performance than the first one as it
stores the metadata in the top row of each column family. This solution requires
less disk space as metadata is not included within each entity. Also,
its management is much simpler as changes in metadata  require only updating it
in every column family. However, compared with the previous solution,  this
one requires an additional search operation to retrieve the metadata, therefore  
the performance is slightly worse. In this case, the trade-off to decide upon
involves lower disk space requirements and simpler management  of metadata
\textit{versus} the delay of an additional search operation to retrieve the metadata.


% The second solution has a slightly worse performance than the first one as it
% stores the metadata in the top row of each column family, thus requiring an
% additional search operation to retrieve it.  However, this solution requires
% less disk space as metadata is not included within each entity.
% Also, its management is much simpler as changes in metadata only require
% updating it once in each column family. Even when the performance of this
% solution is slightly poorer that the previous one, the trade-off to decide upon
% involves lower disk space requirements and simpler management of metadata
% \textit{versus} the delay of an additional search operation to retrieve the
% metadata.


The third solution has the worst performance of all. In this case, the
constraints relevant to every entity are  stored in a single column family
named \texttt{Metadata}, thus providing centralized metadata storage. Moreover, 
changes to the constraints require updating only the \texttt{Metadata} column
family. However, the performance is significantly reduced
because validations have to perform additional accesses to \texttt{Metadata} to
retrieve the relevant constraints. Thus, the trade-off in this
approach is low disk space requirements and simple management of metadata
\textit{versus} a  poor performance.

Lastly, the fourth solution has the best performance of all. It stores  
the metadata in a single column family (as the previous solution), but in  a
separate dedicated cluster. This approach requires to connect to such a cluster
every time metadata needs to be accessed. However, once metadata is
retrieved for the first time, it is stored in cache in order to re-use it
and avoid future connections to the cluster. Caching is the key aspect that
makes this solution have a superior performance than the rest, but when
 metadata is altered, additional mechanisms are required  to properly
 have the cache updated. The trade-off here is between high performance,  low
 disk space requirements, and simple management of metadata \textit{versus} having a
 potentially stale cache or implementing mechanisms to prevent it.


% The choice of a solution depends on the application requirements in terms of its
% metadata, keyspace size and additionally, the trade-offs that are acceptable to
% the application. Considering these factors, it can be summarised that Solution~1
% favors applications that handle small keyspaces with a small amount of
% constraints as metadata which is not prone to changes, while Solution~2 suits
% applications with larger keyspaces having more number of constraints. Solution~3
% is suitable for applications that handle keyspaces with sizeable  metadata
% (several constraints) that is prone to changes.
% Solution~4 favors applications that handle multiple keyspaces each containing
% large metadata, so that all the metadata can be centrally managed in one
% dedicated cluster. However, considering its high performance it can be used by
% most applications as long as the metadata is not prone to many changes.


In summary, this thesis has incorporated validation mechanisms for maintaining
referential integrity  in Apache Cassandra,  a prominent cloud column-oriented
key-value \ac{DBMS}. These mechanisms are implemented using four different
solutions to store the referential integrity constraints. The four solutions
were designed and implemented into the \ac{API} in order to perform experiments
and assess the response time and throughput of each solution. The results showed
that incorporating referential integrity into cloud NoSQL \acp{DBMS} clearly
affects the performance of the Create, Update and Delete operations. Moreover,
the storage and handling of metadata is a decisive factor in the performance of
referential integrity validations. Furthermore, the solutions presented have
different trade-offs between performance, disk space requirements, and metadata
management, all of which have to be carefully considered by the  application in
order to choose the most appropriate solution according to its demands.




% 
%  were designed and implemented to impose
% referential integrity in  column-oriented key-value cloud \acp{DBMS}, where
% constraints are preserved as metadata. These solutions were incorporated into a
% single \ac{API} and implemented in Cassandra and their performance was analysed
% upon a set of experiments, thus, successfully achieving the general and specific
% objectives of this thesis. It was concluded from the results that the approaches
% adopted to save metadata in order to perform referential integrity validations
% affect the performance of the operations and the cloud \ac{DBMS}.
% The solutions present some trade-offs that have to be carefully chosen before an
% application uses the \ac{API}. Nevertheless, the four solutions
%  preserve data integrity and maintain dependencies so that applications using
% them can carry on with their operations and leave such validations to the
% \ac{API}.


% The way dependency information is stored influences the time taken for an
% operation to complete when referential integrity validations are triggered.
% % In this thesis, Interestingly, % the time consumed to complete operations
% was shorter when metadata was embedded % with the entities than when metadata
% was stored separately.
% Although each solution improves data integrity, these also present some
% trade-offs.
% The performance of these solutions is analy  are provided within an \ac{API}
% that acts as a bridge between applications and Cassandra using an entity
% relationship model.
% that applications can use and extend the incorporation of referential
% integrity constraints in  Apache Cassandra, a prominent cloud column-oriented
% key-value \acp{DBMS}, by providing an \ac{API}  that applications can use and
% extend      The \ac{API} implements four solutions that maintain referential
% integrity constraints by preserving it as metadata within a keyspace. These
% four solutions store metadata in different ways, where metadata is either
% embedded with the actual data or is stored in a centralised way and maintained
% as a separate entity class.
% The performance of each solution on a set of experiments is analysed with
% respect to its response time and throughput .
% The results from the experiments showed that imposing such constraints and
% validations does affect the performance of a cloud \ac{DBMS}. The way
% dependency information is stored influences the time taken for an operation to
% complete when referential integrity validations are triggered.
% % In this thesis, Interestingly, % the time consumed to complete operations
% was shorter when metadata was embedded % with the entities than when metadata
% was stored separately.
% Although each solution improves data integrity, these also present some
% trade-offs.
% The first solution embeds the metadata within each entity thus providing
% immediate access to the metadata of an entity. However, the amount of space
% required to store all the metadata is increased significantly as every entity
% contains the metadata and when the number of constraints  stored is large, the
% utilisation of space is poor.
% % it presents a compromise in terms of space since all the entities contain
% the % same metadata.
% Moreover, changes to the metadata  requires updating it in every single
% entity, thus making metadata management difficult. While performance is faster
% due to quick access to constraints for any entity, much space is consumed in
% this approach which presents a  trade-off: large space requirements coupled
% with complex metadata management versus good performance.
% The second solution  stores the metadata as the top row within each column
% family and thus reduces the redundancy of metadata. This approach provides
% access to the metadata without having to connect to other entity classes or
% clusters and consumes less space. Furthermore, changes to metadata requires
% changing the values only in the top row of every entity class, making metadata
% management efficient to a certain extent. However, it performs slightly slower
% since it requires additional search operations to locate the metadata in an
% entity class. The trade-off is less  space requirements versus slightly worse
% performance than Solution~1.
% The third solution centralises the metadata by storing it separately in a
% column family within the same cluster. This centralised approach provides
% easier management of metadata as changes to the metadata require updating only
% the \texttt{Metadata} column family. However, the performance is significantly
% reduced during validations as additional accesses to \texttt{Metadata} are
% required each time constraints have to be retrieved. Thus, the trade-off is
% poor performance versus  efficient  space and metadata management.
% Similarly, the fourth solution centralises metadata by storing it in an entity
% class but in a separate dedicated cluster. This approach requires a separate
% connection to the dedicated cluster every time the \texttt{Metadata} column
% family has to be accessed, which tends to make it slower than Solution~3.
% However, the performance is significantly improved by caching the metadata and
% re-using it in validations. Nonetheless, if metadata is altered, mechanisms to
% update the cache appropriately is required.
% While the experiment results show that the performance of this solution is
% superior than the rest of the solutions, it comes with the cost of having a
% potentially stale cache. The trade-off is high performance versus
% cache-management if metadata is susceptible to changes.
% The choice of a solution depends on the application requirements in terms of
% its metadata, keyspace size and additionally, the trade-offs that are
% acceptable to the application. Solution~1 favors small keyspaces with a small
% amount of metadata which is not prone to changes, while Solution~2 suits
% larger keyspaces with slightly bigger metadata.
% Solution~3 fits applications with sizeable  metadata that is prone to changes.
% Solution~4 favors applications that handle multiple keyspaces with large
% metadata,although considering its high performance, it can be used by most
% applications.
% In summary, four solutions were designed and implemented to impose referential
% integrity constraints and validations in  column-oriented key-value cloud
% \acp{DBMS}, where dependency information is preserved as metadata. The four
% solutions were incorporated into a single \ac{API} and implemented in
% Cassandra and their performance was analysed, thus, successfully achieving the
% general and specific objectives of this thesis. It was concluded from the
% results that the approach adopted to save metadata affects the performance of
% the operations and the cloud \ac{DBMS}. The solutions present some trade-offs
% that have to be carefully chosen before an application uses the \ac{API}.
% Nevertheless, the four solutions  preserve data integrity and maintain
% dependencies so that applications using them can carry on with their
% operations and leave such validations to the \ac{API}.



This thesis can be further extended by:

\begin{itemize}
   \setlength{\itemsep}{0pt}%
  \item Incorporating mechanisms in the \ac{API} to support thread-safe operations
  on the data such that referential integrity is not compromised when multiple
  concurrent operations take place. In order to achieve this, locking mechanisms
  and transation support for Apache Cassandra might be implemented using
  libraries such as Cages.
  
  \item Implementing trigger procedures in order to initiate events before and
  after performing any \ac{CRUD} operations. This could be implemented by
  extending the validation handlers to execute methods for such events in every
  operation. Also, entities could provide annotations for such methods.
  
  \item Implementing the four solutions on another key-value \ac{DBMS} such as
  Google's BigTable or Apache's HBase in order to compare the performance with 
  the results presented in this thesis. 
  
  \item Making the \ac{API} \ac{DBMS}-agnostic. That is, allowing the
  possibility to switch the underlying cloud NoSQL \ac{DBMS} with minimal
  reconfiguration in order to provide consistent usage of the \ac{API} on
  \acp{DBMS} such as BigTable, HBase, and others.
  
  \item Incorporating constraints for composite primary keys and implementing
  their respective handling.
  
  
  \item Evaluating the performance impact of Hector by experimenting with other
  \acp{API} for Cassandra such as Pelops or Kundera.
  
  
  \item Adapting the \ac{API} to work on a real cloud environment such as Amazon
  EC2 where heterogeneity and  dynamism  are inherent properties as nodes may
  have different characteristics and more nodes can be added or
  removed as well.
  Moreover, useful information about the solutions proposed in this thesis 
  can be drawn from performing the experiments in such environments.
  
  \item Consolidating the performance of the \ac{API} and the solutions 
  by using additional benchmarks such as \ac{YCSB} which measures latency,
  scalability, and other features of cloud \acp{DBMS}.
 
 \end{itemize}
  
  \begin{comment}
  
  analysing the performance of the four solutions on Amazon
  EC2 using the Yahoo Cloud Service Benchmark (YCSB) which provides benchmarks for several
  cloud \ac{NoSQL} \acp{DBMS}. This benchmark can asses the performance of the
  four solutions on Amazon EC2, using specific metrics to measure latency,
  scalability, and other features of a cloud \ac{DBMS} and can thus give detailed
  information on how such features affect the performance of the solutions.
  
  This will assess the performance of the four solutions in a dynamic and
  heterogeneous environment, where the nodes in a cluster have different
  characteristics.  The results of the Amazon EC2 implementation can be
  compared with the results of this thesis to determine if the solutions perform
  differently in a real cloud environment. Such a comparison will help in
  improving the solutions to make it more efficient on cloud environments
  
  . Experimentation canperformance of  
  the four solutions on the new \ac{DBMS} can be analysed and
  compared with the results from this thesis, in order to understand the way the
  features of a \ac{DBMS} affects the performance. Understanding such effects
  will be beneficial as necessary improvements can be introduced to
  the four solutions to make it \ac{DBMS} independent.
  
  
  
  the \ac{CRUD} operations are not affected adversely by  concurrent users. 
  This would require using additional libraries such as  Cages in order to 
  provide locking mechanisms and transaction support for cloud
  \ac{NoSQL} \acp{DBMS}. Using such mechanisms, correct writes and updates are
  performed even when multiple users concurrently change the values of a data
  item .
    
    \item Incorporating other integrity constraints 
    
    entity constraints such as 
    
  \item Incorporating composite primary keys in the referential integrity
  validations as well as  providing additional validation mechanisms for other
  integrity constraints such as entity constraints and user defined integrity
   constraints. This will provide applications using the \ac{API} with the
  mechanisms to ensure data integrity in all respects.
%   \item extending the \ac{API} to include composite primary keys in the
%   referential integrity validations and also incorporating additional validation
%   mechanisms for other integrity constraints such as entity constraints and user
%   defined integrity constraints. This will provide applications using the
%   \ac{API} with the mechanisms to ensure data integrity in all respects.
  
  \item incorporating mechanisms in the \ac{API} to provide the thread safety
  such that the \ac{CRUD} operations are not affected adversely by large number
  of concurrent users. This would require using additional libraries such as
  Cages in order to provide locking mechanisms and transaction support for cloud
  \ac{NoSQL} \acp{DBMS}. Using such mechanisms, correct writes and updates are
  performed even when multiple users concurrently change the values of a data
  item .
% Moreover, such external libraries provide  transactions, which is a feature
% planned for Cassandra at the time of writing.
% This will provide support for multiple and concurrent operations on data while
% ensuring integrity as it allows operations to be rolled back if the operations
% are not successfully completed.
  
  \item using a different client \ac{API} for Cassandra instead of Hector, such
  as Pelops or Kundera which are compatible with Java applications.
  The performance of the solutions when such new and advanced client \acp{API}
  are implemented can be analysed in order to determine whether migrating from
  Hector to these new \acp{API} will improve the performance of the solutions in
  any way.
  
  \item implementing the four solutions on another key-value \ac{DBMS} such as
  BigTable or HBase.
% This might require some changes since a different client \ac{API} will have to
% be incorporated for the four solutions to interact with the new \ac{DBMS} as
% Hector is exclusively used for Cassandra.
  The performance of the four solutions on the new \ac{DBMS} can be analysed and
  compared with the results from this thesis, in order to understand the way the
  features of a \ac{DBMS} affects the performance. Understanding such effects
  will be beneficial as necessary improvements can be introduced to
  the four solutions to make it \ac{DBMS} independent.
% independent of the underlying cloud \ac{DBMS} unlike in this thesis , where
% the performance of all the solutions are affected by Cassandra features such
% as tombstone delete.
  
  \item implementing the \ac{API} on a real cloud environment such as Amazon
  EC2. This will assess the performance of the four solutions in a dynamic and
  heterogeneous environment, where the nodes in a cluster have different
  characteristics.  The results of the Amazon EC2 implementation can be
  compared with the results of this thesis to determine if the solutions perform
  differently in a real cloud environment. Such a comparison will help in
  improving the solutions to make it more efficient on cloud environments.
  
  \item analysing the performance of the four solutions on Amazon
  EC2 using the Yahoo Cloud Service Benchmark (YCSB) which provides benchmarks for several
  cloud \ac{NoSQL} \acp{DBMS}. This benchmark can asses the performance of the
  four solutions on Amazon EC2, using specific metrics to measure latency,
  scalability, and other features of a cloud \ac{DBMS} and can thus give detailed
  information on how such features affect the performance of the solutions.
  

  
  

% \item dynamic metadata changes are supported in the solutions.





% Hello: \citet{tokyo-copy}


\end{comment}


