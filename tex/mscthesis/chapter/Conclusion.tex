%ब
\chapter{Conclusions and Future Work}
% % Cloud data storage has been progressing and evolving to match and scale the
% % advances made in cloud computing. 
% Cloud \acp{DBMS} adopt various data models in order to provide  efficient and
% easy ways of storing data on the cloud.
% % and many different kinds of such.
% % \acp{DBMS} have been developed and used on the cloud.
% % Such \acp{DBMS} have evolved and adapted to address different problems related
% % to the cloud environment.
% These \acp{DBMS} commonly adopt the key-value data model to meet 
% scalability, flexibility, robustness and other features that are essential in
% the cloud environment. 
% This key-value data model allows data to be stored in 
% key-value pairs without strict or rigid schemas as seen in traditional \acp{RDBMS}.
% However, this data model   compromises some
% useful \ac{DBMS} features, such as integrity constraints, efficient querying,
% amongst others, in order to provide cloud-suitable features.
% This thesis specifically addressed the limitation of lack of
% referential integrity  constraints in  cloud column-oriented key-value
% \acp{DBMS}. 

\acp{DBMS} on the cloud commonly adopt the key-value data model which stores
data as key-value pairs without strict or rigid schemas (contrary to the traditional
relational model).
This key-value data model provides cloud \ac{NoSQL} \acp{DBMS} with features
like scalability, flexibility and robustness, which are essential in the
cloud environment.  However, in order to provide these, other features  such as
referential integrity  are often sacrificed.

In such cloud \acp{DBMS}, maintaining referential integrity incurs an additional
computational cost that is avoided in order to provide high data availability
and partition-tolerance, following the CAP theorem~\citep{Brewer,Stonebraker}.
However, when referential integrity is not maintained, data dependencies are not
necessarily correct as dangling references can be introduced. Hence, 
maintaining referential integrity is left to be dealt with by the application.


This thesis has incorporated validation
mechanisms for maintaining referential integrity  in Apache Cassandra,  a
prominent cloud column-oriented key-value \ac{DBMS}. These mechanisms are
provided within an \ac{API} that serves as a middle layer between the
applications and the \ac{DBMS}. The \ac{API} prevents the execution of
operations that violate referential integrity constraints and does so by
triggering validations which ensure that referential integrity constraints
applied on data are always satisfied.
More importantly, the \ac{API} presents four approaches, as solutions, to store
the referential integrity constraints as metadata.

% The four solutions were designed and implemented into the \ac{API} in order to
% perform experiments and assess the performance of each solution. 


The performance of each solution is analyzed in terms of response time and
throughput upon a set of experiments.
These experiments were designed to validate and test the \ac{API} and the four
solutions and  performed \ac{CRUD} operations upon artificial data created
for an example application, specifically designed for this purpose. These
experiments were developed  to determine how each solution affects the
performance of the \ac{CRUD} operations,  specifically those where referential
integrity validations are triggered (namely Create,  Update and Delete). In
order to perform reliable experiments and obtain unbiased results,  a
homogeneous set of nodes in the university labs were used and manually configured to form a
Cassandra cluster. These experiments presented results which showed  clear
differences in  performance across the solutions. The results  helped in clearly
understanding the performance of each solution as it was compared and analysed
against the baseline experiment where no referential integrity validations were
implemented. However, the final word on which one is better depends on the
application as each solution presents different trade-offs.




% In summary, this thesis  These mechanisms are implemented using four different
% solutions to store the referential integrity constraints.

The first solution has the second-best performance as it embeds the metadata
within each entity, thus providing immediate access to the constraints once the
entity is retrieved. However, this solution requires a large amount of disk
space as it stores the metadata within every entity. Additionally, if changes
are to be performed on the metadata, these must be updated in all entities, thus
making its management difficult.
While this solution has a rather good performance due to its quick access to the
constraints, the trade-off  involves large space requirements
coupled with complex metadata management \textit{versus} good performance.


The second solution has a slightly worse performance than the first one as it
stores the metadata in the top row of each column family. This solution requires
less disk space as metadata is not included within each entity. Also, its
management is much simpler as changes in metadata  require only updating it in
every column family. However, compared with the previous solution,  this one
requires an additional search operation to retrieve the metadata, therefore the
performance is slightly worse. In this case, the trade-off to decide upon
involves lower disk space requirements and simpler management  of metadata
\textit{versus} the delay induced by an additional search operation to retrieve
the metadata.


% The second solution has a slightly worse performance than the first one as it
% stores the metadata in the top row of each column family, thus requiring an
% additional search operation to retrieve it.  However, this solution requires
% less disk space as metadata is not included within each entity.
% Also, its management is much simpler as changes in metadata only require
% updating it once in each column family. Even when the performance of this
% solution is slightly poorer that the previous one, the trade-off to decide upon
% involves lower disk space requirements and simpler management of metadata
% \textit{versus} the delay of an additional search operation to retrieve the
% metadata.


The third solution has the worst performance of all. In this case, the
constraints relevant to every entity are  stored in a single column family
named \texttt{Metadata}, thus providing centralized metadata storage. Moreover, 
changes to the constraints require updating only the \texttt{Metadata} column
family. However, the performance is significantly reduced
because validations have to perform additional accesses to \texttt{Metadata} to
retrieve the relevant constraints. Thus, the trade-off in this
approach is low disk space requirements and simple management of metadata
\textit{versus} a  poor performance.

Lastly, the fourth solution has the best performance of all. It stores  
the metadata in a single column family (as the previous solution), but in  a
separate dedicated cluster. This approach requires to connect to such a cluster
every time metadata needs to be accessed. However, once metadata is
retrieved for the first time, it is stored in cache in order to re-use it
and avoid future connections to the cluster. Caching is the key aspect that
makes this solution have a superior performance than the rest, but when
 metadata is altered, additional mechanisms are required  to properly
 have the cache updated. The trade-off here is between high performance,  low
 disk space requirements, and simple management of metadata \textit{versus} having a
 potentially stale cache or implementing mechanisms to prevent it.




In summary, this thesis has presented a study about the existing data models
used by cloud \acp{DBMS}, and explored the challenge of imposing referential
integrity constraints, especially in cloud \acp{NoSQL} \acp{DBMS}. Based on
this, an \ac{API} is designed and implemented to maintain referential integrity
in Apache Cassandra. More importantly, four solutions to store the referential
integrity constraints were devised and incorporated into the \ac{API}.
A set of experiments was performed and their results  were analysed to asses the
performance of the solutions. The results consolidated the metadata storage as
an influential and important aspect regarding the performance of referential
integrity validations in cloud \ac{NoSQL} \acp{DBMS}.  Furthermore, it was
concluded that the solutions present different trade-offs between performance,
disk space requirements, and metadata management, all of which have to be
carefully considered by  applications in order to choose the most appropriate
solution according to their demands.\newline
% Such a cluster was deployed in an uncontrolled environment where it was not
% possible to use it as a dedicated cluster as it was used for grid jobs or by
% students. Moreover, such an environment required that some configurations were
% manually edited for Cassandra to run correctly.


This research can be further extended by:

\begin{itemize}
  \item Incorporating mechanisms in the \ac{API} to support thread-safe operations
  on the data such that referential integrity is not compromised when multiple
  concurrent operations take place. In order to achieve this, locking mechanisms
  and transaction support for Apache Cassandra might be implemented using
  libraries such as Cages.
  
  \item Implementing trigger procedures in order to initiate events before and
  after performing any \ac{CRUD} operations. This could be implemented by
  extending the validation handlers to execute methods for such events in every
  operation. Also, entities could provide annotations for such methods.
  
  \item Implementing the four solutions on another key-value \ac{DBMS} such as
  Google's BigTable or Apache's HBase in order to compare the performance with 
  the results presented in this thesis. 
  
  \item Making the \ac{API} \ac{DBMS}-agnostic. That is, allowing the
  possibility to switch the underlying cloud NoSQL \ac{DBMS} with minimal
  reconfiguration in order to provide consistent usage of the \ac{API} on
  \acp{DBMS} such as BigTable, HBase, and others.
  
  \item Incorporating constraints for composite primary keys and implementing
  their respective handling.
  
  
  \item Evaluating the performance impact of Hector by experimenting with other
  \acp{API} for Cassandra such as Pelops or Kundera.
  
  
  \item Adapting the \ac{API} to work on a real cloud environment such as Amazon
  EC2 where heterogeneity and  dynamism  are inherent properties as nodes may
  have different characteristics and more nodes can be added or
  removed as well.
  Moreover, useful information about the solutions proposed in this thesis 
  can be drawn from performing the experiments in such environments.
  
  \item Consolidating the performance of the \ac{API} and the solutions 
  by using additional benchmarks such as \ac{YCSB} which measures latency,
  scalability, and other features of cloud \acp{DBMS}.
 
\end{itemize}


