%ब
\chapter{Conclusion}
Cloud data storage has been progressing and evolving to match and scale the
advances made in cloud computing. In order to provide for efficient and easy
ways of storing data on the cloud, many different kinds of cloud \acp{DBMS} that
adopt various data models have been developed and used by several users.
% Such \acp{DBMS} have evolved and adapted to address different problems related
% to the cloud environment.
Cloud \acp{DBMS} commonly adopt the key-value data model to meet the
scalability, flexibility, robustness and other features essential in the cloud
environment. The key-value data model allows data to be stored in  key-value
pairs without strict or rigid schemas as seen in traditional \acp{RDBMS}.
However, this data model is prone to limitations and have to compromise some
useful \ac{DBMS} features in order to be adaptable on the cloud.
This thesis specifically addressed one such challenge of maintaining
referential integrity in  cloud \acp{DBMS} that adopt the column-oriented key-value data
model. 

Referential integrity constraints are not provided as a cloud \ac{DBMS} feature
as commonly  consistency is sacrificed for data availability and
partition-tolerance, following the CAP theorem. These constraints help preserve
data dependencies in  \acp{DBMS} and prevent operations that violate
referential integrity. By not having such constraints in cloud \acp{DBMS} data
dependencies are not correctly maintained or applications have to incorporate
the logic to maintain referential integrity, which increases the application
workload. This thesis addresses such problems and contributes by providing four
solutions with different approaches to incorporate referential integrity
constraints and their validations in a cloud column-oriented key-value
\ac{DBMS}.

The four solutions maintain the dependency information as referential integrity
constraints preserved as metadata within a keyspace.  The performance of each
solution was analysed with respect to their response time and throughput after
running experiments. The results from the experiments  showed that the way
dependency information is saved influences the time taken for an operation to
complete after performing referential integrity validations. Generally when
metadata is embedded within the actual data, the time consumed to complete
operations was shorter than when metadata is centralised and maintained as a
separate entity class. Nonetheless, caching the metadata when it is centralised
improves the performance significantly. However each solution presnts a
trade-off along with the  benefits such of preserving the dependency
information and providing referential integrity validations, they also present
some trade-offs.

The first solution embeds the metadata within each entity thus providing easy
access to the constraints during validations. Nonetheless, when the number of
constraints to be stored is large, it presents a compromise in terms of space as
each entity will contain large text values. While performance is faster with
quick access to constraints for any entity, it can consume much space as
metadata resides within each entity. Moreover changes to metadata will require
changing the metadata text in every single entity, thus making metadata management
difficult.

The second solution overcomes the redundancy of metadata in Solution~1 and saves
metadata as the top row of an entity class. This approach provides access to the
metadata of an entity without having to connect to other entity classes or
clusters and consumes less space when compared to Solution~1.  Furthermore,
changes to metadata requires changing the values only in the top row of every
entity class, making metadata management slightly more efficient. However, it
performs slightly poorly than Solution~1 and again pose a problem when the
constraints are large, which requires more searches within the entity class to
locate the constraints. The trade-off is between space and performance
 that include additional metadata retrieval operations.

The third solution centralises metadata and stores it separately in an entity
class, in the same cluster. While this approach provides better and easier
management of large metadata, it affects the performance during validations as
additional accesses are required each time metadata has to be accessed. It thus
becomes a balance of performance or efficient space and metadata management.

Similarly, the fourth solution centralises metadata and stores it in a separate
entity class in a separate cluster which is dedicated to storing metadata. Such
an approach is ideal for applications that handle multiple keyspaces and large
metadata for such keyspaces. Moreover this solution caches the metadata for a
keyspace and re-uses the metadata for each validation on an entity. However, if
metadata is altered in any way, the cache would need to be updated as well.
While the experiment results show that the performance of this solution is
better than the rest of the solutions, it presents a trade-off between superior
performance but cache-management if metadata is dynamic or susceptible to
changes.




