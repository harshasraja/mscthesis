%ब
\chapter{Results and Discussions}
 

The performance of the solutions is measured in terms of response time and
throughput while validating referential integrity in the experiments.  Response e
time and throughput are common \ac{DBMS} performance indicators.  The response
time  indicates the time taken for an operation to be completed  while
throughput measures the number of operations that can be  completed in a unit of
time. 
The performance of the operations when referential integrity validations are not
enforced is also measured and considered as the baseline with which the
solutions are compared.   Such a comparison determines the difference in
performance  when such additional validations are enforced using the \ac{API}
and provides a guideline to asses the performance impact of each solution. 

The results from the experiments are analysed and discussed in this chapter. 
Section~\ref{s:results-overview} presents an overview of the  performance of the
four solutions.  
Section~\ref{s:results-Baseline} presents the performance without referential
integrity validations.  
Sections~\ref{s:results-insert},  ~\ref{s:results-update}
and~\ref{s:results-Delete} compares the results
of all the solutions for the \texttt{insert},   \texttt{update} and
\texttt{delete} operations (respectively).   
% Section presents the analysis for the \texttt{update}
% operation for all the solutions.  
% Section discusses the results of the solutions for the
% \texttt{delete} operation.  
Section~\ref{s:comparisonOfOperations} presents an overall comparison of the
operations.  
 Finally,   Section~\ref{s:results-summary} presents a summary of this chapter.  

\newcommand{\Width}{0.5\textwidth}
\newcommand{\TB}[1]{\textbf{#1}} 

%ब 
\section{Overview of Results} \label{s:results-overview}

The experiments were performed to evaluate the response time and throughput of
the solutions in order to determine the impact of the metadata storage and
referential integrity validations on the performance of Cassandra. 
% The results from these experiments are presented in
% Tables~\ref{tres:ResponseTime}, ~\ref{tres:ResponsetimeRatio}, ~\ref{tres:Throughput}
% and~\ref{tres:ThroughputRatio}. 
Tables~\ref{tres:ResponseTime} and~\ref{tres:Throughput} present the mean and
standard deviation of the average response time for all the solutions and the
throughput of each operation for each solution.  Notice that the solution with the lowest
response time and highest throughput has a better performance than rest,  while
the solution with the highest response time and lowest throughput has the worst. 
This means that the better performing solution can complete more operations in a
second and excute each operation with the least amount of time. 



As seen in these tables,  Solution~4 performs the best amongst all since its
response times for all the  operations on every entity are the least. 
Conversely,  Solution~3 performs the worst amongst all with high response times
for all the operations and the lowest throughput in all the cases respectively. 
Regarding Solutions~1 and 2,  they perform similarly although Solution~1 is
faster with slightly smaller response times and higher throughput.  Note that
Solution~4 is slower than  Baseline  because the baseline does not perform any
referential validations or store metadata. 

% The results show the throughput and the response time of each solution to
% complete an operation with referential integrity validation on a single
% entity.  The throughput shows how many operations are performed in one second
% in each solution. 
This can be further seen in the ratios of the response time and throughput
presented in Tables~\ref{tres:ResponsetimeRatio} and~\ref{tres:ThroughputRatio}. 
The former shows the ratio of the response time of each solution when compared
to that of the baseline,  and it indicates the factor by which any solution is
slower than the baseline;  while
% many times faster or slower each solution is when compared to the baseline. 
the latter shows the ratio of the throughput when compared to that of the
baseline. 
% The results show that all the solutions perform differently and have different
% response times and throughput. 
The variations in the performance of the solutions is due to ways these store
and handle metadata.  Recall that Solutions~1 and 2 store metadata along with the
actual data where Solution~1 stores it in every row and Solution~2 stores it as
the top row of a column family.  On the other hand,  Solutions~3 and 4 store
metadata separately from the actual data  in a \texttt{Metadata} column family
but such a  \texttt{Metadata} column family is in a separate cluster in
Solution~4. 
% The results show that the way metadata is stored and retrieved in each solution
% affects its performance during validation,  when constraints are retrieved and
% used. 

From these results,  it can be seen that Solution~4  is faster than the other
solutions when performing the validations  since it caches the list of
constraints and avoids connecting to the external cluster to access the
\texttt{Metadata} column family each time  operations are invoked on entities. 
Therefore,  to locate the relevant \ac{FK} and \ac{PK} constraints of an entity, 
the constraints stored in the cache memory are re-used. 
Performance is improved significantly just by caching the 
\texttt{Metadata} column family as it reduces the number of accesses to the
column family. 

On the other hand,  Solution~3 is the slowest  because of the
way it accesses the metadata every time from \texttt{Metadata} column family. 
% of the way the metadata is accessed for the entities in this solution. 
% % irrespective of whether an entity is a parent or child the %
% \texttt{ValidationHandler} performs the same check the metedata of the entity
% % for all the solutions.  It is when this check is made it is clear if the
% entity % is a parent or child. 
% In this solution accessing
In this solution,  in order to retrieve the relevant \ac{FK} constraints for
an entity the \texttt{Metadata} column family has to be accessed.  This means
that for each operation on an entity,  an additional access is required to
\texttt{Metadata} which consumes more time. 
Moreover,  to retrieve the information about any referencing constraints within
the relevant \ac{FK} constraints of an entity,  \texttt{Metadata} is accessed
again. 
% For example,  in order to get the information of a \ac{PK} constraint stored in the
% \texttt{RConstraintName} column of a \ac{FK} constraint,  the \texttt{Metadata}
% column family is again accessed and the \ac{PK} constraint is searched.  
Thus,  in order to
complete each validation,  \texttt{Metadata} is accessed more than once. 
% \texttt{Metadata} column family has to be accessed using the connection
% object. 
Unlike Solution~4,  metadata is not cached for re-use thus costing multiple
access to \texttt{Metadata} column family. 

Meanwhile,  Solutions~1 and~2 have approximately similar response times as both
the solutions store the whole list of constraints with the actual data and 
requires no additional connections to access the relevant as well as
referenced constraints of an entity.  Note that, in Solution~1 since the constraints
are stored with each entity it performs slightly better than
Solution~2 because the latter has an additional search operation to identify the
top row in a column family to locate the relevant constraints of an entity.  Both
solutions are faster than Solution~3 mainly because these have the whole
list of constraints along with the actual data.  However,  they are slower than
Solution~4 as these have to access the constraints from each entity every time and do not
use a cache. 
% their metadata access for these solutions are easier as metadata is a part of
% the entity and no additional connection to a metadata column family is
% required. 


% Unlike this,  Solution~4 caches  metadata for entities and re-uses it thus
% saving time by not having to access a separate column family for each entity
% insertion. 

The performance of the solutions in each
operation performed on the entities is discussed in detail in the following
sections.  Notice that,  in all the tables anf figures,   the
entities \texttt{Student},  \texttt{Course} and \texttt{Enrolment} are referred
to as '\texttt{s}',  '\texttt{c}' and '\texttt{e}' for the sake of brevity. 

%ब 
\section{Baseline Experiment} \label{s:results-Baseline}

This experiment was designed to measure the performance of the \ac{CRUD} 
operations when no referential integrity validations are triggered. That is, 
\begin{inparaenum}[a)]
\item no checks are performed to ensure that parent entities exist before inserting 
or updating child entities; and
\item no checks are performed to ensure that child entities exist before deleting
parent entities or updating their primary key values.
\end{inparaenum}
Thus, this experiment is taken as a baseline to assess the impact in performance of 
the \ac{CRUD} operations when referential integrity validations are incorporated and 
the list of constraints on the entities is stored in different locations and handled 
according to each of the solutions. 


The results from this experiment can be seen in Figure~\ref{fres:Baseline}.
Specifically, Figure~\ref{fres:Baseline-responsetime} presents the average
response time of the operations on a single entity in the three column families.
Similarly, Figure~\ref{fres:Baseline-throughput} presents the throughput of such
operations. These results show that the performance of the  \texttt{insert}
operation is rather similar between the entities, like the \texttt{delete}
operation. On the \texttt{update} operation, the performance for updating
\texttt{Student} and \texttt{Course} is rather similar, but drastically better
when it comes to updating \texttt{Enrolment}.
 
	\begin{figure}[H]		
		\subfigure[Response time]
		{\includegraphics[width=\Width]{figure/result/barplot-Baseline-rt.pdf}\label{fres:Baseline-responsetime}}
		\subfigure[Throughput]
		{\includegraphics[width=\Width]{figure/result/barplot-Baseline-tp.pdf}\label{fres:Baseline-throughput}}
		\caption{Performance of Baseline}\label{fres:Baseline}
	\end{figure}
The performance of the  \texttt{insert} operation is expected to be similar
across solutions since no referential integrity validations are performed. However, the figure
shows a slightly  better average performance when inserting in
\texttt{Enrolment}. This subtle difference might be due to the smaller number of
columns as well as the smaller size of the contents that \texttt{Enrolment}
entities have when compared to \texttt{Course} and even more when compared to
\texttt{Student}. Also, external factors such as network latency are expected to
affect the performance slightly.

The performance of \texttt{delete} operations is similar across entities as
well, and quite similar to the performance of the \texttt{insert} operations.
This is because the tombstone delete paradigm in Cassandra does not allow a
complete removal of the super columns, but rather it keeps the row keys and
writes empty values to the columns to mark the super column as deleted. Thus, it
is expected as well to perform similar to the \texttt{insert} operations.


Finally, the performance of the \texttt{update} operation is worse
as it takes more time to complete because it involves  \texttt{insert} and
\texttt{delete} operations in the cases of \texttt{Course} and
\texttt{Student}. However, the \texttt{update} operation in \texttt{Enrolment}
is much better as it does not change the primary keys of these entities,
instead, since only the foreign keys are changed. Thus, this operation on
\texttt{Enrolment} acts as an \texttt{insert} operation.

%The performance of the operations when referential integrity validations are
%introduced  is compared with a baseline experiment where such operations  do not
%trigger any such validations.  This baseline serves as a reference to
%determine the difference in performance of the \ac{DBMS} when validations are
%imposed using the \ac{API} and to analyse the performance of the solutions.

% In the baseline experiment,  the operations on the entities represent how
% operations on data are performed in Cassandra without referential integrity
% validations. In order to be consistent with the solutions, the operations
% \texttt{Create}, \texttt{Update} and \texttt{Delete} for the baseline are
% measured in the same way operations are measured for the solutions and the
% artificial data for the baseline experiment is created the same way as well.
% Note that, \texttt{Create} inserts all the entities for \texttt{Student},
% \texttt{Course} and \texttt{Enrolment}.  \texttt{Update} performs changes on the
% primary keys of \texttt{Student} and \texttt{Course} entities,  and on the
% foreign keys (\texttt{CourseId}) of \texttt{Enrolment} while \texttt{Delete}
% removes all the \texttt{Student},  \texttt{Course} and \texttt{Enrolment}
% entities. These operations are performed precisely the same way as the
% solutions, so that the comparison and analysis of results are balanced and
% unbiased.
% 
% The results in terms of average response time and throughput for the baseline
% experiment are presented as  bar-plots in Figure~\ref{fres:Baseline}. 
% Specifically,  Figure~\ref{fres:Baseline-responsetime} shows the response time 
% of each operation on a single entity in the three column families and 
% Figure~\ref{fres:Baseline-throughput} presents the throughput of each operation
% on the same column families within one second. 
% % The analysis of the performance of each operation on an entity is discussed as
% % follows. 

% 	\begin{figure}[h] \centering
% 	\includegraphics[width=. 8\textwidth]{. /figure/result/barplot-Baseline-rt.pdf}
% 		\caption{Baseline}\label{fr:Solution0-barplot}
% 	\end{figure}
	


% As seen from these results,  the \texttt{delete} operation takes in average the
% lowest response time and the highest throughput.  That is,  the
% time to delete an entity is shorter when compared to \texttt{insert} or \texttt{update}, 
% which in turn means that more \texttt{delete} operations can be performed within
% a second.  On the other hand,  \texttt{update} takes the most time to complete all
% the operations thus providing a low throughput while \texttt{insert} is
% faster than \texttt{update} and quite similar to \texttt{delete}. 
% 
% % The \texttt{delete} operation performs faster than the other operations
% % because Cassandra performs  tombstone deletes,  where data is not physically
% % but only logically marked as deleted.
% The \texttt{update} operation takes the most time because it
% % because it requires searching by index to access the correct columns in the
% % column family.  Moreover,   an \texttt{update}
% involves  \texttt{insert} and
% \texttt{delete} operations in the cases of \texttt{Course} and \texttt{Student}
% hence taking more time that the other operations. However, the
% \texttt{update} in \texttt{Enrolment} takes similar time to other
% operations because it does not change the primary keys of these entities,
% instead, only the foreign keys are changed.  % On the other hand,
% % \texttt{update} on \texttt{Student} and \texttt{Course} entities changes the
% % primary keys,  which involves performing an \texttt{insert} and a
% % \texttt{delete} operation for each entity.
% % The \texttt{insert} operation takes slightly more time than \texttt{delete}
% % since data has to be physically written into the column families.
% 
% The time taken to insert and delete entities from \texttt{Student}, 
% \texttt{Course} and \texttt{Enrolment} are rather similar since there are no
% referential integrity validations.  However, the slight differences might be 
% due to differences in number of columns,  size of contents or even external
% factors.








% ब
\section{Insert} \label{s:results-insert}
% \newcommand{\Width}{0. 5\textwidth} \newcommand{\TB}[1]{\textbf{#1}} An
% \texttt{insert} operation triggers a referential integrity validation whenever
% a child entity containing foreign keys is inserted where the
% \texttt{ValidationHandler} validates that the foreign keys exist as primary
% keys in the parent entities.
Across all the solutions in the experiments,   \texttt{insert}  triggers
a validation when \texttt{Enrolment} entities are inserted as it is a child entity
containing foreign keys referencing to \texttt{Student} and \texttt{Course}.
On the other hand,  \texttt{Student} and \texttt{Course} entities have no
referential integrity constraints to be checked as these do not contain any
references to other column families.

The average response time and throughput for completing an \texttt{insert} on a
single entity for all the solutions is presented in Figure~\ref{fres:Insert}.
Specifically, Figure~\ref{fres:Insert-responsetime} shows the average time
consumed by the baseline and each solution to complete an \texttt{insert}
operation,  and Figure~\ref{fres:Insert-throughput} shows the  the number of
\texttt{insert} operations that can be completed in one second.
\newpage
	\begin{figure}[H]
		\subfigure[Response time for Insert operation]
		{\includegraphics[width=\Width]{figure/result/barplot-insert-rt.pdf}\label{fres:Insert-responsetime}}
		\subfigure[Throughput for Insert operation]
		{\includegraphics[width=\Width]{figure/result/barplot-insert-tp.pdf}\label{fres:Insert-throughput}}
		\caption{Performance of Solutions in Insert}\label{fres:Insert}
	\end{figure}
	
These results show that \texttt{insert} on a single entity of \texttt{Student}
and \texttt{Course} take approximately the same time to complete, and this is
consistent across solutions.  Inserting \texttt{Student} and \texttt{Course}
entities into their respective column families is faster than \texttt{insert} in
\texttt{Enrolment} because these are parent column families that have no
referencing constraints. Thus, the \texttt{insert} operation on these entities
involves only accessing the relevant \ac{FK} constraints from the metadata in
order to determine whether it is a parent or child entity.
% If the entity is a parent,  the validations are not triggered which is the
% case for \texttt{Student} and \texttt{Course} entities.
On the other hand,  \texttt{insert} on  \texttt{Enrolment}  takes the most time
in all the solutions as these entities have existing \ac{FK} constraints
indicating that they reference a parent entity which requires to retrieve
additional constraints. Moreover, its validation involves not only identifying
its relevant constraints but also accessing its parent column families
(\texttt{Student} and \texttt{Course}) to ensure that foreign keys match primary
keys. The results highlight the difference in response time when
foreign key validations are required in the case of \texttt{Enrolment}.
% the validation of one \texttt{Enrolment} entity and for other entities that do
% not have validations.
Note that these observations stand true across all the solutions.

More detailed information about the performance of each solution when
\texttt{insert} operations are performed is presented in
Figures~\ref{fres:insert-response-time} and~\ref{fres:insert-throughput}.
These figures show the average response time and throughput for the
\texttt{insert} operation on  each entity individually.
% Figure~\ref{fres:insert-user} presents the results for \texttt{insert} on a
% single \texttt{Student} entity in all the solutions.  Similarly
% Figures~\ref{fres:insert-course} and~\ref{fres:insert-enrolment} show the
% performance of \texttt{insert} on a \texttt{Course} and \texttt{Enrolment}
% entity in the solutions.
It can be seen that Solution~4 takes the least time to complete an 
\texttt{insert} on all the entities while Solution~3 takes the most time. 
Solution~4 takes the least time since it caches all the metadata thus avoiding 
multiple accesses to the \texttt{Metadata} column family, whereas Solution~3 
requires accessing \texttt{Metadata} each time a constraint is required. 
Regarding Solutions~1 and 2,  both perform similarly although Solution~2 takes 
slightly  more time than Solution~1 due to its additional search operation to 
locate the top row. Both the solutions are slightly slower than Solution~4 as 
constraints from these solutions are retrieved from the column family and  not 
from a cache. However, both solutions are faster than Solution~3 since 
retrieving constraints require no additional connections to access the metadata.




When compared to the baseline,  it is clear that the referential integrity
validations as well as metadata access caused the increased response time for
\texttt{insert} in all the solutions.  Since the validations are the same for
all solutions,  the performance differences in the solutions are due to the
different ways of accessing and processing the metadata.  From
Table~\ref{tres:ResponsetimeRatio}, Solutions~1 and 2  are almost 4 times slower
than the baseline, while Solution~3 is more than 11 times slower, and 
Solution~4 is almost 3 times slower than the baseline.
% Solution~3 takes the most time to perform one \texttt{insert} on all the
% entities.

Notice that when no referential integrity constraints need to be satisfied (e.g.
\texttt{Student} and \texttt{Course}), Solutions~1 and 2 are nearly 2 times
slower than the baseline,  Solution~3   more than 5 times slower, and  
Solution~4  almost similar to the baseline. Such differences are due to the
computational cost incurred while retrieving the metadata. 
% Note that the differences in the performance of the solutions is due to the
% distinctive way in which these store and acess metadata, as explained previously.
% A possible reason for this is that baseline operations are affected by the
% initialization as its \texttt{insert} operations are the very first operations
% to be executed.

\begin{landscape}
		\begin{figure}
		\centering
		\newcommand{\W}{.4\textwidth}
			\subfigure[Insert on Student]
			{\includegraphics[width=\W]{figure/result/barplot-insert_student-rt.pdf}
			\label{fres:insert-user}}
			\subfigure[Insert on Course]
			{\includegraphics[width=\W]{figure/result/barplot-insert_course-rt.pdf}
			\label{fres:insert-course}}
			\subfigure[Insert on Enrolment]
			{\includegraphics[width=\W]{figure/result/barplot-insert_enrolment-rt.pdf}}
			\caption{Response time inserting entities}\label{fres:insert-response-time}
			
			\subfigure[Insert on Student]
			{\includegraphics[width=\W]{figure/result/barplot-insert_student-tp.pdf}}
			\subfigure[Insert on Course]
			{\includegraphics[width=\W]{figure/result/barplot-insert_course-tp.pdf}}			
			\subfigure[Insert on Enrolment]
			{\includegraphics[width=\W]{figure/result/barplot-insert_enrolment-tp.pdf}}
			\caption{Throughput inserting entities}\label{fres:insert-throughput}
		\end{figure}
		
\end{landscape}
 
%ब 

\section{Update} \label{s:results-update}
The \texttt{update} operation triggers  referential integrity validations
whenever entities of \texttt{Student},  \texttt{Course} and \texttt{Enrolment}
are updated with new values. Notice that the \texttt{Update} operation is
performed  on the primary keys of \texttt{Student} and \texttt{Course} entities,
 and on the foreign keys (\texttt{CourseId}) of \texttt{Enrolment}.
Figure~\ref{fres:Update} presents the results of the \texttt{update} operation
on each entity for all the solutions.
Figures~\ref{fres:Update-responsetime} and~\ref{fres:Update-throughput} present
the average response time and throughput of \texttt{update} on each entity in
all the solutions.

	\begin{figure}[H] 
		\subfigure[Response time for Update operation]
		{\includegraphics[width=\Width]{figure/result/barplot-update-rt.pdf}\label{fres:Update-responsetime}}
		\subfigure[Throughput for Update operation]
		{\includegraphics[width=\Width]{figure/result/barplot-update-tp.pdf}\label{fres:Update-throughput}}
		\caption{Performance of Solutions in Update}\label{fres:Update}
	\end{figure}
 
It can be seen from the results that the \texttt{update} operation on
\texttt{Enrolment} is  faster than \texttt{update} on a \texttt{Student} or
\texttt{Course} in all the solutions. 
The \texttt{update} operation  on a \texttt{Student} entity takes the most time
in all the solutions. 
The \texttt{update} operation on a \texttt{Course} entity always takes more time
than \texttt{update} on \texttt{Enrolment} but it is faster than updating
\texttt{Student} entities. The differences in  performance of 
\texttt{update} on the  entities is because of the referential integrity
rules that are applied during validation. 


Updating \texttt{Enrolment} is faster since before inserting the new values,  it
only  involves  identifying relevant \ac{FK} constraints in the metadata and 
accessing the parent column families \texttt{Student} and \texttt{Course} to
ensure that the new foreign key values exist.  Moreover,  \texttt{update} in
\texttt{Enrolment} involves changing the foreign key attributes and not the
primary key column. 

Updating \texttt{Student} is slower since the primary key is changed and it is a
cascaded operation that updates \texttt{Enrolment}.  After accessing the
relevant \ac{FK} constraints,  the child dependencies are retrieved from
\texttt{Enrolment} and updated with the new value for the \texttt{StudentId}.
This means that an \texttt{update} accesses the metadata as well as the child
column family and performs \texttt{insert} in \texttt{Student} and the child
column family \texttt{Enrolment} with the new values.  This operation also
involves performing a \texttt{delete} to remove the old primary key value in
\texttt{Student}.


Updating \texttt{Course} takes less time than \texttt{update} on
\texttt{Student} because it is not a cascaded operation as the
\texttt{DeleteRule} for \texttt{Course} entities is \texttt{NoDelete}\footnote{Notice that for the
  sake of simplicity, this rule is also used for update operations.} and
\texttt{Enrolment} contains its child dependencies.  Thus,  exceptions are
raised each time an \texttt{update} is performed on \texttt{Course} entities, 
which means that the response time includes measuring the time for validation as
well as raising exceptions.  But this is slower than \texttt{update} on
\texttt{Enrolment} because it involves accessing the \texttt{Enrolment} column
family to identify existing child dependencies.  Since these child dependencies
exist when the experiments are run,  the exceptions are raised each time.

% The time involved for an \texttt{update} on a \texttt{Course} entity is  more as
% seen in Figures~\ref{fres:Update} and~\ref{fres:update-course}.  In this
% operation the relevant \texttt{FK} constraints and child dependencies of the
% \texttt{Course} entity are identified and its  relevant \texttt{DeleteRule}
% is also determined. 
% Since the \texttt{DeleteRule} for \texttt{Course} entities is \texttt{NoDelete}
% an exception is raised.  Moreover,  \texttt{Enrolment} column family is accessed
% to identify existing child dependencies.  These additional operations and the
% exceptions raised make \texttt{update} on \texttt{Course} consume more time to
% complete. 
% 
% The results in Figures~\ref{fres:Update} and~\ref{fres:update-user} show that
% updating a \texttt{Student} entity takes the most time in all the solutions. 
% The relevant \ac{FK} constraints are accessed for this entity and
% its child dependencies are identified from \texttt{Enrolment},  which is similar
% to \texttt{update} on \texttt{course} entities.  However,  update on a
% \texttt{Student} entity is cascaded and involves updating all the child
% dependencies in \texttt{Enrolment} column family.  This means that an
% \texttt{update} causes writes not only in \texttt{Student} but also in the child
% column family \texttt{Enrolment}.  
Note that \texttt{update} on \texttt{Student} entities causes values to be
updated in two column families,  while on \texttt{Enrolment} values are
updated in only one column family and on \texttt{Course} no values are updated.  


Further details of the performance of each solution when the \texttt{update}
operation is executed on each entity is presented in
Figures~\ref{fres:update-response-time} and~\ref{fres:update-throughput}.
These results show that Solution~4 is the fastest amongst all the solutions, 
while Solution~3 is the slowest.
Solutions~1 and 2 perform almost similarly although the additional search for
the top row in Solution~2 makes it just slightly slower than Solution~1.
Note that in Solution~3 the \texttt{Metadata} column family is accessed multiple
times in each validation making it the slowest.  Multiple accesses are needed in
order to first retrieve the relevant \ac{FK} constraints and then to retrieve
information about the child or parent entities.  Although Solution~4 stores
metadata separately like Solution~3,  it caches and re-uses
% the \texttt{Metadata} column family and reuses the cache to avoid such multiple
% accesses to the column family.  
the list of constraints and avoids connecting to the external cluster to access
the \texttt{Metadata} column family each time  operations are invoked on entities.

When compared to the baseline,  the \texttt{update} operation on all the entities
take considerably more time in all the solutions because of their different
metadata storage designs.  The validations and metadata access for parent
entities make Solutions~1 and 2  almost more than 26 times slower than baseline. 
Solution~3 is almost 60 times slower than baseline while Solution~4 is only 20
times slower than the baseline  in such updates.  

However,  updates on child entities
make  Solution~1 and 2  more than 3 times slower than the baseline.  Solution~3
is nearly 8 times slower while 
Solution~4 is only 2 times slower than the baseline in such updates. 

\begin{landscape}
		\begin{figure}
		\centering
		\newcommand{\W}{.4\textwidth}
			\subfigure[Update on Student]
			{\includegraphics[width=\W]{figure/result/barplot-update_student-rt.pdf}}
			\subfigure[Update on Course]
			{\includegraphics[width=\W]{figure/result/barplot-update_course-rt.pdf}}
			\subfigure[Update on Enrolment]
			{\includegraphics[width=\W]{figure/result/barplot-update_enrolment-rt.pdf}}
			\caption{Response time updating entities}\label{fres:update-response-time}
			
			\subfigure[Update on Student]
			{\includegraphics[width=\W]{figure/result/barplot-update_student-tp.pdf}}			
			\subfigure[Update on Course]
			{\includegraphics[width=\W]{figure/result/barplot-update_course-tp.pdf}}
			\subfigure[Update on Enrolment]
			{\includegraphics[width=\W]{figure/result/barplot-update_enrolment-tp.pdf}}
			\caption{Throughput updating entities}\label{fres:update-throughput}
		\end{figure}
 \end{landscape}
%ब 
\section{Delete} \label{s:results-Delete}

The \texttt{delete} operation triggers referential integrity validations
whenever entities are deleted.
Figure~\ref{fres:Delete} presents the results of the \texttt{delete} operation
on each entity for all the solutions.
Specifically, Figure~\ref{fres:Delete-responsetime} shows the average
response time to perform a single \texttt{delete} on each entity according to
each solution and Figure~\ref{fres:Delete-throughput} presents the
respective throughput for this operation.

	\begin{figure}[H] 
	\newcommand{\W}{.5\textwidth}
		\subfigure[Response time for Update operation]
		{\includegraphics[width=\W]{figure/result/barplot-delete-rt.pdf} \label{fres:delete-}\label{fres:Delete-responsetime}}
		\subfigure[Throughput for Update operation]
		{\includegraphics[width=\W]{figure/result/barplot-delete-tp.pdf} \label{fres:delete-}\label{fres:Delete-throughput}}
		\caption{Performance of Solutions in Update}\label{fres:Delete}
	\end{figure}
 
The results show that the \texttt{delete} operation on \texttt{Enrolment}
is the fastest in all the solutions. Deleting \texttt{Enrolment} entities is
faster as these have no referential integrity constraints to satisfy since
\texttt{Enrolment} has no child dependencies in it.
Nonetheless, this operation is  slower than the baseline in all the solutions
because it involves accessing metadata to retrieve the relevant constraints of
\texttt{Enrolment} in order to determine if any child dependencies exist or
not.

The \texttt{delete} operation on \texttt{Student} is the slowest. Deleting
\texttt{Student} entities is a cascaded operation which involves deleting the
child entities in the \texttt{Enrolment} column family. This operation is the
slowest because  the  child entities in \texttt{Enrolment}  which have a
reference to  \texttt{Student} have to be deleted first.

Finally, the \texttt{delete} operation on \texttt{Course} is faster than
deleting \texttt{Student} entities. Deleting \texttt{Course} entities also
involves accessing the relevant constraints and finding the child dependencies
in \texttt{Enrolment}. However, at this stage,  all the entities in
\texttt{Enrolment} are already deleted before \texttt{delete} is invoked on
\texttt{Course} entities. Hence, \texttt{delete} in \texttt{Course} actually
deletes the entities as there are no existing child dependencies.

% Notice that \texttt{delete} on \texttt{Course} involves the time to access
% metadata as well as \texttt{Enrolment} in order to search for any existing child
% dependencies and values are deleted from a single column family. However,
% \texttt{delete} on \texttt{Student} entities involve deleting values from two
% column families (\texttt{Student} and \texttt{Enrolment}). This makes
% \texttt{delete} in \texttt{Course} faster than \texttt{delete} in
% \texttt{Student}.

% As seen in \texttt{update}, the difference in  performance of the operation
% on the different entities is because of the referential integrity rules on
% child and parent entities. 





More detailed inforation about the performance of this operation can be seen in
Figures~\ref{fres:delete-response-time} and~\ref{fres:delete-throughput}.
It can be seen from these results that Solution~4 takes the least time to
complete a \texttt{delete} operation on each entity, while Solution~3 takes the
most time. Since Solution~4 caches the metadata of all the entities, it avoids
multiple accesses to the \texttt{Metadata} column family, whereas Solution~3
requires accessing \texttt{Metadata} each time a constraint has to be accessed
for an entity. The performance of Solutions~1 and 2 are comparable to each other
even though Solution~2 takes slightly more time due to its additional search
operation to locate the top row.



When compared to the baseline, all the solutions take longer to delete entities.
As mentioned previously, this is because all the solutions involve accessing
relevant constraints and performing referential integrity validations. In this
operation, when entities have child dependencies (\texttt{Student} and
\texttt{Course}), Solutions~1 and 2 are more than 23 times slower than the
baseline, Solution~3 almost 80 times slower and Solution~4   up to 17 times
slower than the baseline. On the other hand, deletes on child entities make
Solutions~1 and 2  almost 2 times slower than baseline, Solution~3  almost 7
times slower ,while Solution~4 is almost similar to the baseline, which shows
that accessing the metadata does not cause much difference in the performance.


\begin{landscape}
		\begin{figure}
		\centering
		\newcommand{\W}{.4\textwidth}
			\subfigure[Delete on Student]
			{\includegraphics[width=\W]{figure/result/barplot-delete_student-rt.pdf}
			\label{fres:delete-user}}
			\subfigure[Delete on Course]
			{\includegraphics[width=\W]{figure/result/barplot-delete_course-rt.pdf}
			\label{fres:delete-course}}
			\subfigure[ Delete on Enrolment ]
			{\includegraphics[width=\W]{figure/result/barplot-delete_enrolment-rt.pdf}
			\label{fres:delete-enrolment}}
			\caption{Response time deleting entities}\label{fres:delete-response-time}
						
			\subfigure[Delete on Student]
			{\includegraphics[width=\W]{figure/result/barplot-delete_student-tp.pdf} \label{fres:delete-}}
			\subfigure[Delete on Course]
			{\includegraphics[width=\W]{figure/result/barplot-delete_course-tp.pdf} \label{fres:delete-}}
			\subfigure[Delete on Enrolment]
			{\includegraphics[width=\W]{figure/result/barplot-delete_enrolment-tp.pdf} \label{fres:delete-}}
			\caption{Throughput deleting entities}\label{fres:delete-throughput}
		\end{figure}
\end{landscape}


% Moreover,   when the experiments are run \texttt{insert} on these entities are
% the first operations to take place and the results can be slightly influenced
% by the initialisation of the column families and the keyspace.   However,   the
% difference is small and only a fraction of a millisecond. 
% the number  of columns that are updated in \texttt{Enrolment} is much lesser
% than the other column families.   This means that \texttt{update} on
% \texttt{Enrolment} involves fewer search by indexes and writes for the new
% values,   to be precise it involves only three searches and writes to update
% its three columns.    On the other hand,  \texttt{Student} and \texttt{Course}
% column families have more number of columns to be updated in each
% \texttt{update} operation. 

\section{Comparison of the Operations} \label{s:comparisonOfOperations}
 
In order to compare the operations,  their performance  is grouped by solutions
and presented in Figures~\ref{fres:ResponseTimeOfSolutions}
and~\ref{fres:ThroughputOfSolutions}. 
Generally,  the \texttt{insert} operation takes the least time across the
solutions as it does not involve any cascaded operations and referential
integrity constraints have to be satisfied only for the child entities.  This
involves ensuring the existence of foreign keys as primary keys in the parent
column families and then inserting the child entities into their respective
column families. 

On the other hand,   the \texttt{update} operation takes the most time  in every
solution,   mainly due to its cascaded behaviour on parent entities,   which
involves changing the parent primary key,   accessing child column families and
changing its foreign key values.  Note that \texttt{update} on \texttt{Enrolment}
is similar to \texttt{insert} on \texttt{Enrolment} across the solutions, 
because both operations involve checking whether the foreign keys exist in the
parent column families and inserting the values only in \texttt{Enrolment}. 
On the other hand,   \texttt{update} on parent entities (\texttt{Student} and
\texttt{Course})  take more time than inserting parent entities because
\texttt{update} involves additional searches and operations (\texttt{insert} and
\texttt{delete}) in both the parent and child column families. 

The \texttt{delete} operation is slower than the \texttt{insert} in the case of
parent entities and faster than \texttt{insert} in the case of child entities. 
This is because  referential integrity constraints have to be satisfied only in
the case of the  parent entities for this operation. 
% However,  inserting child entities require referential integrity constraints to
% be satisifed which makes \texttt{delete} faster than \texttt{insert} in the
% case of child entities. 
When \texttt{delete} and \texttt{update} operations are compared,  it can be seen
that the ratio of their performances  on the entities are similar,  which is
because both these operations are cascaded on parent entities. 
For example,  \texttt{update} on \texttt{Student} takes the most time amongst
entities,  and similarly,  \texttt{delete} on \texttt{Student} takes the most time
amongst the entities.  However,  in general,  deleting entities is faster than
updating them.  This is because updating entities involves more operations as
both \texttt{insert} and \texttt{delete} are performed on  child and
parent column families,  while deleting entities involves inserting empty values
in the place of the entity attributes to mark them as deleted (tombstone
effect).  Moreover,  in \texttt{update},  referential integrity constraints
have to satisfied for both parent and child entities,  but in
\texttt{delete},  these have to be satisfied only for the parent entities. 

Further information about the results for the operations and solutions are
provided in Appendices~\ref{apx:insert},  \ref{apx:update},  and
~\ref{apx:delete}. 


% Conversely,  deleting child entities is faster than inserting child entities
% because referential integrity constraints. 
%  
%  inserting parent entities do not cause referential integrity validations,   but
%  deleting them does.   Conversely,   deleting child entities is faster than
% inserting child entities since it does not trigger such validations. 

% 
% Across all solutions,  the \texttt{insert}
% operation is the one that has best performance when it comes to \texttt{Student}
% and \texttt{Course} as these entities do not have referential integrity
% constraints to be satisfied.  Such is not the case for \texttt{Enrolment}
% entities as these have foreign key values that reference entities in
% \texttt{Student} and \texttt{Course}.  Hence,  the referential integrity
% validation ensures that 
% 
% 
% 
% From these figures it can be seen that the
% \texttt{insert} operation  takes the least time to complete when compared
% to \texttt{update} and \texttt{delete} operations.   This is mainly because in
% \texttt{insert},  validations are triggered on only the \texttt{Enrolment} column
% family. 
% 
% On the other hand,   the \texttt{update} operation takes the most time  in every
% solution,   mainly due to its cascaded behaviour on parent entities,   which
% involves changing the parent primary key,   accessing child column families and
% changing its foreign key values. 
% Note that \texttt{update} on \texttt{Enrolment} is similar to \texttt{insert} on
% \texttt{Enrolment} because both operations involve checking whether the foreign
% keys exist in the parent column families and inserting the values only in
% \texttt{Enrolment}. 
% However,   \texttt{update} on \texttt{Student} and  \texttt{Course} entities take
% more time than inserting students and courses because \texttt{update} involves
% additional searches and operations (\texttt{insert} and \texttt{delete})
% in more than one column family.  
% 
% The \texttt{delete} operation is faster than \texttt{update} in all the
% solutions since entities are not immediately deleted due to the tombstone
% paradigm in Cassandra.   Moreover,  it involves only a single operation
% unlike the \texttt{update} operation which causes both an \texttt{insert} and a
%  \texttt{delete}.   Moreover,  deleting child entities do not cause validations
%  but updating any entity causes validations. 

% When compared with the \texttt{insert} operation,   \texttt{delete} is slower in
% the case of parent entities,   \texttt{Student} and \texttt{Course}.   This is
% because inserting parent entities do not cause referential integrity
% validations,   but deleting them does.   Conversely,   deleting child entities is
% faster than inserting child entities since it does not trigger such validations. 


\begin{landscape}

		\begin{figure}
		\newcommand{\W}{.345\textwidth}
		\centering
			\subfigure[Solution 1]
			{\includegraphics[width=\W]{figure/result/barplot-Solution1-rt.pdf}
			\label{fres:Summary-Solution1}}
			\subfigure[Solution 2]
			{\includegraphics[width=\W]{figure/result/barplot-Solution2-rt.pdf}
			\label{fres:Summary-Solution2}}
			\subfigure[Solution 3]
			{\includegraphics[width=\W]{figure/result/barplot-Solution3-rt.pdf}
			\label{fres:Summary-Solution3}}
			\subfigure[Solution 4]
			{\includegraphics[width=\W]{figure/result/barplot-Solution4-rt.pdf}
			\label{fres:Summary-Solution4}}
			\captionof{figure}{Response Time of the
			Solutions}\label{fres:ResponseTimeOfSolutions}
			
			\subfigure[Solution 1]
			{\includegraphics[width=\W]{figure/result/barplot-Solution1-tp.pdf}
			\label{fres:Summary-Solution1}}
			\subfigure[Solution 2]
			{\includegraphics[width=\W]{figure/result/barplot-Solution2-tp.pdf}
			\label{fres:Summary-Solution2}}
			\subfigure[ Solution 3]
			{\includegraphics[width=\W]{figure/result/barplot-Solution3-tp.pdf}
			\label{fres:Summary-Solution3}}
			\subfigure[ Solution 4]
			{\includegraphics[width=\W]{figure/result/barplot-Solution4-tp.pdf}
			\label{fres:Summary-Solution4}}
			\caption{Throughput of the Solutions}\label{fres:ThroughputOfSolutions}
		\end{figure}
\end{landscape}


\section{Summary} \label{s:results-summary}

This chapter presented the results and discussions from the experiments designed
to evaluate the performance of the different \ac{CRUD} operations under
different referential integrity constraints in each of the solutions.
The results were assessed in terms of the average response time and throughput
of each operation. The results reflected that Solution~4 performs the best
amongst the solutions, and performs similar to the baseline when no referential
integrity constraints need to be satisfied (e.g. inserting parent entities),
because it caches the metadata and re-uses it to avoid multiple access to the
\texttt{Metadata} column family.
Solution~3 performs the worst amongst all and is slower than the baseline even
when no referential integrity constraints need to be satisfied because
simply accessing the metadata from a separate column family each time affects
its performance.
Solutions~1 and 2 perform similarly in all the operations on the entities which
is mainly because the metadata is embedded with the actual data.   Solution~2
consumes slightly more time than Solution~1 as it searches for the top row to
identify constraints on each operation.

The results showed that amongst the operations,   \texttt{insert} took the least
time while \texttt{update} took the most time,   and \texttt{delete} was faster
than \texttt{insert} only in the case of child entities.   These variations were
mainly due to the different referential integrity rules that are applied on
parent and child entities,   especially because of the \texttt{DeleteRule}
applied on these entities.  

The entities required different behaviours in each operation due to the various
referential integrity rules as well as  the data manipulation rules applied on
them.
\texttt{Enrolment} entities required to satisfy referential integrity
constraints during \texttt{insert} and \texttt{update} operations as it is a child entity,
while \texttt{Student} and \texttt{Course} are parent entities and required to
satisfy these constraints in both \texttt{update} and \texttt{delete}.   Thus,
parent entities are faster to operate upon in an \texttt{insert} operation,  
while child entities are faster only in a \texttt{delete} operation.
	
