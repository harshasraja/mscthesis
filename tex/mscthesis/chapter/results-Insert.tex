%ब 
\section{Insert} \label{s:results-insert}
% \newcommand{\Width}{0. 5\textwidth}
% \newcommand{\TB}[1]{\textbf{#1}}
% An \texttt{insert} operation triggers a referential integrity validation
% whenever a child entity containing foreign keys is inserted where the
% \texttt{ValidationHandler} validates that the foreign keys exist as primary keys
% in the parent entities. 
Across all the solutions in the experiments,   \texttt{insert}  triggers a
validation when \texttt{Enrolment} entities are inserted as it is a child entity
containing foreign keys referencing to \texttt{Student} and \texttt{Course}. 
However,  \texttt{Student} and \texttt{Course} entities do not trigger
validations as these do not contain any references to other column families.  

The average response time and throughput for completing an \texttt{insert} on a
single entity for all the solutions is presented in Figure~\ref{fres:Insert},  
where  the average time consumed by the baseline and each solution to complete one \texttt{insert} operation is presented in 
Figure~\ref{fres:Insert-responsetime} and 
 the number of \texttt{insert}
operations that can be completed in one second across the solutions is
presented in  Figure~\ref{fres:Insert-throughput}. 

	\begin{figure}[H]
		\subfigure[Response time for Insert operation]
		{\includegraphics[width=\Width]{figure/result/barplot-insert-rt.pdf}\label{fres:Insert-responsetime}}
		\subfigure[Throughput for Insert operation]
		{\includegraphics[width=\Width]{figure/result/barplot-insert-tp.pdf}\label{fres:Insert-throughput}}
		\caption{Performance of Solutions in Insert}\label{fres:Insert}
	\end{figure}
	
These results show
% and the results in Figures~\ref{fres:insert-user} and~\ref{fres:insert-course}
that \texttt{insert} on a single entity of \texttt{Student} and \texttt{Course}
take approximately the same time to complete and this is consistent across
solutions.  Inserting \texttt{Student} and \texttt{Course} entities into their
respective column families is faster than \texttt{insert} in \texttt{Enrolment}
as these are parent column families and do not trigger any validation. 
The \texttt{insert} operation on these entities involves only accessing the
relevant \ac{FK} constraints from the metadata in order to determine whether it
is a parent or child entity.  
% If the entity is a parent,  the validations are not
% triggered which is the case for \texttt{Student} and \texttt{Course} entities. 
On the other hand,  \texttt{insert} on a single \texttt{Enrolment} entity takes
the most time in all the solutions as these entities have existing \ac{FK}
constraints indicating that they reference a parent entity which in turn triggers
referential integrity validations.  Therefore,   its validation involves not only
identifying its relevant constraints but also accessing its parent column
families (\texttt{Student} and \texttt{Course}) to ensure the existence of the
foreign keys. 
The results highlight the difference in response time when validation is
triggered in the case of \texttt{Enrolment}. 
% the validation of one \texttt{Enrolment} entity and for other entities that do
% not have validations. 
Note that these observations stand true across all the solutions. 

More information about the performance of each solution when an \texttt{insert}
operation is executed on each entity is presented in
Figures~\ref{fres:insert-response-time}
and~\ref{fres:insert-throughput}. 
These figures show the average response time and throughput for the
\texttt{insert} operation on  the  entities in every solution. 
% Figure~\ref{fres:insert-user} presents the results for \texttt{insert} on a
% single \texttt{Student} entity in all the solutions.  Similarly
% Figures~\ref{fres:insert-course} and~\ref{fres:insert-enrolment} show the
% performance of \texttt{insert} on a \texttt{Course} and \texttt{Enrolment}
% entity in the solutions.  
It can be seen that Solution~4 takes the least time to
complete an \texttt{insert} on all the entities while Solution~3 takes the most
time.  Regarding Solutions~1 and 2,  both perform similarly and are slightly
slower than Solution~4.  

When compared to the baseline,  it is clear that the referential integrity
validations as well as metadata access caused the increased response time for
\texttt{insert} in all the solutions.  Since the validations are the same for all
solutions,  the performance differences in the solutions are due to the different
ways of accessing and processing the metadata.  From Table~\ref{tres:ResponsetimeRatio}
Solutions~1 and 2  are almost 4 times slower than the
baseline while Solution~3 is more than 11 times
slower.  Solution~4 is almost 3 times slower than the baseline to perform the
validations on \texttt{insert}.  
% Solution~3 takes the most time to perform one \texttt{insert} on all the
% entities. 

However,  when no validations are triggered (\texttt{Student} and
\texttt{Course}),   Solutions~1 and 2 are nearly 2 times slower than the baseline
while Solution~3 is  more than 5 times slower. 
Solution~4 performs almost similar to the baseline  despite having metadata
access (Figures~\ref{fres:insert-user} and~\ref{fres:insert-course}).  
% A possible reason for this is that baseline operations are affected by the
% initialization as its \texttt{insert} operations are the very first operations
% to be executed. 

\begin{landscape}
		\begin{figure}
		\centering
		\newcommand{\W}{.4\textwidth}
			\subfigure[Insert on Student]
			{\includegraphics[width=\W]{figure/result/barplot-insert_student-rt.pdf}
			\label{fres:insert-user}}
			\subfigure[Insert on Course]
			{\includegraphics[width=\W]{figure/result/barplot-insert_course-rt.pdf}
			\label{fres:insert-course}}
			\subfigure[Insert on Enrolment]
			{\includegraphics[width=\W]{figure/result/barplot-insert_enrolment-rt.pdf}}
			\caption{Response time inserting entities}\label{fres:insert-response-time}
			
			\subfigure[Insert on Student]
			{\includegraphics[width=\W]{figure/result/barplot-insert_student-tp.pdf}}
			\subfigure[Insert on Course]
			{\includegraphics[width=\W]{figure/result/barplot-insert_course-tp.pdf}}			
			\subfigure[Insert on Enrolment]
			{\includegraphics[width=\W]{figure/result/barplot-insert_enrolment-tp.pdf}}
			\caption{Throughput inserting entities}\label{fres:insert-throughput}
		\end{figure}
		
\end{landscape}
