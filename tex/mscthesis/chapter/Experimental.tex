%ब
\chapter{Experimental Design}

\todo{Check number of enrolments,  introduce all metadata,  and retrieve 
details of Cassandra cluster from my computer}

% The implementation of the four solutions  introduces referential integrity
% constraints and validations in Cassandra and each solution performs
% differently . 
In order to evaluate the performance of the four solutions when deployed on
Cassandra,  experiments were conducted by implementing the experimental \ac{API}. 
The goal of the experiments were to determine how referential integrity
validations and different metadata storage techniques affected Cassandra in its
performance. 

 In the experiments,  the experimental \ac{API} is loaded with entities
 belonging to a prototype keyspace  on which the \ac{CRUD} operations are
 performed.  Each \ac{CRUD} operation causes the \ac{API} to interact with a
 Cassandra cluster which is  a homogeneous group of multiple nodes with
 similar characteristics and configurations.  Each operation on an entity  is
 measured based on the performance indicators,  namely,  response time and
 throughput. 

The details of the experimental setup are described in the following sections. 
Section~\ref{sexp:BenchmarkKeyspace} describes the prototype keyspace used for
the experiments.  Section~\ref{sexp:CassandraCluster} provides the details of the
nodes used in the Cassandra cluster.  Section~\ref{sexp:PerformanceIndicators}
describe the performance indicators
 considered for measuring the results from the 
 experiments.  Section~\ref{sexp:ExperimentalSetup}
describes the details of the experimental setup.
Section~\ref{sexp:Summary} presents a summary of the chapter. 


\section{Benchmark keyspace} \label{sexp:BenchmarkKeyspace}
% In order to asses the performance of Cassandra when the four solutions are
% executed, 
In the experiments,  the experimental \ac{API} is loaded with
entities that belong to a prototype keyspace designed for these experiments. 
The prototype keyspace is modelled on a University keyspace that stores the details
of students and courses along with the enrolment details of the students. The
class diagram for the University keyspace is shown in
Figure~\ref{fexp:ClassDiagram}.  The entities are saved as the following column
families in Cassandra. 

	\begin{itemize}
	  \item \texttt{Student} to store the student attributes \texttt{StudentId}
	  (primary key),  \texttt{FirstName},  \texttt{LastName},  \texttt{Email} and
	  \texttt{Age}. 
	  \item \texttt{Course} to store course information with the attributes
	  \texttt{CourseId} (primary key),  \texttt{CourseName},  \texttt{Trimester}, 
	  \texttt{Level} and \texttt{Year}. 
	  \item \texttt{Enrolment} to preserve the relationship between students and
	  the courses they are enrolled in.  The attributes for \texttt{Enrolment} are
	  \texttt{RowId} (primary key),  \texttt{StudentId} and \texttt{CourseId}.  The
	  \texttt{StudentId} and \texttt{CourseId} are foreign keys. 
	\end{itemize}
	
	\begin{figure}[h] \centering
		\includegraphics[width=1\textwidth]{./figure/Solutions/classdiagram-experimental.png}
		\caption{Class Diagram for University}\label{fexp:ClassDiagram}
	\end{figure} 

The constraints of the keyspace are the \ac{PK} and \ac{FK} constraints
applicable on each of the column families in this keyspace.  The list of constraints are the same for
all the solutions  and are shown in Table~\ref{texp:ListConstraints}. 


\begin{table}[h] \label{texp:ListConstraints}
\centering
\caption{Metadata}	
	\newcolumntype{C}{@{\hspace{2.5pt}}>{\scriptsize}c@{\hspace{2.5pt}}}
	\begin{tabular}{CCC CCC CC}
		\toprule
		\bfseries ConstraintName & \bfseries Keyspace & \bfseries ConstraintType &
		\bfseries ColumnFamily & \bfseries RKeyspace & \bfseries RConstraintName &
		\bfseries RColumn & \bfseries DeleteRule\\
		\midrule
		CONST100 & University & P & Student & University & & StudentId &\\
		\rc CONST200 & University & P & Course & University & & CourseId &\\
		CONST300 & University & P & Enrolment & University & & RowId &\\
% 		\hline
% 		\hline
		\rc CONST400 & University & R & Enrolment & University & CONST100 & StudentId
		& CASCADE\\
		CONST500 & University & R & Enrolment & University & CONST200 & CourseId &
		NODELETE\\
		\rc CONST600 & University & F & Course & University & CONST500 & CourseId &
		NODELETE\\
		CONST700 & University & F & Student & University & CONST400 & StudentId &
		CASCADE\\
		\bottomrule 
	\end{tabular}
\end{table}

% The \texttt{ValidationHandler} in each solution checks these constraints to
% validate referential integrity within this keyspace.  The entities are loaded
% generically by the \texttt{EntityManager} for each solution.  
% In the experiment, 
% the number of entities inserted for each column family in all the solutions are
% shown in Table~\ref{texp:EntityList}.  
% 	
% 	\begin{table} \label{texp:EntityList}
% 	\centering
% 	\newcolumntype{C} {@{\hspace{2. 5pt}}>{\scriptsize}c@{\hspace{2. 5pt}}}
% 		\begin{tabular}{CC}
% 			
% 			\toprule
% 			\bfseries ColumnFamily & \bfseries No.  of Entities \\
% 			\midrule
% 			Student & 1000 \\
% 			\rc Course & 1000 \\
% 			Enrolment & 10000  \\
% 	% 		\hline
% 	% 		\hline
% 			
% 			\bottomrule
% 		\end{tabular}
% 	\end{table}


\section{Cassandra cluster} \label{sexp:CassandraCluster}

The environment to deploy Cassandra is  a homogeneous cluster conformed by 10
nodes.  That is,  all 10 nodes have the same characteristics in software and
hardware.  These nodes emulate a cloud environment in which each node runs
Cassandra and saves the data on the local disks of the machines.  The
characteristics of these nodes are:

	\begin{table}[H] \label{texp:Nodeconfig}
	\centering
	\newcolumntype{C} {@{\hspace{2.5pt}}>{\scriptsize}c@{\hspace{2.5pt}}}
		\begin{tabular}{CC}
			\toprule
% 			\bfseries System configurations\\
			
			Linux kernel version & Linux 3. 2. 4-1-ARCH i686 (62-bit)\\
			\rc CPU & Intel(R) Core(TM)2 Duo CPU     E8400  @ 3. 00GHz \\
			CPU cores & 4  \\
			\rc Network Card & 3 Gigabit \\
			Allocated memory & ?? \\
			\bottomrule
		\end{tabular}
	\end{table}
	
The nodes used in the cluster are a part of the ECS grid system of VUW and are
used locally by VUW  students from the ECS labs.  The experiments did not have
any control over the cluster environment and it was not possible to measure the
additional workload of these nodes at a given time as many grid jobs or remote
remote processes can run on them. 
In order to reduce such variables,  the experiments were performed over weekends
or during weeknights,  although most experiments were performed prior to the
trimester start which meant the number of users were the least during that
period.  

On all the nodes Cassandra was run remotely as a background process.  
% The
% configuration of Cassandra on each machine in the cluster
% are the following. 
The version of Cassandra used is 0. 8. 4 and the version of Hector used is
0. 8. 0-2. 

% Linux kernel version: Linux 3. 2. 4-1-ARCH i686
% 
% CPU: Intel(R) Core(TM)2 Duo CPU     E8400  @ 3. 00GHz
% 
% CPU cores: 4


%             total       used       free     shared    buffers     cached
% Mem:          3195       2941        254          0        212       1493

% \begin{table}[H] \label{texp:Nodeconfig}
% 	\centering
% 	\newcolumntype{C} {@{\hspace{2. 5pt}}>{\scriptsize}c@{\hspace{2. 5pt}}}
% 		\begin{tabular}{CC}
% 			\toprule
% % % 			\bfseries  Cassandra configurations \\
% %  			\midrule
% 			Cassandra version & 0. 8. 4\\
% 			\rc Hector Version & 0. 8. 0-2 \\
%  			Hinted Handoff & True  \\
% 			Partitioner & RandomPartitioner \\
%  			\bottomrule
% 			
% 		\end{tabular}
% 	\end{table}

% Cassandra configurations:
% Version: 0. 8. 4  Hector Version: 0. 8. 0-2  Replication strategy:
Some values in the configuration files on each node is changed before starting
the cluster of nodes.  For every node,  the \texttt{listen\_address} and
\texttt{rpc\_address} are set to its  hostname.  The nodes are added to the
cluster in a sequential order.  One of the nodes is chosen as the first node and
is made the seed node (i. e.  host).  This node becomes the contact point for other
nodes that join the cluster.  The list of seed nodes for a node is specified in
its configuration file in the \texttt{seeds} option.  For the first node,  this
option is set to its loopback address ``127. 0. 0. 1" to indicate that it does not
contact any other node to join the cluster.  For nodes that are not seed nodes, 
this option contains the hostnames that it can contact to learn about the
cluster.  In the experiments,  except for the first node all
the remaining 9 nodes have at least 2 hosts as seeds. 

The seed node has its \texttt{auto\_bootstrap} option set to \texttt{true} to
allow other nodes to migrate data from it when data is partitioned or when other
nodes join the cluster.  For nodes that are not seed nodes,  this option is set
to \texttt{false}.  This is because all the nodes are started prior to the 
experiments and do not have data to partition yet. 
 
% The Random Partitioner distributes rows in a cluster evenly and is the
% default configuration setting when Cassandra is installed. 
All the remaining
settings in the Cassandra configuration file are set to the  default values
for all the nodes.  The directories for saving the data,  commit logs and saved
caches are saved on the local disk of each node in its \texttt{tmp} folder and
not in the shared network drive.  This is in order to avoid using the network
for such storage needs. 


Note that for the Metadata cluster used in Solution~4,  the  cluster name is
\texttt{MetadataCluster}and the ports for Thrift clients  and TCP sessions are
changed.  

Once the cluster is started,  the experiments are initiated and the operations
are measured based on performance metrics.  The metrics used for the experiments
are described in the following section. 

%ब
\section{Performance Indicators} \label{sexp:PerformanceIndicators}
% Performance of database systems is commonly measured in terms of the
% \textit{Response time} and \textit{Throughput}.  

Response time and throughput are the indicators used to gauge the
performance of the four solutions under the implemented \ac{API} and the
respective referential integrity constraints specified by the example
application.  Response time refers to the time a \ac{DBMS}  takes to process an
operation and produce results to the end user~\citep{boral} (\todo{cite Demurjian, 
Berkely, serverside,  }).  Throughput  refers to  the number of operations that can be
processed by the \ac{DBMS} in a unit of time. 

In the experiments,  the response time is computed by dividing the total
execution time of an operation for a set of entities by the number of entities.  In other
words,  the response time measures the amount of time required to perform a
single operation on one entity.  On the other hand,  the throughput is the
inverse of the response time and is computed accordingly by dividing the number of entities by
the total execution time of an operation for a set of entities.  In other words, 
the throughput measures the number of operations that are performed in a unit
of time.  The following equations define the response time $r$ and throughput
$t$,
 

% \begin{tabular}{cc}
% $\displaystyle r = \frac{1}{n}\sum_{i = 1}^{n}{o_i} \label{eq:response-time}$ &
% $\displaystyle t = n / \sum_{i = 1}^{n}{o_i} \label{eq:throughput}$
% \end{tabular}

\begin{equation*}
\begin{gathered}
\displaystyle r = \frac{1}{n}\sum_{i = 1}^{n}{o_i}
\end{gathered} \hspace{2cm}
\begin{gathered}
t = n / \sum_{i = 1}^{n}{o_i}  
\end{gathered}
\end{equation*}

% \begin{align}
% \displaystyle
% 	 
% \end{align}

\noindent where $o_i$ is the time for an operation over entity $i$,  and  $n$ is
the number of entities. 


Notice that external variables such as network latency,   simultaneous processes
in the operating systems of each node,   and other variables are not considered
for the analysis of results.   Even when they are present,   it is expected that
results will not be significantly biased by them.   Nonetheless,   the
experiments will be  performed at night time over weekends as this is the time
 when the cluster is least used,   thus reducing the presence of such variables
and hence their impact on the results.  



% In the experiments,  the throughput of all the operations triggering referential
%  integrity validation across all solutions is measured as operations per second. 
%  A single operation stands for each time an entity is inserted or updated or
%  deleted.  Note that only the operations that introduce the referential integrity
%  validation are measured and thus \texttt{read} operations are not measured in
%  terms of response time or throughput.    


  


% Measuring response time for a
% database operation is similar to a black-box evaluation because it is measured 
% without considering the internal functioning  of the database system.   According
% to (\todo{cite Demurjian}) such an evaluation is ideal for a complete database
% system to measure its performance and to give the users details about its 
% efficiency and speed in performing operations.   
% The response time of Cassandra when such validations
% are not in place is also measured and considered as a baseline with which to
% analyse the solutions.   Such a comparison determines the degree of change in
% speed of Cassandra when such overheads are introduced and gives users useful
% information about how each solution affects the performance of the database
% system.  



% For example,   inserting 1000 students means that 1000 \texttt{insert}
% operations are processed by Cassandra.  

% The traditional TPC benchmarks are not considered as performance measures in
% this experiment  because these benchmarks are centred around transactions and
% OLTP workloads.   The principal metrics for these benchmarks are the transaction
% rate,   query per hour,   cost indicators of a system,   among others,   which are
% suitable indicators for \ac{DBMS} with ACID properties~\citep{TPC}.   Hence,   for
% assessing Cassandra which lacks SQL queries and  ACID properties,   these
% benchmarks are not suitable indicators of performance. 


%  it is
% essential to measure it in terms of what is critical to application  using Cassandra.   In this experiment it is critical to
% measure the difference in time for an operation to complete in Cassandra when
% referential integrity validation is activated or not activated.  


% These operations which trigger referential integrity validation for an entity
% namely the \texttt{insert},   \texttt{update},   \texttt{delete} operations are
% were measured in terms of the throughput in the experiments.   Throughout
% commonly referes to the number of operations performed

% It has to be noted that the operations are prone to  external factors like
% network latency,   bandwidth,   network routing,   network workload among others
% which typically affect a network consisting of several machines and users.  
% This is because the Cassandra cluster used in the experiments is deployed over
% a network that is used by many users concurrently thus exposing the operations
% to such factors.   Identifying such factors and analysing them is beyond the
% scope of this thesis and the analysis is strictly in terms of how the metadata
% storage and referential integrity validation affects Cassandra's performance.  
% It is a general practise for applications to incorporate code within
% applications to log the timestamps for transactions in traditional
% \acp{DBMS}~\citep{IBMPerformance}.  


% In order to determine the response time and throughput,   the output log files are
% are analysed using R.   (\todo{explain how it is imported to R and graphs
% produced--SOS Juan!})





%ब
\section{Experimental setup}\label{sexp:ExperimentalSetup}

% Since it is not a
% controlled environment, the experimentation involves performing several runs
% where data is inserted, updated, and deleted on the Cassandra cluster. 
% The
% environment is not controlled as variables like network latency, parallel
% processes in the nodes, and other variables are expected to affect the
% performance.


The experimentation consists of performing \ac{CRUD} operations upon artificial
data created for the University example application. Specifically, the
operations of interest are \texttt{Create}, \texttt{Update} and \texttt{Delete}
as these are the ones that trigger referential integrity validations. Notice
that, since the experiments are not performed in a controlled environment, all
the operations are repeated 100 times such that the effect of external factors
(e.g. network latency, parallel processes running in nodes, etc.)  is minimized.


The artificial data upon which operations are performed  is made up of 500
students, 500 courses, and 5000 enrolments. These numbers were chosen such that
the experiments could be performed in a reasonable amount of time. 
The format of the artificial data is:

	\begin{itemize}
		\item \texttt{Student} has a
		unit-increasing \texttt{StudentId}  which is merged into the fields \texttt{FirstName}
		 and \texttt{LastName} as ``First Name (StudentId)'' and ``Last Name
		(StudentId)''.  \texttt{Email} is composed in a similar way as
		``First.Last@email.(StudentId).com'' and \texttt{Age} is a random number
		between 18 and 60.
		
		\item  \texttt{Course} has a unit-increasing \texttt{CourseId} which is
		appended to the prefix ``COMP''.  It also has a composed \texttt{CourseName}
		as ``Engineering (CourseId)''.  \texttt{Trimester}, \texttt{Level}
		and \texttt{Year} are randomly generated numbers.
		
		\item  \texttt{Enrolment} contains a unit-increasing \texttt{RowId}  and the
		respective foreign keys of student and course,  which are \texttt{StudentId}
		and \texttt{CourseId}. 
	\end{itemize}
 
		
The order of the operations to be performed on the data in each run is as
follows.
\texttt{Create} inserts all the entities for \texttt{Student},  \texttt{Course}
and \texttt{Enrolment}.  \texttt{Update} performs changes on the primary keys of
\texttt{Student} and \texttt{Course} entities,  and on the foreign keys of
\texttt{Enrolment} (the one relative to courses,  specifically).  Finally, 
\texttt{Delete} removes all the \texttt{Student},  \texttt{Course} and
\texttt{Enrolment} entities.
Notice that the primary keys in every column family are different in each run
(create,  update,  delete), in order to avoid introducing biases to the results
as product of the tombstone delete paradigm that Cassandra utilizes.  That is, 
since Cassandra does not completely  remove the primary keys of the inserted
entities (tombstone delete),  reinsertion  using the same primary key might
yield faster times as the key already exists.  After each run,  all the column families
(\texttt{Student},  \texttt{Course},  and \texttt{Enrolment}) are emptied and
ready for the next run.   The details  of the \texttt{Create}, \texttt{Update}
and \texttt{Delete} operations are explained further in the following sections. 
		
% The operations to be performed on each run are as follows. Firstly,
% a batch of students, courses and enrolments are inserted in that precise order.
% Secondly, foreign keys of enrolments and primary keys of courses and students
% are updated. Lastly, enrolments, students and courses are deleted. Thus, having
% their respective column families emptied and ready for the next run. Notice
% that by empty, it refers to the fact that the column values of each row are
% emptied, but not the row key as the deletion conforms to the tombstone paradigm
% explained in previous chapters. The next run ensures to increase the
% primary key of the entities such that they do not exist on the column families.

	
\subsection{Create}
The \texttt{Create} operation inserts all the
\texttt{Student},  \texttt{Course} and \texttt{Enrolment} entities in that
precise order due to the nature of the referential integrity constraints
presented in Table~\ref{texp:ListConstraints}.   In the \texttt{Student} and
 \texttt{Course} column families,  \texttt{Create} does not trigger any
 referential integrity validation as these entities do not contain foreign keys. 
 Contrarily,  \texttt{Create} on \texttt{Enrolment} triggers foreign key
 validation checks on both \texttt{Student} and \texttt{Course} column families. 
		
\subsection{Update}
The \texttt{Update} operation is performed after the creation of all entities.
First,  an attempt is made to update the primary key of each \texttt{Course}
entity.  This operation triggers referential integrity  validations that result
in exceptions thrown as the \texttt{DeleteRule} \footnote{Notice that for the
sake of simplicity, this rule is also used for  update operations.} for all
\texttt{Course} entities is \texttt{NoDelete} and enrolments referencing the
courses are still present in \texttt{Enrolment}. Hence, the times recorded for
updating the \texttt{Course} column family represent the time required to
identify a constraint violation and throw the respective exceptions.
					
Next,  the \texttt{Enrolment} column family is updated.  In this case,  the
\texttt{CourseId} for each \texttt{Enrolment} entity is changed to a different
existing value,  ensuring that the distribution of \texttt{Student} and
\texttt{Course} entities remains the same. The update on the \texttt{Enrolment}
column family triggers referential integrity validation checks to ensure that
the course to which every \texttt{Enrolment} entity is being updated actually
exists in \texttt{Course} column family. 
					
Finally,  the primary key for each \texttt{Student} entity is updated to a new
integer value that has never existed in the column family. Thus, given
the \texttt{DeleteRule} for \texttt{Student} (i.e. \texttt{Cascade}),  this
operation triggers a cascaded update on the \texttt{Enrolment} column family   
 by respectively updating the student foreign key (\texttt{StudentId}) in all
 its existing \texttt{Enrolment} entities.
		
\subsection{Delete} 
The deletion of entities occurs first on the
\texttt{Enrolment} column family,  where all of its records are deleted without
requiring referential integrity checks as this is a child entity.  The times are
recorded for each \texttt{Delete} operation and then all of the entities are
reinserted with the same primary keys in order to assess the cascaded
\texttt{Delete} of \texttt{Student} entities next. 
				
Secondly,  all  the \texttt{Student} entities are deleted from the
\texttt{Student} column family. Hence, given the \texttt{Cascade}
\texttt{DeleteRule} of these entities,  the \texttt{ValidationHandler} ensures
to delete first all of the child entities before deleting a \texttt{Student} entity. 
Thus,  the times recorded for this operation also include the time required for
performing a cascaded \texttt{Delete} on the student dependencies in
\texttt{Enrolment}.
Notice that the dependencies exist at this point as they will have been reinserted into
\texttt{Enrolment} in the previous step. 
				
Finally,  all the \texttt{Course} entities are deleted.  Despite the courses
having a \texttt{NoDelete} rule,  notice that at this point the
\texttt{Enrolment} column family is empty,  so courses can be deleted as there
are no child dependencies.  Thus,  the times recorded for this operation measure
referential integrity validation as well as the \texttt{Delete} operation
of the \texttt{Course} entity.  After this final operation,  all column families
are emptied but all the primary keys still exist due to Cassandra's tombstone
delete.  However,  the whole keyspace is ready for the next batch of operations as
the primary keys of all column families will be different. 
	
	





\section{Summary} \label{sexp:Summary} 

This chapter  presented the experimental design to evaluate the performance of
each  solution and the experimental \ac{API} itself using the prototype keyspace
that is used as an example across this thesis.  The experimental design involves
assessing the performance of the CRUD operations on the different solutions
proposed for referential integrity. 
The analysis of results is to be based on response time and throughput,  two
performance indicators that serve as guidelines for assessing the trade-offs
between the different solutions proposed. 
	
	
The next chapter presents the results and their discussions of the experimental
design presented in this chapter
 






