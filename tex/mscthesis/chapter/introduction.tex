%ब
\chapter{Introduction}

% \citet{sys:msdn} say that this thesis is awesome.
% 
% Crap is the other thesis~\citep{sys:postgres}.
  
Cloud computing is a paradigm that is rapidly shifting the way IT services
and tools are being used. The fundamental reason is that it provides
 essential computing services that can be accessed through the Internet. 
 Amongst these services, cloud providers offer platforms that
 include hardware equipment and software applications, infrastructures such as
 servers and other network equipments, data storage and \acp{DBMS}, and
 others. All these services come with major benefits such as reduced maintenance
 costs,  automation of common tasks, and global access~\citep{wilkes}. Moreover, such cloud
 services and other resources are hosted on the Internet and offered to  
 users for availing according to their needs.  Such characteristics made it
  natural for the  cloud to become  the backbone of application providers and
  users, and promoted the creation of novel theories and more applications to
 enhance the cloud .
 
  
  Cloud data storage is an important service that has been progressing, evolving
  and adapting alongside cloud computing. As the number of
  applications and users keep increasing, so does the demand for scalable, 
  efficient, and consistent data storage mechanisms offered by
  application providers. These storage mechanisms allow  users to store
  their data on the cloud without having to deal with the hassle of buying and
  maintaining database servers~\citep{SNIA}.

 The distributed nature of the cloud data storage, requires
 cloud \acp{DBMS} to satisfy essential requirements that are inherent to these
 environments. Some of these requirements are related to
  \begin{inparaenum}[a)]
 \item  scalability by matching the increase and decrease of active users
 and allowing the storage of large amounts of data,
 \item  robustness by maintaining several copies of data to make data
 available despite failures,
 \item  flexibility by allowing the storage of unstructured data, and
 \item  consistency of data by making sure data is never stale.
 \end{inparaenum}
   These requirements are not necessarily met by traditional \acp{RDBMS}. For
   example, scalability in \acp{RDBMS} is limited as these are designed to run
   on a single node. Robustness is lacking in terms of  a single point of
   failure would  make data unavailable. Also,     Thus, the  prevalent
   features of traditional \acp{RDBMS}  become issues when moved to the cloud.
    Nonetheless, there is still one characteristic that is desirable from
   \acp{RDBMS}: referential integrity in data.
   
   Referential integrity constraints  ensure that  relationships
   between data items are preserved. For example, it ensures that a course
   exists before students can enrol in it. Such relationships are 
   inherent to real-world data and have to be maintained in databases
   whether it is a traditional \ac{RDBMS} or a cloud \ac{DBMS}.
   Moreover, these constraints  ensure that no operations violate  the
   integrity between  data items~\citep{Navathe}.   Despite its
   importance, referential integrity is currently not a prominent feature in most  cloud
   column-oriented key-value \acp{DBMS}. 
   
   
   Column-oriented key-value \acp{DBMS}, also  referred simply to as cloud
   \ac{NoSQL} databases,  adopts the column-oriented key-value data model which
   is a widely used model for cloud \acp{DBMS}.These \acp{DBMS}  support many
   features required in cloud,  such as elastic scalability to match varying
   user demands, data replication,  schema-less data storage ,   and many other
   features. In these \acp{DBMS},  referential integrity constraints are not
   provided because these were not conceived to maintain data relationships,
   partly due to the denormalized and decentralized nature of data storage
   ~\citep{Navathe}.
   Instead, the responsibility of maintaining referential integrity is delegated
   to the application layer.
    However, such an approach implies a significant overhead for applications as
    they have incorporate the referential integrity validation themselves. Not
    to mention that these \acp{DBMS} commonly handle large amounts of
    interconnected, dependant and widely replicated data which makes the validation more
    difficult. Hence, it becomes a  critical problem for applications to handle
   such  data where dependencies have to be correctly maintained and preserved.
   
   
	Inspired by such problems, this thesis studies the existing
modelling of data dependencies in cloud \ac{NoSQL}  \acp{DBMS}  and contributes by
providing four solutions to ensure  that referential integrity is
effectively maintained. These solutions  extend the consistency of data
relationships and  ensure  data integrity even when it is widely replicated or
even spread across data-centers. Also, they are provided as an \ac{API} between
the application layer and the database layer. This approach reduces the
workload of applications, as the responsibility
of validating referential integrity is delegated to these solutions.
%    
%    susceptible to poor data integrity as
% these do not normalise data nor maintain relationships. As such These \acp{DBMS} 
% lack integrity constraints that validate the correctness of data as seen in
% traditional \acp{RDBMS}  and one  important integrity constraint absent in most
% cloud \ac{NoSQL}  \acp{DBMS}  is the Referential Integrity Constraint.  These
% constraints  ensure that relationships between data items are preserved
% and maintained in databases and prevent any data operations that violate the
% integrity between data items. Without such constraints,  data could be inconsistent and unreliable. 
%      
%    
%  
   

% \acp{RDBMS}  get really
% complex when it comes to scaling to multiple nodes as these are best suited to
% run on a single node and  manipulating or accessing data spread on multiple
% nodes becomes complicated involving complex queries. 
% This is mainly due to
% normalisation,  as data is not duplicated.  This adversely affects robustness in
% cloud as well.  Since \acp{RDBMS}  enforce rigid schema or structure for
% databases,  flexibility of storing unstructured data on the cloud will be limited. 
% Data is available and consistent in \acp{RDBMS}  as long as the single node it is
% designed to run on is alive,  but this can also
% be a single point of failure causing complete unavailability of data. 

	 
	
	
 




% Databases on cloud adopt data models and database architecture that are still
% evolving and very different from the popular and traditional data models like
% relational data model. 
% 
% While cloud \ac{NoSQL}  \acp{DBMS}  support many  features suitable to the cloud, 
% it comes with a few disadvantages like poor data integrity,  less security, 
% limited functionalities and others. 


%  But it is common to find relationships or dependencies between data in the
%  business world and these have to be preserved when it is modelled in the cloud
%  environment too. 
 % upon storage in cloud \ac{NoSQL}  database systems too. 
% The replicated and distributed nature of the cloud \ac{NoSQL}  \acp{DBMS}  makes
% maintaining data dependencies complex and unfeasible in terms of speed and efficiency. 
% Thus,  referential integrity constraints to help preserve such dependencies are
% not offered by these \acp{DBMS}  and are left for the users to implement at the
% application layer instead.  This could mean immense
% workload for the application since cloud databases commonly have
%  large amounts of interconnected ,  dependant and
% widely replicated data,  which is usually spread across several data centers. 
% Handling such data where
% dependencies have to be correctly maintained and preserved becomes a critical
% problem for applications. 
% 
% Inspired by such problems of data dependencies,  this thesis studies the existing
% modelling of data dependencies in cloud \ac{NoSQL}  \acp{DBMS}  and contributes by
% suggesting four solutions so that referential integrity is effectively
% maintained , without sacrificing any existing benefits,  in these \acp{DBMS} . 
% These solutions extend the consistency of data relationships,   ensuring  data
% integrity even when it is widely replicated or spread on different data-centers. 
% Additionally,  it reduces the workload of the applications by delegating the
% responsibility of validating referential integrity to the \ac{DBMS} . 
% 



% 
% \chapter{Introduction} 
% Cloud computing is a major paradigm that is rapidly shifting the way IT services
% and tools are being used.  
% 
% %Services and applications are hosted on the internet
% % and users avail them according to their needs. 
% It offers economic benefits in infrastructure,  operating systems,  \acp{DBMS}  and
% other resources as these are hosted on the Internet and provided as cloud
% services which users avail  according to their needs. 
% % For these resources,  users pay only for what they use and do not have to
% % invest in buying expensive hardware or install operating systems,  while they
% % can avail these inexpensively from the cloud. 
% This leads to reduced maintenance costs,  automation of common tasks,  global
% access,  and many other advantages.  
% % For these advantages,  cloud soon became a
% % strong backbone for many applications and encouraged the migration of more
% % applications as well as  data storage to the cloud.  Also,  it  promoted many
% % novel theories and applications to enhance cloud (\todo{cite
% % (Wilkes, 2010)} ). 
% For these advantages,  it is natural for the cloud to be the
% backbone of  many applications 
% point of  encouraging the migration of other  applications  as well as data storage to the cloud.  Also,  it  promoted many novel
% theories and applications to enhance cloud (\todo{cite (Wilkes, 2010)} ). 
% 
% Cloud computing has been evolving and adapting to offer many different
% services.  Cloud data storage is an important part of it which is progressing
% alongside. 
% The demand for scalable,  efficient and consistent data storage mechanisms has
% been increasing as more applications  and users store their data  on the cloud, 
% rather than buying and maintaining database servers \todo{cite SNIA}  . 
% 
% 
% 
% Cloud \acp{DBMS}   need to satisfy requirements that are exclusive and essential
% in a distributed environment.  Some of the expected features in these \acp{DBMS} 
% are the scalability to match the increase and decrease of active users, 
% robustness by maintaining several copies of data,  flexibility to store
% unstructured data,  availability of data despite failures, ability to store large
% amounts of data,  consistency even during  concurrent operations and many other
% features.  Most of these features are inherent in distributed environments and
% are not necessarily addressed by traditional \acp{RDBMS} .  \acp{RDBMS}  get really
% complex when it comes to scaling to multiple nodes as these are best suited to
% run on a single node and  manipulating or accessing data spread on multiple
% nodes becomes complicated involving complex queries. This is mainly due to
% normalisation,  as data is not duplicated.  This adversely affects robustness in
% cloud as well.  Since \acp{RDBMS}  enforce rigid schema or structure for
% databases,  flexibility of storing unstructured data on the cloud will be limited. 
% Data is available and consistent in \acp{RDBMS}  as long as the single node it is
% designed to run on is alive,  but this can also
% be a single point of failure causing complete unavailability of data. 
% % users,  high data availability  despite failures,  replication of data on
% % several machines,  support of high volumes of data and data consistency even
% % during concurrent operations and many other features. 
% % These features are different to the ones offered by popular and commonly used
% % traditional \acp{RDBMS}  where ACID properties,  normalised and non redundant
% % data,  rigid schemas,  efficient query support make these \acp{DBMS}  efficient, 
% % but in a non distributed environment. 
% % While these traditional \acp{DBMS}  come with a rich set of features,  these
% % features hinder its adaptability on the cloud.  For example,  normalisation of
% % data prevents data replication as it removes redundancy in data and rigid
% % schemas reduce the flexibility of storing unstructured data on cloud. 
% % Scalability  is affected poorly due to ACID properties which attempt to
% % prevent concurrent operations to make data consistent and isolated. 
% Such problems led to the rise of many Cloud \acp{DBMS}  with different data
% models and architectures supporting a wide array of cloud adaptable features. 
% However,  \acp{RDBMS}  have the desirable feature of referential integrity
% constraints,  which is not currently applied in cloud column-oriented key-value
% \acp{DBMS} . 
% 
% Column-oriented key-value \acp{DBMS} on the cloud are called cloud
% \ac{NoSQL}  \acp{DBMS}  or just cloud \ac{NoSQL}  databases and are fundamentally
% different from  traditional \acp{RDBMS} . These \acp{DBMS}  support many features
% required in cloud,  such as elastic scalability to match varying user
% demands, data replication,  schema-less data storage ,   and many other features. 
% These \acp{DBMS}  are still evolving and introducing new features in their
% architecture and becoming specialised to solve specific cloud data storage
% issues.  However,  these \acp{DBMS}  are susceptible to poor data integrity as
% these do not normalise data nor maintain relationships.  These \acp{DBMS}  lack
% integrity constraints that validate the correctness of data as seen in
% traditional \acp{RDBMS}  and one  important integrity constraint absent in most
% cloud \ac{NoSQL}  \acp{DBMS}  is the Referential Integrity Constraint.  These
% constraints  ensure that relationships between data items are preserved
% and maintained in databases and prevent any data operations that violate the
% integrity between data items. Without such constraints,  data could be inconsistent and unreliable. 
% 
% % Databases on cloud adopt data models and database architecture that are still
% % evolving and very different from the popular and traditional data models like
% % relational data model. 
% % 
% % While cloud \ac{NoSQL}  \acp{DBMS}  support many  features suitable to the cloud, 
% % it comes with a few disadvantages like poor data integrity,  less security, 
% % limited functionalities and others. 
% 
% 
%  But it is common to find relationships or dependencies between data in the
%  business world and these have to be preserved when it is modelled in the cloud
%  environment too. 
%  % upon storage in cloud \ac{NoSQL}  database systems too. 
% The replicated and distributed nature of the cloud \ac{NoSQL}  \acp{DBMS}  makes
% maintaining data dependencies complex and unfeasible in terms of speed and efficiency. 
% Thus,  referential integrity constraints to help preserve such dependencies are
% not offered by these \acp{DBMS}  and are left for the users to implement at the
% application layer instead.  This could mean immense
% workload for the application since cloud databases commonly have
%  large amounts of interconnected ,  dependant and
% widely replicated data,  which is usually spread across several data centers. 
% Handling such data where
% dependencies have to be correctly maintained and preserved becomes a critical
% problem for applications. 
% 
% Inspired by such problems of data dependencies,  this thesis studies the existing
% modelling of data dependencies in cloud \ac{NoSQL}  \acp{DBMS}  and contributes by
% suggesting four solutions so that referential integrity is effectively
% maintained , without sacrificing any existing benefits,  in these \acp{DBMS} . 
% These solutions extend the consistency of data relationships,   ensuring  data
% integrity even when it is widely replicated or spread on different data-centers. 
% Additionally,  it reduces the workload of the applications by delegating the
% responsibility of validating referential integrity to the \ac{DBMS} . 
% 


\section{Objectives} 

This thesis  focuses particularly on the column-oriented
key-value \acp{DBMS}  on the cloud.  Four  solutions 
using different metadata management techniques are suggested to incorporate
referential integrity constraints in such \acp{DBMS}.  These
solutions are implemented and tested on Apache Cassandra, a popular and
prominent column-oriented key-value \ac{DBMS}  in the cloud. 

The performance of these solutions is assessed  in terms of response time and
throughput.  All the proposed solutions extend data integrity
and  preserves the data dependencies regardless of the scalability and workload
of the \ac{DBMS}. 

\subsection{General Objective}
 Incorporate validation mechanisms for maintaining referential integrity in
 column-oriented key-value \acp{DBMS}.

\subsection{Specific Objectives}
\begin{itemize} 
  \item Design four solutions to implement referential integrity constraints,
  using metadata in various ways to store the information about data
  dependencies. 
  \item Develop an \ac{API}  to implement the four solutions in one of the
  existing column-oriented key-value \acp{DBMS}, namely Cassandra.  
  \item Analyse the performance of the solutions to determine their
   feasibility and practicality. 
\end{itemize} 

% \subsection{Contributions}
% The main contributions of this thesis are an \ac{API}  that can be
% used as an interface to implement referential integrity constraints in cloud
% column-oriented key-value \acp{DBMS},  and a performance evaluation about the
% implementation of the four approaches used in the \ac{API}   in
% Cassandra. 

\section{Organization} 

The remainder of this thesis is structured as follows.  Chapter \textmd{2} 
presents cloud computing and describes the column-oriented key-value data model, 
its challenges and the general architecture of Cassandra on which the
proposed solutions are implemented and tested. 
Chapter 3 describes the design of the proposed solutions along with the
motivation for the design and the implementation of these solutions in
Cassandra.  Chapter 4 details the experiments performed to measure the
performance of the solutions.  Chapter 5 presents the analysis of the results
from the experiments.  Chapter 6  presents the conclusions and ideas for future
work.
