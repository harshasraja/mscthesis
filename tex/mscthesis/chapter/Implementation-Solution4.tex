%ब
\section{Solution 4:  Metadata Cluster}\label{s:Implementation-Solution4}
In Solution~4, metadata is saved in a separate column family called
\texttt{Metadata} similar to Solution~3, but it is stored and maintained in a
separate cluster. Figure~\ref{fd:MetadataCluster-Solution4} illustrates the
separate clusters used in this solution. To access the metadata of an entity for 
performing validations, the
metadata cluster has to be accessed primarily. The approach to retrieve, process
and access metadata are discussed in the sections below.

\subsection{Metadata Retrieval}
The constraints are treated as entities and inserted into the \texttt{Metadata}
entity class as in Solution~3. Like Solution~3, the constraints
are provided by the users and inserted using the \texttt{insert} operation which
is designed to prevent referential integrity validations for metadata entities.
 To insert constraints into \texttt{Metadata}, a separate
connection is established to the \texttt{MetadataCluster} and to insert the other entities
into their respective entity classes in the keyspace another connection is
maintained to the \texttt{KeyspaceCluster}.

The constraints are retrieved from  \texttt{Metadata}, which 
maintained in the \texttt{MetadataCluster}, using the connection object
established for connecting to this cluster.  \texttt{Metadata}
has the necessary getter and setter methods to access its attributes which are
the different parts of a constraint. Additional operations like string parsing
are not required in this solution although a separate connection is required.

\subsection{Metadata Access}

 
Whenever an operation is invoked on an entity, the \texttt{ValidationHandler}
accesses \texttt{Metadata} and the relevant constraints of the entity are
extracted from \texttt{Metadata} by identifying constraints that have  the entity as the value
for its \texttt{ColumnFamily} field, which is similar to Solution~3.The relevant constraints
are stored as a list and maintained as cache till the operation is completed.
Hence, all the relevant \ac{PK} and \ac{FK} constraints for an entity class are
stored in the cache and reused for future operations on the entities of that
type.% 
% stored as cache by . For all the operations on entities of the entity class, the
% cached metadata is used. A connection is established to the metadata cluster in
% order to access the \texttt{Metadata} column family.  
The \texttt{ValidationHandler} accesses the values from the relevant constraints
using the getter methods of the \texttt{Metadata} and validates referential
integrity for entities.

% In this solution, the relevant constraints of an
% entity are accessed from the \texttt{Metadata} entity class using its getter
%  methods, similar to the previous solutions.