%ब
\section{Solution 3:  Metadata Column Family}\label{s:Implementation-Solution3}

In Solution~3, metadata is saved in a separate column family called
\texttt{Metadata}within the cluster, which contains all the constraints within a
keyspace. To validate referential integrity, the appropriate constraints of an
entity have to be retrieved and accessed from \texttt{Metadata} by the handlers
in the experimental \ac{API}. The way metadata is retrieved, deciphered and accessed
are discussed in the following sections.
 
\subsection{Metadata Retrieval}
In this solution, constraints are inserted into \texttt{Metadata} column family
in the same way as entities are inserted into their respective entity
classes. 
% For example, students are inserted into \texttt{Student} entity
% class. 
Here, constraints are treated as entities of \texttt{Metadata} entity
class. Note that users of the experimental \ac{API} have to provide the
constraints of the entire keyspace to insert them into \texttt{Metadata}. Moreover, 
 constraints are inserted into \texttt{Metadata} using the \texttt{insert}
operation and this operation is designed to prevent referential
integrity validations for metadata entities.

In this solution, \texttt{Metadata} is considered as an entity class and has the
necessary setter and getter access methods for its attributes. The attributes
for this entity class are the different parts of a constraint. The
\texttt{Metadata} entity class is the same as shown in
Figure~\ref{fi:MetadataEntityClass}. Unlike Solutions~1 and 2, Solution~3
performs no additional operations like String parsing  to
retrieve constraint information from the metadata.

\subsection{Metadata Access}

% To ensure referential integrity, the \texttt{ValidationHandler} accesses the
% relevant constraints of an entity.


For the \texttt{ValidationHandler} to perform validations, it is essential that
it accesses only the relevant constraints of an entity from all the constraints
stored in the \texttt{Metadata} entity class. To distinguish the relevant
constraints of an entity, the \texttt{ValidationHandler} iterates through the
constraints of \texttt{Metadata} and identifies the constraint that have the
entity as the value for its \texttt{ColumnFamily}.
Consider Figure~\ref{fd:Metadata-Solution3} which shows a few constraints in
\texttt{Metadata}. To locate the constraints of \texttt{Student} entity class,
the constraints with \texttt{ColumnFamily} containing the value
\texttt{Student} have to be identified, which in this example is constraint
\texttt{CONST100}.
Thus, the relevant \ac{PK} and \ac{FK} constraints of an entity are identified
by the \texttt{ValidationHandler}.

 Once the relevant constraints are
accessed,  the\texttt{ValidationHandler} performs
the necessary validations by accessing the values from the relevant
constraints using \texttt{Metadata} getter methods, similar to Solutions~1 and
2.


