%ब
\section{Cloud Databases}\label{s:cloud-databases}

Most cloud applications store,   process and provide large amounts of data like
the user information,   application data or some stored data which maybe
accessed by the users.  Storage of such data during all times is essential for
the cloud applications to operate correctly~\citep{Kennedy}.
Traditionally, users store data in files or databases residing on dedicated
database servers or on local disks, but the requirements for data storage on the
cloud are very different and need a distributed approach in data storage, where
data is spread across several machines.  Cloud \acp{DBMS} require more features
than the traditional \acp{DBMS} for an efficient data management on the
cloud~\citep{Abadi,Florescu,Hackl,Shietal, Stonebraker2,Stonebraker}.
Unlike traditional \acp{DBMS}, cloud \acp{DBMS} are simple in their structure
with minimum querying support and have a simple API for database administration.
 These have been made scalable to support the diverse and large number of users
who store structured data and to support various applications that users use.
Today,   cloud \acp{DBMS} are replicated,  distributed, simplified and often
specialised~\citep{Cooper}. Databases on the cloud (also known as cloud
databases) are replicated so that multiple copies of data are available to cater
to many users who access the same data at the same time.  This also helps in
cases of server crashes or network failures,   as copies of the data are
available.  These databases are distributed as data is replicated on several
machines.  Most cloud \acp{DBMS} are simplified for ease of use and specialised
to address certain cloud related problems.  For example,   some cloud \acp{DBMS}
are built to provide high scalability while others are built to store huge
amounts of interconnected data.
 
Most cloud \acp{DBMS} are deployed in data centers owned by hosting companies
like Google,   Amazon etc.  Data centres house many servers,   computers and
telecommunication infrastructure,   including back up and security facilities
and users can rent or buy the storage space they need.  Within data centres,  
data is stored on remote machines,   which could be any server within the data
centre or in a different data centre.  When users connect to cloud databases
through the Internet,   they remain unaware of the exact location of their stored
data.   Users are guided to their databases by \acp{API} of the cloud
\ac{DBMS}~\citep{Wuetal}. 

Cloud databases have to be scalable across these servers so that data is
available to any user at any given point in time.  Scalability in the context of
cloud storage refers to the ability of dynamically incorporating changes to the
number of users or storage space,   without affecting the functioning of the
databases or the availability of data to the users.  In other words,   when more
machines are added to increase storage capacity,   or when more users access the
same data,   cloud databases should cope with the increased workload and yet
maintain the same throughput.

In general,   cloud \acp{DBMS} are found to be less efficient than traditional
\acp{DBMS} because of this dynamic scalability required to support a changing
user-base~\citep{Abadi,Hogan,Stonebraker}. \citet{Hogan} claims that data
partitioning in cloud databases increases complexity as a database is spread
across several servers and querying the database would involve complex Joins and
more time.  This moves the databases and the user applications farther apart,
increasing latency. Hence,   data is split into distinct
individual parts and saved on different nodes in the data centre across several
databases. Thus,   nodes could have a subset of data or rows from each table in
the database~\citep{dewitt}. This eventually means that querying would take
longer time as the data is spread across several databases, possibly on
different servers and would include multiple joins on the datasets.

Most traditional \acp{DBMS} are relational and give data a structure,   that
adheres to a schema.  This is mainly achieved through the process of
normalisation where each table in a database is evaluated according to its
functional dependencies and primary keys,   to reduce redundancy and minimise
integrity anomalies~\citep{Navathe}.  Normalisation causes databases to have
smaller and structured tables by removing duplicate data from large and badly
organised tables and by imposing constraints on the data.  Tables are normalised
to at-least \ac{1-NF} ensuring data is organised and less redundant.
Redundancy is further reduced by bringing the database schema into at-least
\ac{3-NF} (or \ac{BCNF}). Throughout the chapters normalization refers to making
databases at least in \ac{1-NF}.

Cloud \acp{DBMS} are mostly non relational and follow a different data model,
which is explained in Section~\ref{s:cloud-data-models}.  Such Cloud \acp{DBMS}
are loosely termed as cloud \ac{NoSQL} \acp{DBMS} and do not provide support for
efficient querying or query languages such as SQL. Unlike \acp{RDBMS}, cloud
\ac{NoSQL} \acp{DBMS} do not aim to be
ACID-compliant~\citep{Florescu,Han,Stonebraker2,Stonebraker}.
ACID stands for the properties Atomicity,   Consistency,   Isolation and
Durability, which ensure the completeness and reliability of a database
operation.  In general,   unless operations are not ACID compliant in
\acp{RDBMS},   it would not be considered valid. But ACID compatibility in cloud
\ac{NoSQL} \acp{DBMS} is a bottleneck as it does not suit the distributed nature
of the cloud environment~\citep{Stonebraker2,Wada}.
Cloud \ac{NoSQL} \acp{DBMS} require to have high throughput,   high availability
and also require to be elastically scalable to increasing resources or users.
This requires cloud \ac{NoSQL} \acp{DBMS} to part with some traditional
\ac{RDBMS} functionalities (like JOINS) and ACID operations mainly because of
its distributed nature~\citep{Wada}.  The distributed nature, across different
environments, sometimes make cloud \acp{NoSQL} \acp{DBMS} to be prone to node
failures.  Node failures and the elasticity prevailing in cloud environments
affect consistency of data,   which adversely affect the '\texttt{C}' of ACID
properties,   i. e. ,
  Consistency.  Network and data partitioning also play a major role in
affecting consistency and availability of data.   A partition takes place when a
node fails or there is a network failure at some point in the network.  In cloud
\ac{NoSQL} \acp{DBMS} this is a problem as cloud databases rely on more than one
server,   unlike traditional \acp{DBMS} sometime ago.  When there is only a
single server involved there is no issue of partitioning,   but when there is a
failure,   i. e.  this single node fails, no data is available during that time.
 But cloud \ac{NoSQL} \acp{DBMS} distribute data across several servers and
nodes,   mainly to replicate data for high availability and fault tolerance.
These \acp{DBMS} also aim for partition tolerance,   which is the ability to
continue their operations despite node failures and partitions.  To achieve
these features it is commonly found that cloud \ac{NoSQL} \acp{DBMS} sacrifice
data consistency.  Commonly,   most web applications aim to have their data
available at anytime,   as many users could access the data at the same time and
in a business model,   applications lose valuable customers if they are not kept
satisfied with the services in terms of speed,   availability and consistency.
Instead of ACID properties,   cloud \ac{NoSQL} \acp{DBMS} aim to achieve
different properties,   namely Consistency,   Availability and
Partition-tolerance (CAP) properties stated in the CAP
theorem~\citep{Gilbert,Ramakrishnan,Wada}.
These properties and the CAP theorem are explained below.

\subsection{CAP Theorem} \label{ss:cap}
In distributed environments or web-based applications it is noted that the three
main system requirements necessary for designing and deploying applications are:
Consistency,   Availability and Partition tolerance.  CAP theorem,
proposed by~\citet{Brewer}, claims that it is not possible for a distributed system to
achieve all these three properties at the same given time.

\begin{itemize}
  
	\item Consistency: When a request is made to access data,   a system is called
	consistent if it provides the correct and latest version of the
	data~\citep{Tai,browne,hp,Ramakrishnan,Wada}.  For example, in the case of an
	online shopping website, consistency would ensure that the stock of items is
	always correct.  When a user attempts to buy an item and the same item is being
	bought by another user,   the system will have to ensure that both the users get
	the most recent stock details available.  So if there is only a single item
	left,   then the second user is informed that his request can not be completed
	as there is no stock available. This means that the data is consistent and users
	do not get stale data.
		
	\item Availability: A system is considered highly available when all parts of
	the system are always available, despite any failures or problems.  It is
	expected that all requests would be addressed any given point of
	time~\citep{Tai,browne,Ramakrishnan,Wada}.  In the previous example,   this
	means that even when the website is busy,   with many users accessing it,   it
	is expected to have all user requests addressed and to be up and running always.
			
	\item Partition-tolerance: Generally, distributed services are run on several
	machines across different networks and these services are prone to network
	partitions~\citep{Brewer,Gilbert,Ramakrishnan}. Network partitions happen when
	there is a failure of a segment or component of a network such that nodes cannot
	communicate with each other.
		% As previously mentioned, this means that data is
	% partitioned across the distributed network on several machines and a partition
	% takes place when there is a failure in the network.
	A system is considered  partition-tolerant when despite such partitions,   it
	continues to provide its services and address user requests.
	
\end{itemize}

The CAP theorem states that at a given point of time,   only two of these
properties can be achieved or satisfied by any application.  This means that an
application that is distributed like cloud \ac{NoSQL} \acp{DBMS} has to make
trade-offs on one of the properties always.  Trade-offs are common in the real
world,   where some features are sacrificed for other features,   that may suit
the business or operation model better.  In most distributed applications
trade-offs are always considered from the design stages.   Similarly most cloud
\ac{NoSQL} \acp{DBMS} have chosen priorities and trade-offs too.  For example,  
Cassandra focuses on '\texttt{A}' and '\texttt{P}' while Bigtable focuses on
'\texttt{C}' and '\texttt{A}'~\citep{bigtable}.

What such trade-offs mean in relation to the CAP theorem is examined
below~\citep{abadi2,Tai,Brewer2,hp,Ramakrishnan}:

\begin{description}
	\item [Case 1: Achieving '\texttt{C}' and '\texttt{A}' properties:] This means
	that when an application aims to achieve consistency and availability, it will
	be less partition tolerant.
	When data is partitioned,   there is more time involved in accessing the data
	from the various points in the distributed network.  Moreover,   failures mean
	more time delays.  So to achieve high consistency and availability it is
	required that the application depends on fewer nodes.  Having all data stored on
	a single machine means there is no partition of data and this data will be
	always consistent and available as long as this machine is up and running.   On
	the other hand,   when systems that are not partition tolerant face a partition,
	it becomes unreliable.  It could either give inconsistent data or become
	unavailable or both.
		
	\item [Case 2: Achieving '\texttt{A}' and '\texttt{P}' properties:]
	Commonly,   this is what most cloud \ac{NoSQL} \acp{DBMS} aim to achieve and these
	prefer to pay less attention to consistency of data~\citep{Wada}.
	A system lacking consistency is thus mostly available even during partitions,   but
	may give stale data or incorrect data occasionally to the users.  This suits
	most business models as this ensures that users are always able to
	access their data.  For example,   in the  online shopping example,   when
	data is not available or the request fails,   users can get anxious whether
	they lost their money during the transaction.  In order to avoid such cases,  
	users are presented with data as soon as possible,   despite being stale,   since users
	will be able to see the data rather than being left unsure. 
	Since a cloud \ac{NoSQL} \ac{DBMS} that has poor consistency is unrealistic,  
	these \acp{DBMS} tend to provide eventual consistency
	
	\item [Case 3: Achieving '\texttt{C}' and '\texttt{P}' properties:] This means
	that while a system is consistent and tolerant to partitions or
	failures,   it may not always be available and running.  Such a system 
	provides correct data while tolerating network failures but may not be
	accessible during failures preventing operations to be performed on data.  This
	leads to a less reliable system,   where data is correct but unavailable and
	inaccessible during network failures.
	
	 
\end{description}

Interestingly,   while cloud \ac{NoSQL} \acp{DBMS} do not comply with ACID
properties,   the CAP system has lead to a new set of properties called BASE and
is considered as an alternative of ACID properties in distributed and scalable
systems~\citep{Pritchett}.
BASE refers to the properties Basically Available,   Soft-state and Eventually
consistent.  This means that data is basically available although at some point
not all data will be available.  Soft-state indicates that data could be lost if
not properly maintained,   i. e. ,   data has to be refreshed and
version-checked for it to remain saved.  Eventually consistent,   as mentioned
previously,   is a weak form of consistency where in a cluster of nodes,   every
node would get the updates eventually at some point in time.
BASE could be understood as being closer to  \texttt{Case 2} mentioned above,
where consistency takes a back seat.  But this leads to conflicts where a new
update or a new read request could be made before all nodes get the latest
update.  To resolve such conflicts there are some types of repairs used by cloud
\ac{NoSQL} \acp{DBMS}, like read-repairs,   write-repairs and asynchronous
repairs~\citep{Terry}. When a read or write operation takes place,   such
repairs check for inconsistency in data before correctly updating the data. 
Some cloud \ac{NoSQL} \acp{DBMS} also rely on APIs to work around such issues. 
It is often considered that with good design,   \acp{API} can work around these
problems and provide better consistency.
% Unlike such traditional \acp{DBMS},   cloud \acp{DBMS} are simple in their
% structure with minimum querying support and have a simple API for users for
% database administration.  Cloud databases have been made scalable to support
% the diverse and large number of users who store structured data and to support
% various applications that users use.  Today,   cloud databases are replicated,
% distributed,   simplified and often specialised (Cooper,   2010).  Cloud
% databases are replicated so that multiple copies of data are available to
% cater to many users who access the same data at the same time.  This also
% helps in cases of server crashes or network failures,   as copies of the data
% are available.  The cloud database are distributed as data is replicated on
% several machines.  Most cloud databases are simplified for ease of use and
% specialised to address certain cloud related problems.  For example,   some
% cloud databases are built to provide high scalability while others are built
% to store huge amounts of interconnected data.

All these characteristics make cloud \ac{NoSQL} \acp{DBMS} very different from
traditional \acp{DBMS} that are used outside cloud networks.  As mentioned
previously, the underlying data model of the cloud \ac{NoSQL} \acp{DBMS} is
fundamentally different from the relational data model of \acp{RDBMS} and this
is explored in the following sections.
