%ब
\section{Referential Integrity in Key-Value
Model}\label{s:referential-integrity} 

Referential integrity is a fundamental property of data within databases,  
which ensures that data dependencies between tables are maintained correctly in
the database~\citep{blaha,date,Navathe,george}.  These dependencies
could be a part of the business rule and need to be enforced for proper data integrity.  Users define
conditions or rules on the tables in a database so that data integrity is
ensured at all times.  These conditions are called integrity constraints and
need to be mandatorily satisfied at all times in order to ensure that users or
applications do not enter incorrect or inconsistent data into the databases.

The Referential Integrity Constraint is just one amongst other constraints,  
and generally in \acp{RDBMS},   these constraints ensure that the value of
foreign keys in a table matches the values of primary keys in another table. 
The table containing the foreign key is the referencing table (or child table),
while the table with the primary or unique key is the referenced table (or
parent table).
For example,   in the University database,   \texttt{Enrolment} is the
referencing table while \texttt{Student} and \texttt{Course} are the referenced
tables.  Foreign keys are also known as the referencing key and
the primary keys as the referenced keys. 

It has been defined by many researchers that referential integrity is enforced
by the combination of a primary (or unique) key and a foreign key,   and that
every foreign key has to match the primary
key~\citep{blaha,Navathe,george,pathivada}.
In the University example,   every foreign key in the \texttt{Enrolment} table must
match one of the primary keys in the \texttt{Student} and \texttt{Course}
tables.
Hence,   if any foreign key refers to a non-existing primary key,   the
referential integrity constraint is violated.   For example,   if
'\texttt{StudID100}' is a foreign key for a student in the \texttt{Enrolment}
table,   but '\texttt{StudID100}' does not exist as a primary key in the
\texttt{Student} table,   then it is a violation of referential integrity.
 
Referential integrity constraints also describe the data manipulation that is
allowed on the referenced values.  Some of the widely associated rules are:

	\begin{itemize}
	
		\item \texttt{Restrict} or \texttt{No delete}: which prevents any update or
		deletion of data that has references. 
		
		\item \texttt{Set to NULL}: which sets all foreign keys to NULL values,   on
		updating or deleting the referenced key. 
		
		\item \texttt{Set to Default}: which sets all the foreign
		keys to a default value,   on updating or deleting the referenced key. 
		
		\item \texttt{Cascade}: which updates or deletes all the
		associated dependant values accordingly,   when the referenced data is updated or
		deleted. 
		
		\item \texttt{No Action}: which performs checks only at the end of a
		statement and is similar to \texttt{Restrict}
		
	\end{itemize}

Existing \acp{DBMS} may not always support all of the above rules.  Some \acp{DBMS} may
have the \texttt{Cascade} rule by default like Oracle,   while some may have the
\texttt{Restrict} rule by default.  

Generally,   in \acp{RDBMS} the database manager enforces a set of rules to
prevent any data operation,   like insert,   update or delete,   to change data
in such a way that referential integrity is not violated as seen in
Figure~\ref{f:RI}. 

	\begin{figure}[H]
		\centering
		%\includegraphics[width=5cm,   height=5cm]{. /figure/random. jpg}
		\includegraphics[width=.8\textwidth]{./figure/Example/RI-Figure.png}
		\caption{Referential Integrity Rules}\label{f:RI}
	\end{figure}

\todo{Fix this paragraph}
Referential integrity constraints have been a relational feature in traditional
\acp{RDBMS} and are imposed due to the way the \acp{RDBMS} enforce
normalisation.
In order to improve the data dependency in cloud \ac{NoSQL} \acp{DBMS},   this
thesis proposes solutions that implement referential integrity constraints using
different approaches. These solutions are
deployed and evaluated using Apache Cassandra, a column-oriented key-value cloud
Database Management System (DBMS). The following section gives an overview of
Cassandra and discusses some of its key architectural concepts.

% These rules are explained below:

% 	\begin{itemize}
\subsection{Insert rule}
		An insert operation triggers a referential integrity
		validation when data is being inserted into a referencing table,   i. e. ,   the
		child table.  In such an event,   prior to entering the values in the
		referencing table,   it is checked if the foreign keys exist in the referenced
		table.  For example,   in the University \ac{RDB},   when a row is inserted in
		the \texttt{Enrolment} table with foreign key values for \texttt{StudentID} and
		\texttt{CourseID},   a check is triggered to verify whether these foreign keys
		exist in the \texttt{Course} and \texttt{Student} tables as primary keys.  If
		the foreign keys do not exist in the referenced tables,   then the insert
		operation is not allowed.
		
\subsection{Update rule}
 When data is updated either in the referencing table or the
		referenced table,   a referential integrity validation is needed.  When any
		primary key is updated in the referenced table, then it is verified whether this
		key is a foreign key in any of the referencing tables.  If a dependency is found
		to exist,   then the applicable data manipulation rule is checked.  For
		instance,   if it is a \texttt{Cascade} rule,   then the associated foreign keys
		in the referencing table are updated prior to updating the key in the referenced
		table.  In the University \ac{RDB}, if the primary key '\texttt{SWEN100}' for a
		course is updated to '\texttt{SWEN101}',   then all the records in
		\texttt{Enrolment} that have '\texttt{SWEN100}' as a foreign key have to be
		updated to '\texttt{SWEN101}', if it has a \texttt{Cascade} rule.
		
		When any foreign key is being updated in a referencing table,   then a
		referential integrity validation has to be performed.  It is ensured that the
		new updated value exists as a primary key in the referenced table.  For example,
		  in the \texttt{Enrolment} table,   if \texttt{CourseID} in a row is updated to
		a new value,   then it is verified that the new value is an existing primary key
		in the \texttt{Course} table.  If the new value does not exist,   the update is
		not allowed generally.
		
\subsection{Delete rule} A delete operation triggers a referential integrity
		validation when data is deleted from the referenced table.  When data that is
		marked for deletion is found to have dependencies in other referencing tables,  
		the data manipulation rule applicable for this operation has to be checked. 
		This means that if the rule allows \texttt{Cascade},   then the depending values
		in the referencing table have to be removed prior to deleting values from the
		referenced tables.  For example,   when a student record is deleted from the
		\texttt{Student} table,   a check is performed to see if the
		'\texttt{StudentID}' is a foreign key in any other table.  Therefore,  
		\texttt{Enrolment} is checked and when the '\texttt{StudentID}' is found
		as a foreign key,   the appropriate action is performed depending on the data
		manipulation rule.  If it is '\texttt{Cascade}',   the enrolment details for the
		'\texttt{StudentID}' are removed from \texttt{Enrolment} and then the student
		record is deleted from
		\texttt{Student}. 
	
% 	\end{itemize}


 

