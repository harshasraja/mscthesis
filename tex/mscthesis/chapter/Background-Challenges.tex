% ब
\section{Challenges in the Key-Value Data Model}\label{s:challenges-key-value}
Fundamentally,   the key-value data model is different from the relational model
in many ways.  While the relational data model aims at giving data a structure
and providing data integrity,   the key-value data model just store data as
\acp{blob} or string values and generally do not maintain relationships between
data.  In the column-oriented key-value model,   the key-value association and
the grouping of columns in column families can be considered as the minimum
relationship that is maintained.

According to~\citet{Bell},   data dependencies are the most common types of
semantic constraints in relational databases and these determine the database
design.  Data dependencies are the various relationships that may exist between
data entities in a database.  For example,   in the University database,   a
student can enrol into more than one course and this means that there is a
many-to-many relationship between \texttt{Student} and \texttt{Course} since  
one course can have many students enrolled in it.

As seen in Section~\ref{s:key-value-data-model},   the \texttt{Enrolment} table
contains the \texttt{StudentID} and the \texttt{CourseID} as foreign keys, thus
showing the dependency or relationship between students and courses
(Figure~\ref{f:RDB}).
In the \texttt{University} \ac{RDB}, any attempt to delete a course from the
\texttt{Course} table is prevented by a constraint,   unless the dependency
itself is removed first.  In \acp{RDBMS},   such constraints are the referential
integrity constraints ,   which ensure that references
between data entities are valid, consistent and intact~\citep{blaha,date}.
Normalisation,   as well as modelling real world data and relationships enforce
such dependencies in the schema and this causes integrity constraints like
referential integrity constraints,   to be imposed on data entities.

When such constraints are not imposed, the database  is prone to dangling
dependencies.  Consider the case of foreign key references between
\texttt{Course} and \texttt{Enrolment} in the  University database.  If a course
is deleted from the \texttt{Course} table, without removing its dependencies in
\texttt{Enrolment},   the latter will contain active references to the deleted
course.  Another example of a dangling reference occurs during insertion of
data,   where a new student is entered in the \texttt{Enrolment} table,   with a
\texttt{CourseID},   that does not exist in the \texttt{Course} table (i.e.,  
wrong \texttt{CourseID}).  A dangling reference occurs because this inserted
student refers to a nonexistent course.
Such problems violate data integrity and cause inconsistent data to be stored in
databases.  In order to ensure that users get consistent and valid information,
applications  have to implement mechanisms to check or prevent dangling
references.  However, if referential integrity constraints are applied as in
\ac{NoSQL} \acp{DBMS}, as in  \acp{RDBMS},   operations on data that  adversely
affect referential integrity will not be permitted.

As previously mentioned,   \ac{NoSQL} \acp{DBMS} do not normalise data and nor
are any relationships maintained.  However, relationships or dependencies
between data are common when real world data is stored in databases.  For
example,   in the real world,   a course can be taught by more than one lecturer
or a student with an Art major is restricted entry into Chemistry courses etc. 
These relationships and constraints have to be preserved upon storage in cloud
\ac{NoSQL} database systems as well.  As mentioned in
Section~\ref{s:cloud-databases},   cloud databases (both relational and
\ac{NoSQL},   have to replicate data across several machines and need to be
scalable to match the needs of the applications.  The replicated and distributed
nature makes maintaining data dependencies complex and unfeasible in terms of
speed and efficiency.  In cloud \ac{NoSQL} \acp{DBMS},   this effectively means
that the relationship between \texttt{Enrolment}, \texttt{Student} and
\texttt{Course}  will not be strictly enforced and deleting a course in
cloud \ac{NoSQL} \acp{DBMS} is allowed because of the absence of constraints.  As
mentioned before,   this means that students could still be enrolled in deleted
courses as there are no constraints to prevent such deletions or changes in
cloud \ac{NoSQL} \acp{DBMS}.

Commonly,   developers impose such constraints and reference integrity checks on
\ac{NoSQL} data at the application side.  Another way to implement such checks
is to impose these constraints at the persistence layer of the application
server.
 Both these ways  eventually have to handle all the processing and managing
of these constraint checks for all the widely spread data in \ac{NoSQL}
\acp{DBMS}. However,   this could mean immense workload on the application or
the application server,   especially if the data volume is large in the \ac{NoSQL}
database or if it is has many replicas that have to be checked.


This is a serious problem when data is interconnected and dependant on other
data entities as is commonly the case.  For example,   consider a banking
application that uses cloud \ac{NoSQL} \acp{DBMS} where its data is
interconnected and spread across several nodes.  Any debit or credit
transactions made to a customer's account will have to be replicated across all
the nodes and correctly persisted. In such applications, many constraints will
exist for transfer of funds between user accounts and such constraints need to
be validated correctly.
 If a user has multiple accounts,   the relationship between the accounts have
to be maintained.
% Moreover, update operations may not be correctly reflected across all the
% nodes of the database due to eventual consistency, which is discussed in
% Section~\ref{s:Background-Cassandra}.
% However, in spite of eventual consistency,   such constraints must be
% correctly handled and recorded.
When such constraints are not validated correctly,   it leads to incorrect
account balances and wrong updates in the user accounts.  On the other hand,  
when such applications use an \ac{RDBMS}, referential integrity constraints are
imposed to maintain the relationships between the accounts. In such cases, the
referential integrity constraints are defined when tables are created and
validations are triggered whenever any operations are performed on the data.

Although such problems  affect most applications using cloud \ac{NoSQL}
\ac{DBMS}, its impact is application dependant.  For instance,
a banking system as mentioned above could be gravely affected because of
dangling references while in a simple game application such problems can be
trivial.
% Addressing these challenges, implies to introduce referential integrity
% constraints in cloud \ac{NoSQL} database systems.
Motivated by such problems of data dependencies, this thesis studies the
existing modelling of data dependencies in cloud \ac{NoSQL} \acp{DBMS} and
contributes by proposing four solutions to effectively maintain and validate
referential integrity.
% while also not limiting the benefits of  \ac{NoSQL} \acp{DBMS}.
% In order to maintain referential  integrity in a database system, some rules
% and validations are performed when operations are executed on data items
% within a database.
The following section describes referential integrity and the rules that have to
be imposed within a database system to validate referential integrity.
\vfill
