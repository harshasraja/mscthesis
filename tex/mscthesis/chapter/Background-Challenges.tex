%ब
\section{Challenges in Key-Value model}\label{s:challenges-key-value}
Fundamentally,   the key-value data model is different from the relational model
in many ways.  While the relational data model aims at giving data a structure,  
providing data integrity,   the key-value data model just
store data as \acp{blob} or string values and generally do not maintain
relationships between data.  In the column-oriented key-value model,   the
key-value association and the grouping of columns in column families can be
considered as the minimum relationship that is maintained.

According to \todo{Bell and Brockhausen (1995)},   data dependencies are the
most common types of semantic constraints in relational databases and these
determine the database design.  Data dependencies are the various relationships
that may exist between data entities in a database.  For example,   in the
University database,   a student can enrol into more than one course and this
means that there is a many-to-many relationship between \texttt{Student} and
\texttt{Course} since   one course can have many students enrolled in it.

As seen in Section~\ref{s:key-value-data-model},   the \texttt{Enrolment} table
contains the \texttt{StudentID} and the \texttt{CourseID} as foreign keys, thus
showing the dependency or relationship between students and courses
(Figure~\ref{f:RDB}).
In the \texttt{University} \ac{RDB} any attempt to delete a course from the
\texttt{Course} table,   is prevented by a constraint,   unless the dependency
itself is removed first.  In \acp{RDBMS} ,   this constraint is referential
integrity,   which ensures that references between data entities are valid,
consistent and intact (\todo(cite Blaha,   n. d. )).
Normalisation,   as well as modelling real world data and relationships enforce
such dependencies in the schema and this causes integrity constraints like
referential integrity constraints,   to be imposed on data entities.

If such constraints are not imposed,   there could arise many dangling
dependencies in the database.  For instance,   consider the case of foreign key
references between \texttt{Course} and \texttt{Enrolment} in the  University
database.  If a course is deleted from the \texttt{Course} table without
removing its dependencies in \texttt{Enrolment},   the latter would contain
active references to the deleted course.  Another example of a dangling
reference occurs during insertion of data,   where a new student is entered in
the \texttt{Enrolment} table,   with a \texttt{CourseID},   that does not exist
in the \texttt{Course} table (i. e. ,   wrong \texttt{CourseID}).  A dangling
reference occurs because this inserted student refers to a nonexistent course.
Such problems cause inconsistent data to be stored in databases and violate data
integrity.  To ensure that users get consistent and valid information,
applications would have to implement mechanisms to check or prevent dangling
references.  But if referential integrity constraints are applied as in an
\ac{RDB},   operations on data that  adversely affects referential integrity  
would not be permitted.

As previously mentioned,   \ac{NoSQL} database systems do not normalise data and
nor are any relationships maintained.  But relationships or dependencies between
data are common when real world data is stored in databases.  For example,   in
the real world,   a course could be taught by more than one lecturer or a
student with an Art major is restricted entry into Chemistry courses etc.  These
relationships and constraints have to be preserved upon storage in cloud
\ac{NoSQL} database systems too.  As mentioned in
Section~\ref{s:cloud-databases},   cloud databases,   whether relational or
\ac{NoSQL},   have to replicate data across several machines and need to be
scalable to match the needs of the users.  The replicated and distributed nature
makes maintaining data dependencies complex and unfeasible in terms of speed and
efficiency.  In cloud \ac{NoSQL} databases,   this effectively means that the
relationship between \texttt{Enrolment}, \texttt{Student} and \texttt{Course}
tables will not be strictly enforced and deleting a course in cloud \ac{NoSQL}
databases is allowed because of the absence of constraints.  As mentioned
before,   this means that students could still be enrolled in deleted courses,  
since there are no constraints to prevent such deletions or changes in cloud
\ac{NoSQL} databases.

Commonly,   developers impose such constraints and reference checks on
\ac{NoSQL}
data at the application side.  Another way to implement such checks is to give
these constraints at the persistence layer of the application server.  Both
these ways would eventually have to handle all the processing and managing of
these constraint checks for all the widely spread data in \ac{NoSQL} databases.
However,   this could mean immense workload on the application or the
application server,   especially if the data volume is large in the \ac{NoSQL}
database or if it is has many replicas that have to be checked as well for the
constraints.


This is a serious problem when data is interconnected and dependant on other
data entities as is commonly the case.  For example,   consider a banking
application that uses cloud \ac{NoSQL} \acp{DBMS} where its data is spread
across several nodes and is interconnected.  Any debit or credit transactions
made to a users account will have to be replicated across all the
nodes and correctly persisted.  Many constraints will exist for transfer of funds
between user accounts and such constraints need to be validated correctly.  If a
user has multiple accounts,   the relationship between the accounts have to be
maintained too.  When such constraints are not validated correctly,   it will lead
to incorrect account balances and wrong updates in the user accounts.  On the
other hand,   when such applications use an \ac{RDBMS},   referential integrity
constraints would be imposed to maintain the relationships between the accounts
and such constraints would be defined while tables are created and their
validation would be triggered whenever any operations are performed on the data.

Updates may not be correctly reflected across all the nodes of the database due
to  eventual consistency too,   which is discussed in Section~\ref{s:Cassandra}.
However, in spite of eventual consistency,   data dependencies should be
correctly handled and recorded.

Although such problems mostly affect most cloud \ac{NoSQL} \ac{DBMS} users,   it
could be different for different users.  For example, a banking system as
mentioned above could be gravely affected because of dangling references while
in a simple game application such problems could be trivial. 
% Addressing these challenges, 
%  implies to introduce referential integrity constraints in cloud \ac{NoSQL}
% database systems.
Motivated by such
problemd of data dependencies, this thesis studies the existing modelling of
data dependencies in cloud \ac{NoSQL} \acp{DBMS} and contributes by proposing
four solutions to effectively maintain and validate referential integrity.
%  while also not limiting the benefits of  \ac{NoSQL} \acp{DBMS}.
% In order to maintain referential  integrity in a database system, some rules and
% validations are performed when operations are executed on data items
% within a database. 
The following section describes referential integrity and the rules that have to
be imposed within a database system to validate referential integrity.

