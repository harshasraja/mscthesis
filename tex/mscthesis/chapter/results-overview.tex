%ब 
\section{Overview of Results} \label{s:results-overview}

The experiments were performed to evaluate the response time and throughput of
the solutions in order to determine the impact of the metadata storage and
referential integrity validations on the performance of Cassandra. 
% The results from these experiments are presented in
% Tables~\ref{tres:ResponseTime}, ~\ref{tres:ResponsetimeRatio}, ~\ref{tres:Throughput}
% and~\ref{tres:ThroughputRatio}. 
Tables~\ref{tres:ResponseTime} and~\ref{tres:Throughput} present the mean and
standard deviation of the average response time for all the solutions and the
throughput of each operation for each solution.  Notice that the solution with the lowest
response time and highest throughput has a better performance than rest,  while
the solution with the highest response time and lowest throughput has the worst. 
This means that the better performing solution can complete more operations in a
second and excute each operation with the least amount of time. 



As seen in these tables,  Solution~4 performs the best amongst all since its
response times for all the  operations on every entity are the least. 
Conversely,  Solution~3 performs the worst amongst all with high response times
for all the operations and the lowest throughput in all the cases respectively. 
Regarding Solutions~1 and 2,  they perform similarly although Solution~1 is
faster with slightly smaller response times and higher throughput.  Note that
Solution~4 is slower than  Baseline  because the baseline does not perform any
referential validations or store metadata. 

% The results show the throughput and the response time of each solution to
% complete an operation with referential integrity validation on a single
% entity.  The throughput shows how many operations are performed in one second
% in each solution. 
This can be further seen in the ratios of the response time and throughput
presented in Tables~\ref{tres:ResponsetimeRatio} and~\ref{tres:ThroughputRatio}. 
The former shows the ratio of the response time of each solution when compared
to that of the baseline,  and it indicates the factor by which any solution is
slower than the baseline;  while
% many times faster or slower each solution is when compared to the baseline. 
the latter shows the ratio of the throughput when compared to that of the
baseline. 
% The results show that all the solutions perform differently and have different
% response times and throughput. 
The variations in the performance of the solutions is due to ways these store
and handle metadata.  Recall that Solutions~1 and 2 store metadata along with the
actual data where Solution~1 stores it in every row and Solution~2 stores it as
the top row of a column family.  On the other hand,  Solutions~3 and 4 store
metadata separately from the actual data  in a \texttt{Metadata} column family
but such a  \texttt{Metadata} column family is in a separate cluster in
Solution~4. 
% The results show that the way metadata is stored and retrieved in each solution
% affects its performance during validation,  when constraints are retrieved and
% used. 

From these results,  it can be seen that Solution~4  is faster than the other
solutions when performing the validations  since it caches the list of
constraints and avoids connecting to the external cluster to access the
\texttt{Metadata} column family each time  operations are invoked on entities. 
Therefore,  to locate the relevant \ac{FK} and \ac{PK} constraints of an entity, 
the constraints stored in the cache memory are re-used. 
Performance is improved significantly just by caching the 
\texttt{Metadata} column family as it reduces the number of accesses to the
column family. 

On the other hand,  Solution~3 is the slowest  because of the
way it accesses the metadata every time from \texttt{Metadata} column family. 
% of the way the metadata is accessed for the entities in this solution. 
% % irrespective of whether an entity is a parent or child the %
% \texttt{ValidationHandler} performs the same check the metedata of the entity
% % for all the solutions.  It is when this check is made it is clear if the
% entity % is a parent or child. 
% In this solution accessing
In this solution,  in order to retrieve the relevant \ac{FK} constraints for
an entity the \texttt{Metadata} column family has to be accessed.  This means
that for each operation on an entity,  an additional access is required to
\texttt{Metadata} which consumes more time. 
Moreover,  to retrieve the information about any referencing constraints within
the relevant \ac{FK} constraints of an entity,  \texttt{Metadata} is accessed
again. 
% For example,  in order to get the information of a \ac{PK} constraint stored in the
% \texttt{RConstraintName} column of a \ac{FK} constraint,  the \texttt{Metadata}
% column family is again accessed and the \ac{PK} constraint is searched.  
Thus,  in order to
complete each validation,  \texttt{Metadata} is accessed more than once. 
% \texttt{Metadata} column family has to be accessed using the connection
% object. 
Unlike Solution~4,  metadata is not cached for re-use thus costing multiple
access to \texttt{Metadata} column family. 

Meanwhile,  Solutions~1 and~2 have approximately similar response times as both
the solutions store the whole list of constraints with the actual data and 
requires no additional connections to access the relevant as well as
referenced constraints of an entity.  Note that, in Solution~1 since the constraints
are stored with each entity it performs slightly better than
Solution~2 because the latter has an additional search operation to identify the
top row in a column family to locate the relevant constraints of an entity.  Both
solutions are faster than Solution~3 mainly because these have the whole
list of constraints along with the actual data.  However,  they are slower than
Solution~4 as these have to access the constraints from each entity every time and do not
use a cache. 
% their metadata access for these solutions are easier as metadata is a part of
% the entity and no additional connection to a metadata column family is
% required. 


% Unlike this,  Solution~4 caches  metadata for entities and re-uses it thus
% saving time by not having to access a separate column family for each entity
% insertion. 

The performance of the solutions in each
operation performed on the entities is discussed in detail in the following
sections.  Notice that,  in all the tables anf figures,   the
entities \texttt{Student},  \texttt{Course} and \texttt{Enrolment} are referred
to as '\texttt{s}',  '\texttt{c}' and '\texttt{e}' for the sake of brevity. 
