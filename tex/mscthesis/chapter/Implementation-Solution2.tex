%ब
\section{Solution 2:  Metadata as a Top Row}\label{s:Implementation-Solution2}
	

In Solution~2,  metadata is saved as a top row (i.e. the first super column in a
column family) with the unique \texttt{RowId} '\texttt{-1}', having  only
\texttt{Metadata} column containing the metadata. The metadata
 is retrieved, processed and accessed when referential integrity is
validated for an entity and  is explained in the following sections.


\subsection{Metadata Retrieval}

The metadata in the top row of a column family contains all its relevant
constraints and  this metadata is processed and its values  individually
retrieved prior to validation. When an operation is invoked on a column family ,
its metadata is fetched from the top row by searching for the \texttt{RowId} '\texttt{-1}' in
the column family. This metadata is then read as a String by the
\texttt{AbstractEntity} and parsed using the special characters used within the
metadata, which is similar to Solution~1.
Like Solution~1, the values extracted from the constraints are stored as
attributes in the \texttt{Metadata} entity class. The \texttt{Metadata} entity
class is the same as shown in Figure~\ref{fi:MetadataEntityClass} since the
structure of the metadata is consistent across solutions. Using the getter
methods of \texttt{Metadata}, the \texttt{AbstractEntity} sets these tokenized
values as attributes of  \texttt{Metadata}  for the duration of the operation.


\subsection{Metadata Access}

Once  metadata is retrieved, processed and set as attributes of the
\texttt{Metadata} entity class, it is available to be accessed by the handlers
of the \ac{API}. Appropriate getter methods in \texttt{Metadata} is used by the
\texttt{ValidationHandler} to access the different parts of the constraints,
which is similar to Solution~1.

The referential integrity validation and the \ac{CRUD} operations in this
solution are implemented as described in Section~\ref{s:implementation-API}.
% Since the metedata for each entity is parsed and converted ot an entity by the
% \texttt{AbstractEnityt}, the \texttt{ValidationHandler} uses the appropriate
% getter methods to retrieve each part of the metedata. For instance, to get
% information of the
% \texttt{DeleteRule} of a constraint on the entity, \texttt{ValidationHandler}
% uses the \texttt{getDeleteRule} method of the metadtaa entity class and likewise
% for all the other different parts of the metadata respective getter methods are
% used.
	
% 	\subsection{Motivation for the approach}
% 	The motivation for this solution was to overcome the redundancy of metadata
% 	storage in Solution~1. In solution~1 metedata was stored in every super column
% 	of a column family and replicated across the cluster along with the column
% 	family. Solution~2 reduces this redundancy and centralises the
% 	metedata as a top row within the column family and is common for
% 	all instances in an entity class. This solution also ensures that when changes have to be made ot
% 	the metadata the actual data is not accessed and only the column family of the
% 	data is accessed.  Moreover when metadata is large, it
% 	consumes less space as it is not replicated as widely as in Solution~1. 
% 	In other words, the aim of this design was to reduce the redundancy of metadata
% 	in the column family whilst having metadata accessible to all the nodes by
% 	having it embedded with the actual data.
% 	This involves using a lot of disk space for storing the
% 	metadata, especially considering the number of times an entity class is
% 	replicated over a cluster of nodes. When the number of entity classes and
% 	\ac{FK} constraints between them are large, the size of the metadata also
% 	increases. Such large metadata will involve more parsing operations and when
% 	the metadata is repeated in every super column and widely replicated it is
% 	redundant use of storage space.
