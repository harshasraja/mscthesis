\section{Solution 2:  Metadata as Top Row}\label{s:Implementation-Solution2}
	
% 	\subsection{Metadata Storage}
% 	This solution saves the metadata embedded with the actual data and exists in
% 	the same column family as the data. This approach is similar to
% 	Solution~1 where metadata is stored within the same column family. But unlike
% 	Solution~1, this solution saves the metadata only once in the column family as a
% 	top row or the first super column in a column family with the unique
% 	\texttt{RowId} '\texttt{-1}'. This top row has only a single column 
% 	\texttt{Metadata}  containign the metadta information and this is  unlike
% 	other super columns which have different columns. This is possible since
% 	Cassandra allows rows to have different number of columns, as described in
% 	Section~\ref{s::key-value-data-model}. Thus, for each entity class, the
% 	metadata exists only once as a single row and is common for all the instances
% 	of the entity.
% 	In the University example  a \texttt{Student} entity has the
% 	metadata  stored as a top row in Figure~\ref{}.
% 		 
% 	\begin{figure}
% 	\todo{ Insert metadata top row figure for Student}
% 	\end{figure}
% 		
% 	The metadata in this solution contains
% 	special characters '\texttt{\{}', '\texttt{\}}','\texttt{;}' and '\texttt{:}'
% 	to distinguish all the constraints and its different parts and values.
% 	The metadata stored for each column family has the following
% 	constraints:
% 	\begin{itemize}
% 	  \item  \ac{PK} constraint showing the primary key of the column family.
% 	  \item \ac{FK} constraints 
% 			\begin{itemize}
% 				\item In the case of a parent entity the \ac{FK} constraints are the \ac{FK}
% 				constraints of type '\texttt{F}' to identify the child entities when the entity
% 				is being updated or deleted.
% 				\item  In the case of a child entity the \ac{FK} constraint of type '\texttt{R}'
% 				would be stored along with the \ac{PK} constraint which indicates its parent
% 				entities.
% 				\item If an entity is both a parent and child entity, then its metadata would
% 				hold its \ac{PK} constraint, and the \ac{FK} constraints of both types.
% 			\end{itemize}
% 	\end{itemize}
% 	
% 	Consider \texttt{Enrolment}  which is a child entity with parent entities
% 	\texttt{Student} and \texttt{Course} (Figure~\ref{}). Its metadata thus
% 	contains its \ac{PK} constraint \texttt{CONST300} and the \ac{FK} constraints
% 	\texttt{CONST400} and \texttt{CONST500}. Since \texttt{Student} is a child
% 	entity it  stores its \ac{PK} constraint \texttt{CONST100} and its \ac{FK}
% 	constraint \texttt{CONST700}. Similarly, \texttt{Course} stores its \ac{PK} and
% 	respective \ac{FK} constraints.
% 
% 	\begin{figure}
% 	\todo{ Insert metadata top row figure for Enrolment}
% 	\end{figure}
% 	Thus, the metadats describes which is the
% 	primary key for the entity and the \ac{FK} constraints show which child entities dependent on the
% 	entity. The \ac{FK} constraints are particularly useful to .
	
	\subsection{Metadata Retrieval}
	Since the metadata holds all the constraints and its various parts, the
	\ac{API} needs to extract each constraint and its different parts to validate
	referetntial integrity for an entity. For this it performs \texttt{String}
	parsing as described in Section~\ref{s:sol1-real} since metedata is read as a
	\texttt{String} while it is retrieved.
	
	In this solution, an enity class
	for metadata with the required getter and setter access methods is created and
	used to save the metadata of an entity whenevr any operation is invoked on it.
	For every operation on an entity the \texttt{AbstractEntity}
	reads its metadata from the top row of the entity class as a \texttt{String}
	and each part of the metadata is tokenised to column names and its values.
	The \texttt{AbstractEntity} treats metadata just like an
	entity and sets the tokenised values using the metadata entity's setter
	methods.
	The parsing of the metadata and its conversion to an entity using the access methods is the same as
	in Solution~1. The special characters become the delimiters for parsing the
	metadata just like in Solution~1
	
	\subsection{Metadata Access}
	The metadata is accessed by the \texttt{ValidationHandler} to
	validate referential integrity for entities. Since the metedata for each entity is
	parsed and converted ot an entity by the \texttt{AbstractEnityt}, 
	the \texttt{ValidationHandler} uses the appropriate getter methods to retrieve
	each part of the metedata. For instance, to get information of
	the \texttt{DeleteRule} of a constraint on the entity,
	\texttt{ValidationHandler} uses the \texttt{getDeleteRule} method of the
	metadtaa entity class and likewise for all the other different parts
	of the metadata respective getter methods are used.
	
% 	\subsection{Motivation for the approach}
% 	The motivation for this solution was to overcome the redundancy of metadata
% 	storage in Solution~1. In solution~1 metedata was stored in every super column
% 	of a column family and replicated across the cluster along with the column
% 	family. Solution~2 reduces this redundancy and centralises the
% 	metedata as a top row within the column family and is common for
% 	all instances in an entity class. This solution also ensures that when changes have to be made ot
% 	the metadata the actual data is not accessed and only the column family of the
% 	data is accessed.  Moreover when metadata is large, it
% 	consumes less space as it is not replicated as widely as in Solution~1. 
% 	In other words, the aim of this design was to reduce the redundancy of metadata
% 	in the column family whilst having metadata accessible to all the nodes by
% 	having it embedded with the actual data.
% 	This involves using a lot of disk space for storing the
% 	metadata, especially considering the number of times an entity class is
% 	replicated over a cluster of nodes. When the number of entity classes and
% 	\ac{FK} constraints between them are large, the size of the metadata also
% 	increases. Such large metadata will involve more parsing operations and when
% 	the metadata is repeated in every super column and widely replicated it is
% 	redundant use of storage space.
