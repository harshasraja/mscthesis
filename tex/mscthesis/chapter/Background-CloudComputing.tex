%ब
\section{Cloud Computing} \label{s:cloudComputing}
Cloud computing is a major paradigm that is rapidly shifting the way \ac{IT}
services and tools are being used in the industry.  It is perceived that cloud
computing would help extend the capabilities of many \ac{IT} and online services
without the need for costly infrastructure.

Similar to remote computing where other machines or computers are accessed from
the local machine through a network,   cloud computing leverages network
connections to provide various services to the users.  It also brings with it
the virtualization of applications and services,   where it appears to users as
if the applications are running on the user's machine rather than a remote cloud
machine (\todo{Cloud Computing Defined,   2010)}.  This removes the need for
installing the actual software by the users.  Thus,   both expert and naive
users need not worry about the technical details and configurations to use these
cloud services.

Cloud computing is generally based on a subscription model where users pay as
per their usage,   which is very similar to utility services like electricity,  
gas or water etc.  The coalescence of virtualization,   where applications are
separated from the infrastructure is what makes cloud computing easy to use.
Users do not have to invest in software applications as they can access such
applications on the cloud.  Users pay only for the services they use.  For
example,   they pay only for the amount of storage their cloud database uses or
pay only for the bandwidth consumed by the servers they rent from the cloud
providers.  Applications and databases are stored in large server farms or data
centres owned by companies like Google,  IBM etc.

The architecture of cloud computing services has users who avail cloud services
as the front-end.  The back end of the architecture includes the cloud servers,
databases,   and computers etc. ,   which are abstracted from users.  All the
components like the servers,   applications,   the data storages work together
through a web service to provide the users with the cloud services. 

The overall structure of cloud computing and its various services have been
generalised into layers (\todo{ZakiSabbagh,   2010,   Bime,   2008}). 

\begin{description}

	\item [User:] is any hardware or software application that relies on cloud
	computing to perform its work.  Generally,   'client' refers to any software applications or
	\acp{API}  that are used to perform cloud computing,  
	while 'users' represents the end-users,   like  database administrators or
	programmers or anyone who benefits from cloud computing services. 
	
	\item [\acf{SaaS}] is the service provided by the cloud
	providers where users do not have to install the software applications. 
	
	\item [\acf{PaaS}] is the service where a hardware or
	software platform is provided to users.  A platform could be an operating system,  
	programming environment,   hardware,   run-time libraries etc. 
	
	\item [\acf{IaaS}] is the service where users can use
	the expensive hardware like network equipments,   servers etc. 
	
	\item [\acf{DaaS}] is a cloud storage service  that represents
	the storage facilities,   like  \acp{DBMS} which are provided
	as cloud services for which users pay only for the storage space they use
	(\todo{ Wu et al.,  2010)}. 

\end{description}


\ac{DaaS} involves hosting cloud databases in the cloud which offer data
management,   data retrieval,   and other database services.  Due to the
increasing number of users deploying and using web applications on cloud,  
cloud databases form a crucial part to store the increasing amounts of data
on the cloud.  Many companies like Amazon,   Google,   IBM,   and Microsoft
provide \ac{DaaS} and offer varying levels of services (\todo{Mateljan et al. ,  
2010)}. The next section gives a description about cloud databases and its key
features.