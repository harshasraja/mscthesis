%ब
\section{Experimental setup}\label{sexp:ExperimentalSetup}

% Since it is not a
% controlled environment, the experimentation involves performing several runs
% where data is inserted, updated, and deleted on the Cassandra cluster. 
% The
% environment is not controlled as variables like network latency, parallel
% processes in the nodes, and other variables are expected to affect the
% performance.


The experimentation consists of performing \ac{CRUD} operations upon artificial
data created for the University example application. Specifically, the
operations of interest are \texttt{Create}, \texttt{Update} and \texttt{Delete}
as these are the ones that trigger referential integrity validations. Notice
that, since the experiments are not performed in a controlled environment, all
the operations are repeated 100 times such that the effect of external factors
(e.g. network latency, parallel processes running in nodes, etc.)  is minimized.


The artificial data upon which operations are performed  is made up of 500
students, 500 courses, and 5000 enrolments. These numbers were chosen such that
the experiments could be performed in a reasonable amount of time. 
The format of the artificial data is:

	\begin{itemize}
		\item \texttt{Student} has a
		unit-increasing \texttt{StudentId}  which is merged into the fields \texttt{FirstName}
		 and \texttt{LastName} as ``First Name (StudentId)'' and ``Last Name
		(StudentId)''.  \texttt{Email} is composed in a similar way as
		``First.Last@email.(StudentId).com'' and \texttt{Age} is a random number
		between 18 and 60.
		
		\item  \texttt{Course} has a unit-increasing \texttt{CourseId} which is
		appended to the prefix ``COMP''.  It also has a composed \texttt{CourseName}
		as ``Engineering (CourseId)''.  \texttt{Trimester}, \texttt{Level}
		and \texttt{Year} are randomly generated numbers.
		
		\item  \texttt{Enrolment} contains a unit-increasing \texttt{RowId}  and the
		respective foreign keys of student and course,  which are \texttt{StudentId}
		and \texttt{CourseId}. 
	\end{itemize}
 
		
The order of the operations to be performed on the data in each run is as
follows.
\texttt{Create} inserts all the entities for \texttt{Student},  \texttt{Course}
and \texttt{Enrolment}.  \texttt{Update} performs changes on the primary keys of
\texttt{Student} and \texttt{Course} entities,  and on the foreign keys of
\texttt{Enrolment} (the one relative to courses,  specifically).  Finally, 
\texttt{Delete} removes all the \texttt{Student},  \texttt{Course} and
\texttt{Enrolment} entities.
Notice that the primary keys in every column family are different in each run
(create,  update,  delete), in order to avoid introducing biases to the results
as product of the tombstone delete paradigm that Cassandra utilizes.  That is, 
since Cassandra does not completely  remove the primary keys of the inserted
entities (tombstone delete),  reinsertion  using the same primary key might
yield faster times as the key already exists.  After each run,  all the column families
(\texttt{Student},  \texttt{Course},  and \texttt{Enrolment}) are emptied and
ready for the next run.   The details  of the \texttt{Create}, \texttt{Update}
and \texttt{Delete} operations are explained further in the following sections. 
		
% The operations to be performed on each run are as follows. Firstly,
% a batch of students, courses and enrolments are inserted in that precise order.
% Secondly, foreign keys of enrolments and primary keys of courses and students
% are updated. Lastly, enrolments, students and courses are deleted. Thus, having
% their respective column families emptied and ready for the next run. Notice
% that by empty, it refers to the fact that the column values of each row are
% emptied, but not the row key as the deletion conforms to the tombstone paradigm
% explained in previous chapters. The next run ensures to increase the
% primary key of the entities such that they do not exist on the column families.

	
\subsection{Create}
The \texttt{Create} operation inserts all the
\texttt{Student},  \texttt{Course} and \texttt{Enrolment} entities in that
precise order due to the nature of the referential integrity constraints
presented in Table~\ref{texp:ListConstraints}.   In the \texttt{Student} and
 \texttt{Course} column families,  \texttt{Create} does not trigger any
 referential integrity validation as these entities do not contain foreign keys. 
 Contrarily,  \texttt{Create} on \texttt{Enrolment} triggers foreign key
 validation checks on both \texttt{Student} and \texttt{Course} column families. 
		
\subsection{Update}
The \texttt{Update} operation is performed after the creation of all entities.
First,  an attempt is made to update the primary key of each \texttt{Course}
entity.  This operation triggers referential integrity  validations that result
in exceptions thrown as the \texttt{DeleteRule} \footnote{Notice that for the
sake of simplicity, this rule is also used for  update operations.} for all
\texttt{Course} entities is \texttt{NoDelete} and enrolments referencing the
courses are still present in \texttt{Enrolment}. Hence, the times recorded for
updating the \texttt{Course} column family represent the time required to
identify a constraint violation and throw the respective exceptions.
					
Next,  the \texttt{Enrolment} column family is updated.  In this case,  the
\texttt{CourseId} for each \texttt{Enrolment} entity is changed to a different
existing value,  ensuring that the distribution of \texttt{Student} and
\texttt{Course} entities remains the same. The update on the \texttt{Enrolment}
column family triggers referential integrity validation checks to ensure that
the course to which every \texttt{Enrolment} entity is being updated actually
exists in \texttt{Course} column family. 
					
Finally,  the primary key for each \texttt{Student} entity is updated to a new
integer value that has never existed in the column family. Thus, given
the \texttt{DeleteRule} for \texttt{Student} (i.e. \texttt{Cascade}),  this
operation triggers a cascaded update on the \texttt{Enrolment} column family   
 by respectively updating the student foreign key (\texttt{StudentId}) in all
 its existing \texttt{Enrolment} entities.
		
\subsection{Delete} 
The deletion of entities occurs first on the
\texttt{Enrolment} column family,  where all of its records are deleted without
requiring referential integrity checks as this is a child entity.  The times are
recorded for each \texttt{Delete} operation and then all of the entities are
reinserted with the same primary keys in order to assess the cascaded
\texttt{Delete} of \texttt{Student} entities next. 
				
Secondly,  all  the \texttt{Student} entities are deleted from the
\texttt{Student} column family. Hence, given the \texttt{Cascade}
\texttt{DeleteRule} of these entities,  the \texttt{ValidationHandler} ensures
to delete first all of the child entities before deleting a \texttt{Student} entity. 
Thus,  the times recorded for this operation also include the time required for
performing a cascaded \texttt{Delete} on the student dependencies in
\texttt{Enrolment}.
Notice that the dependencies exist at this point as they will have been reinserted into
\texttt{Enrolment} in the previous step. 
				
Finally,  all the \texttt{Course} entities are deleted.  Despite the courses
having a \texttt{NoDelete} rule,  notice that at this point the
\texttt{Enrolment} column family is empty,  so courses can be deleted as there
are no child dependencies.  Thus,  the times recorded for this operation measure
referential integrity validation as well as the \texttt{Delete} operation
of the \texttt{Course} entity.  After this final operation,  all column families
are emptied but all the primary keys still exist due to Cassandra's tombstone
delete.  However,  the whole keyspace is ready for the next batch of operations as
the primary keys of all column families will be different. 
	
	