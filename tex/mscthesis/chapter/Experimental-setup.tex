%ब
\section{Experimental setup}\label{sexp:ExperimentalSetup}


The experimentation involves performing 100 runs where artificial data is
randomly generated for the attributes of each column family.  
This data is then inserted, 
updated and deleted from column families in the Cassandra cluster. 

On each run,  the time required for each operation is recorded in order to assess
the response time of each solution as well as the throughput in terms of
operations per unit of time.  Notice that each operation on the artificial data 
is performed in a batch,  and entities from each column family are randomly
sorted before any operation takes place.  Random sorting is performed in order to
prevent the results to be biased from possible optimization made by Cassandra in
terms of indexes or other criteria. 
		
The artificial data is made up of 1000 students,  1000 courses,  and 10000
enrolments which is the result of assigning 10 different courses to each
student.  Courses are assigned by dividing the number of courses
into 100 groups of 10 courses each,  and assigning a group for each student. 
Notice that such an assignment involves that X students have the same courses
assigned.  The quantity of records to be inserted for each entity was chosen
considering an overall reasonable time for completing the experimentation of all
solutions. 
		
The format of the artificial data created is as follows.  
	\begin{itemize}
	  
		  \item \texttt{Student} has a
		unit-increasing \texttt{StudentId},  which is merged into the fields \texttt{FirstName}
		 and \texttt{LastName} as "First Name (StudentId)" and "Last Name
		(StudentId)".  \texttt{Email} is composed in a similar way as
		``First. Last@email. (StudentId). com'' and \texttt{Age} is a random number. 
		
		\item  \texttt{Course} has a unit-increasing \texttt{CourseId} which is
		appended to the prefix "COMP".  It also has a composed \texttt{CourseName} as
		in \texttt{Student} (merging id and field).  \texttt{Trimester},  \texttt{Level}
		and \texttt{Year} are randomly generated numbers. 
		
		\item  \texttt{Enrolment} contains a unit-increasing \texttt{RowId},  and the
		respective foreign keys of student and course,  which are \texttt{StudentId}
		and \texttt{CourseId}. 
		
	\end{itemize}

		
The order of the operations performed on the data is as follows.  \textbf{Create}
inserts all the entities for \texttt{Student},  \texttt{Course} and
\texttt{Enrolment}.  \textbf{Update} performs changes on the primary keys of
students and courses,  and on the foreign keys of \texttt{Enrolment} (the one
relative to courses,  specifically).  Finally,  \textbf{Delete} removes all the
\texttt{Student},  \texttt{Course} and \texttt{Enrolment} entities. 
Notice that the primary keys in every column family are different in each run
(create,  update,  delete) in order to avoid introducing biases to the results as
product of the tombstone delete paradigm that Cassandra utilizes.  That is,  since
Cassandra does not completely  remove the primary keys of the inserted entities
(tombstone delete),  reinsertion  using the same primary key might yield faster
times as the key already exists.  After each run,  all column families
(\texttt{Student},  \texttt{Course},  and \texttt{Enrolment}) are emptied and
ready for the next run.   The details  of the \ac{CRUD} operations are explained
further in the following sections. 
		

	
\subsection{Create} The \texttt{Create} operation inserts all the
\texttt{Student},  \texttt{Course} and \texttt{enrolment} entities in that
precise order due to the nature of the referential integrity constraints
presented in Section~\ref{s:ed:ri}.  The time required to insert all of the
entities in their respective column families  is recorded.  In the
\texttt{Student} and \texttt{Course} column families,  \texttt{Create} does not
trigger any referential integrity validation as these entities do not contain
foreign keys. 
Contrarily,  \texttt{Create} on \texttt{Enrolment} triggers foreign key
validation checks on both \texttt{Student} and \texttt{Course} column families. 
		
\subsection{Update} The \texttt{Update} operation is performed after the
creation of all entities. 
First,  an attempt to update the primary key of each \texttt{Course} entity is
made.  This triggers referential integrity validations that result in exceptions
thrown as the \texttt{DeleteRule} for all \texttt{Course} entities is
\texttt{NoDelete}.  Hence,  the times recorded for updating the \texttt{Course}
column family represent the time required to identify a constraint violation and throw
the respective exceptions. 
					
Next,  the \texttt{Enrolment} column family is updated.  In this case,  the
\texttt{CourseId} for each \texttt{Enrolment} entity is changed to a different
one,  ensuring that the distribution of courses and students remains the same. 
The update on the \texttt{Enrolment} column family triggers referential
integrity validation checks to ensure that the course to which every
\texttt{Enrolment} entity is being updated actually exists in \texttt{Course}
column family. 
					
Finally,  the primary key for each \texttt{Student} entity is updated to a new
integer value that previously  never existed in the column family.  Given the
\texttt{Cascade} \texttt{DeleteRule} for \texttt{Student},  this operation
triggers a cascaded update on the
\texttt{Enrolment} column family.  
%  by respectively updating the student foreign
% key,  \texttt{StudentId} in all its existing \texttt{Enrolment} entities. 
All the dependant \texttt{Enrolment} entities of this \texttt{Student} entity
are respectively updated on its foreign key \texttt{StudentId}. 
		
\subsection{Delete} The deletion of entities occurs first on the
\texttt{Enrolment} column family,  where all of its records are deleted without
requiring referential integrity checks as this is a child entity.  The times are
recorded for each \texttt{Delete} operation and then all of the entities are
reinserted with the same primary keys in order to assess the cascaded delete of
\texttt{Student} entities next. 
				
Secondly,  all  the \texttt{Student} entities are deleted from the
\texttt{Student} column family.  Given the \texttt{Cascade} \texttt{DeleteRule} of these entities,  the
\texttt{ValidationHandler} ensures to delete first all of the child entities
before deleting a \texttt{Student} entity. 
Hence,  the times recorded for this operation measure the time required for
performing a cascaded delete on the student dependencies in enrolment.  Notice
that the dependencies exist at this point as they will have been reinserted into
\texttt{Enrolment} in the previous step. 
				
Finally,  all the \texttt{Course} entities are deleted.  Despite the courses
having a \texttt{NoDelete} rule,  notice that at this point the
\texttt{Enrolment} column family is empty,  so courses can be deleted as there
are no child dependencies.  Thus,  the times recorded for this operation measure
referential integrity validation as well as the \texttt{Delete} operation
of the respective entity.  After this final operation,  all column families are
emptied but all the primary keys still exist due to Cassandra's tombstone
delete.  However,  the whole keyspace is ready for the next batch of operations as
the primary keys of all column families will be different. 
	
	