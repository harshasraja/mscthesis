%ब
\section{Performance Indicators} \label{sexp:PerformanceIndicators}
% Performance of database systems is commonly measured in terms of the
% \textit{Response time} and \textit{Throughput}.

In this thesis, response time and throughput are the measures used to gauge the
performance of the four solutions while referential integrity validation is
implemented using the \ac{API}.
Response time refers to the time  a database system takes to process an
operation and produce results to the end user(\todo{cite Demurjian,
Berkely,serverside,}) . Measuring response time for a
database operation is similar to a black-box evaluation because it is measured 
without considering the internal functioning  of the database system. According
to (\todo{cite Demurjian}) such an evaluation is ideal for a complete database
system to measure its performance and to give the users details about its 
efficiency and speed in performing operations. 

Response time for each of the  operations that trigger such a validation from
all the solutions are measured during the experiments.
This included the time involved to access and retrieve metadata for the entities
and also the time for validating referential integrity by the
\texttt{ValidationHandler}. The response time of Cassandra when such validations
are not in place is also measured and considered as a baseline with which to
analyse the solutions. Such a comparison determines the degree of change in
speed of Cassandra when such overheads are introduced and gives users useful
information about how each solution affects the performance of the database
system.

The second performance measure used is \textit{Throughput} which is another
classical and commonly used measure of database performance (\todo{cite
BerkleyDB}). Throughput measures the number of operations processed by the
database system in a unit of time. In the experiments the throughput for all the
operations triggering referential integrity validation across all solutions is
measured as operations per second. A single operation stands for each time an
entity is inserted or updated or deleted.Note that only the operations that
introduce the referential integrity validation in Cassandra is measured and thus
\texttt{read} operations are not measured in terms of response time or
throughput.
% For example, inserting 1000 students means that 1000 \texttt{insert}
% operations are processed by Cassandra.

The popular TPC benchmarks are not considered for the performance of this
experiment mainly because TPC benchmarks are centred around transactions and
OLTP workloads. The principal metrics for these benchmarks are the
transaction rate, query per hour, cost indicators of a system, among
others~\citep{TPC}.
Such indicators are suitable for \ac{DBMS} with ACID properties. Hence, for
assessing Cassandra which is a system that does not support SQL queries or
provide ACID proerties, it is essential to measure it in terms of what is
critical to application  using Cassandra. In this experiment it is critical to
measure the difference in time for an operation to complete in Cassandra when
referential integrity validation is activated or not activated.


% These operations which trigger referential integrity validation for an entity
% namely the \texttt{insert}, \texttt{update}, \texttt{delete} operations are
% were measured in terms of the throughput in the experiments. Throughout
% commonly referes to the number of operations performed

% It has to be noted that the operations are prone to  external factors like
% network latency, bandwidth, network routing, network workload among others
% which typically affect a network consisting of several machines and users.
% This is because the Cassandra cluster used in the experiments is deployed over
% a network that is used by many users concurrently thus exposing the operations
% to such factors. Identifying such factors and analysing them is beyond the
% scope of this thesis and the analysis is strictly in terms of how the metadata
% storage and referential integrity validation affects Cassandra's performance.
It is a general practise for applications to incorporate code within
applications to log the timestamps for transactions in traditional
\acp{DBMS}~\citep{IBMPerformance}.
In the experiments, response time and throughput were measured by logging the
time involved to comeplete each operation in all the solutions..
The real time was recorded before and after a validation is triggred by an
operation. When all the iterations are completed for each entity, the time
measurements are written to an output log file.

In order to determine the response time and throughput, the output log files are
are analysed using R. (\todo{explain how it is imported to R and graphs
produced--SOS Juan!})



Notice that external variables such as network latency, simultaneous processes
in the operating systems of each node, and other variables are not considered
for the analysis of results. Even when they are present, it is expected that
results will not be biased by them. Nonetheless, the experiments will be
performed at night time over a weekend as this is the time where the cluster is
less used, thus reducing the presence of such variables and hence their impact
by biasing the results \todo{or something like that :P}