%बोबिठ्
\chapter{Theoretical Background}



\section{What is Cloud Computing?}
 Cloud computing, a major paradigm, would bring
about a shift in the way IT services and tools are going to be used in the
industry. It is perceived that cloud computing would help extend the
capabilities of many IT and online services without the need for costly
infrastructure. According to IDC (2008), the spending on IT cloud services is
likely to triple in the next 5 years, indicating that cloud computing is here to
stay.

Cloud computing, a by-product of remote computing where other machines or
computers are accessed from the local machine through a network, brings with it
the virtualisation of applications and services. Virtualisation gives users the
feeling that the applications are running on the user's machine rather than the
remote cloud machine (Cloud Computing Defined, 2010), removing the need for
installing the actual software by the users. Both expert and naive users could
thus work with applications without worrying about the technical details and
configurations.

Cloud computing usually is a subscription based model where users would pay as
per their usage, making cloud computing very similar to utility services like
electricity, gas or water etc. The coalescence of virtualisation, where
applications are separated from the infrastructure is what makes cloud computing
easy to use. Users need not invest in buying software applications as they can
access such applications on the cloud. Cloud service providers have large server
farms where applications and databases are stored. Users only pay for the
services they use. For example, they pay only for the amount of storage their
cloud database uses or pay only for the bandwidth consumed by the servers they
rent from the cloud providers. This is most beneficial for the medium and small
enterprises as they need not have huge investments in databases or servers.

The architecture of cloud computing services has users who avail cloud services
as the front-end. The back end of the architecture includes the cloud servers,
databases, and computers etc., which are abstracted from users (Figure 1). All
the components like the servers, applications, the data storages work together
through a web service to provide the users with the cloud services.


	This is how you cite in \LaTeX, 
	According to \citet{pso:rapso:rada-vilela11}\ldots.
	
	By now, you should realize that Microsoft Word sucks
	\citep{pso:spso-apso:rada-vilela11}.
	
	

\section{\ldots}

\blindtext