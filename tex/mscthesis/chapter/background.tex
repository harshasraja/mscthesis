%ब
\chapter{Background}

Cloud computing is a major paradigm that is rapidly shifting the way \ac{IT}
services and tools are being used in the industry. It is perceived that cloud
computing would help extend the capabilities of many \ac{IT} and online services
without the need for costly infrastructure.

Similar to remote computing where other machines or computers are accessed from
the local machine through a network,cloud computing leverages network
connections to provide various services to the users. It also brings with it the
virtualisation of applications and services, where it appears to users as if the
applications are running on the user's machine rather than a remote cloud
machine (Cloud Computing Defined, 2010). This removes the need for installing the actual
software by the users. Thus, both expert and naive users need not worry about
the technical details and configurations to use these cloud services.

Cloud computing is generally based on a subscription model where users pay as
per their usage, which is very similar to utility services like electricity, gas
or water etc. The coalescence of virtualisation, where applications are
separated from the infrastructure is what makes cloud computing easy to use.
Users do not have to invest in software applications as they can access
such applications on the cloud. Users pay only for the services they use. For
example, they pay only for the amount of storage their cloud database uses or
pay only for the bandwidth consumed by the servers they rent from the cloud
providers. Applications and databases are stored in large server farms or data
centres owned by companies like Google, IBM etc.

The architecture of cloud computing services has users who avail cloud services
as the front-end. The back end of the architecture includes the cloud servers,
databases, and computers etc., which are abstracted from users. All
the components like the servers, applications, the data storages work together
through a web service to provide the users with the cloud services.

The overall structure of cloud computing and its various services have been
generalised into layers (ZakiSabbagh, 2010, Bime, 2008).

\begin{description}

\item [User:] is any hardware or software application that relies on cloud
computing to perform its work. Generally, ‘client' refers to any software applications or
\acp{API}  that are used to perform cloud computing,
while ‘users' represents the end-users, like  database administrators or
programmers or anyone who benefit from cloud computing services.

\item [\acf{SaaS}] is the service provided by the cloud
providers whereusers do not have to install the software applications.

\item [\acf{PaaS}] is the service where a hardware or
software platform is provided to users. A platform could be an operating system,
programming environment, hardware, runtime libraries etc.

\item [\acf{IaaS}] is the service where users can use
the expensive hardware like network equipments, servers etc.

\item [\acf{DaaS}] is a cloud storage service  that represents
the storage facilities, like  \acp{DBMS} which are provided
as cloud services for which users pay only for the storage space they use (Wu et
al., 2010).

\end{description}


\ac{DaaS} involves hosting cloud databases in the cloud which offer data
management, data retrieval, and other database services. Many companies like
Amazon, Google, IBM, and Microsoft provide \ac{DaaS} and offer varying levels of
services (Mateljan et al., 2010). More details about \ac{DaaS} and cloud
\ac{DBMS} are provided in Literature Review.

In the remainder of this chapter, Section~\ref{s:cloud-databases} presents cloud
databases. Section~\ref{s:cloud-data-models} presents the prevailing cloud data
models. Section~\ref{s:key-value-data-model} presents the Key Value Data Models.
Section~\ref{s:challenges-key-value} presents the challenges in Key Value Data
models and Section~\ref{s:referential-integrity} discusses the Referential
Integrity Constraint in Key Value data model, which is the focus of this research.


\section{Cloud Databases}\label{s:cloud-databases}

Most cloud applications handle data like the user information, application data 
or some stored data which maybe accessed by the users and storage
of such data is essential (Kennedy, 2009). Traditionally, users store data
in files or databases residing on dedicated database servers or on local disks,
but in cloud computing, data is stored within data centres owned by hosting companies like Google, Amazon etc. Data centres house many servers,
computers and telecommunication infrastructure, including back up and security
facilities and users can rent or buy the storage space they need. 

Within data centres, data could be stored on remote machines, which could be
any server within the data centre or in a different data centre. When users
connect to cloud databases through the Internet, they remain unaware of
the exact location of their stored data.  Users are guided to their
databases by \acp{API} of the cloud databases (Wu et al., 2010). 

Cloud databases have to be scalable across these servers so that data is
available to any user at any given point in time. Scalability in the context of cloud storage refers to the ability of dynamically incorporating changes to the number
of users or storage space, without affecting the functioning of the databases or
the availability of data to the users. In other words, when more machines
are added to increase storage capacity, or when more users access the same data,
cloud databases should cope with the increased workload and yet maintain the
same throughput. 

In general, cloud \acp{DBMS} are found to be less efficient than traditional
databases because of this dynamic scalability required by cloud databases to
support a changing user-base (Hogan, 2008). Hogan (2008) claims that data
partitioning in cloud databases increases complexity as a database is spread
across several servers and querying the database would involve complex Joins and
more time. This moves the databases and the user applications farther apart,
increasing latency (Murphy, 2010). Hence, data is split into distinct individual
parts and saved on different nodes in the data centre across several databases. Thus, nodes could have a
subset of data or rows from each table in the database (DeWitt et al.,
nod.).This eventually means that querying would take longer time as the data is
spread across several databases, possibly on different servers and would include
multiple joins on the datasets.

Most traditional databases are relational and gives data a structure, that
adheres to a schema. This is mainly trough the process of normalisation  where
table sare normalised to at-least \ac{1-NF}. This ensures that data is organised
and not redundant. Normalisation causes databases to have smaller and structured
tables by removing duplicate data from large and badly organised tables and by
imposing constraints on the data. This is as a result of normalisation, which is
the process of giving data a structure and organising the data. Throughout the
chapters normalization refers to making databases at least in \ac{1-NF}.

All these characteristics make cloud databases very
different from traditional \acp{DBMS} that are used outside cloud networks.
Unlike such traditional \acp{DBMS}, cloud \acp{DBMS} are simple in their structure with minimum querying support and have a
simple API for users for database administration. Cloud databases have been made
scalable to support the diverse and large number of users who store structured data and to support
various applications that users use. Today, cloud databases are replicated,
distributed, simplified and often specialised (Cooper, 2010). Cloud databases
are replicated so that multiple copies of data are available to cater to many
users who access the same data at the same time. This also helps in cases of
server crashes or network failures, as copies of the data are available. The
cloud database are distributed as data is replicated on several machines. Most
cloud databases are simplified for ease of use and specialised to address
certain cloud related problems. For example, some cloud databases are built to provide high 
scalability while others are built to store huge amounts of interconnected data.

\section{Cloud Data Models}\label{s:cloud-data-models}
Data models define the structure of a database and give the users information on
how a database can be used or implemented. On the cloud, different types of data
models exist. The selection of a data model for a cloud database depends
on the problem the cloud database is specialised to address or a feature it is
incorporating. Some of the current popular data models on the cloud are:

\begin{itemize}
\item Key Value data model 

\item Document data model 

\item Relational data model
\end{itemize}

Generally cloud \acp{DBMS} are non-relational and adopt the key-value data
model, to maintain the data replication, consistency and scalability that is a
part of cloud computing data storage (\todo{cite 440}). Key-value data model is the main
focus of this thesis and is explained further below. Although \acp{RDBMS} on the cloud are
not widely used, there exist some cloud capable \acp{RDBMS} like Amazon Relational
Data Service, Microsoft SQL Azure etc. Just like the traditional relational model, 
relational model on the cloud also supports relations or tables with rows and
columns to store structured data and adheres to a schema. These \acp{RDBMS} provide users with database
administration facilities and APIs to scale relational databases and to perform
operations on stored data, like updating, inserting, deleting data etc. The
cloud \acp{RDBMS} offered today vary according to the vendors and each of the vendors
propose alternative solutions to problems like scalability and latency. But, the
replication of data is restrained due to the relational nature of \acp{RDBMS} and this reduced
replication affects the scalability and performance as well. Most cloud
\ac{RDBMS} are outperformed by the key-value cloud databases i.e, \ac{NoSQL} \acp{DBMS}. 

The key value databases, document databases and other databases that support
non-relational data models, on the cloud are loosely termed as  \ac{NoSQL}
databases. \ac{NoSQL} \acp{DBMS} are considered the next generation cloud \acp{DBMS} that aim to provide non-relational distributed \acp{DBMS} with open-source content and
development for the cloud (\ac{NoSQL}, n.d.)



\section{Key Value Data Model}\label{s:key-value-data-model}
In basic terms, the key value data model represents data as a key-value tuple
consisting of a key, value and a timestamp. The value is the actual data that
has to be saved and it is associated with the key, which is used to retrieve the
value from the database. The value is commonly a string data type. This is
similar to the way data is stored in a map. A timestamp is a 64-bit integer and
it records the time at which the value was inserted or updated in any way.

Generally, key value data model on cloud adopts the column-oriented approach,
which is adopted from Google's  cloud database, Bigtable.
The building blocks of column-oriented key value databases are columns,
SuperColumns, ColumnFamily, SuperColumnFamily, KeySpace. Using an example of a
relational database named \texttt{University}, where students are enrolled in
different courses, this data model is explained below. The details of the students are
saved in a table called \texttt{Student} and the course details in the \texttt{Course} table.
The student-course relationship is maintained in a separate table called
\texttt{Enrolment} which has foreign keys for both \texttt{Student} and \texttt{Course} tables.

\begin{itemize}
\item Columns:  A column is the basic unit of a table and it contains data represented
as a key, value and a timestamp tuple.

A column in a key value databases is equivalent to a cell in an \ac{RDB}. A few
columns in a key value database are shown below,
where \Quote{SWEN301}, \Quote{Software Engineering} etc. would
each be a single column with a key and timestamp as seen in
Figure~\ref{f:column-in-kvdb}.

\begin{figure}[h]
	\centering
	\begin{tabular}{ c c c c }
		\toprule
		\textbf{Key} & \textbf{Value} & \textbf{Timestamp} \\
		\midrule
		CourseId & SWEN301 & 89765432\\		
		\bottomrule
	\end{tabular}
	\caption{Column in a Key-Value database}\label{f:column-in-kvdb}
\end{figure}

\begin{table}[h]
	\centering
	\begin{tabular}{ c c c c }
		\toprule
		\textbf{Key} & \textbf{Value} & \textbf{Timestamp} \\
		\midrule
		CourseName & Software Engineering & 98732587\\
		\bottomrule
	\end{tabular}
\end{table}

\item SuperColumns: A SuperColumn is a key-value pair, where key or row key, is
the unique identifier for each SuperColumn. A SuperColumn can be understood as a row
in a table in \acp{RDB}. Row keys are just arbitrary string values and can be
provided by the users in most key value databases. The value of the key-value
pair is a map of columns. Hence, the values in a SuperColumn are actually
key-value pairs. Unlike columns, SuperColumns do not have timestamps for its
key-value pairs.

\begin{table}[h]
	\centering
	\begin{tabular}{ c c c c c }
		\toprule
		\multirow{2}{*}{SuperColumnKey} & \multicolumn{4}{c}{Values}\\

		& CourseName & Trimester & Level & Year\\ 
		\midrule
		SWEN301 & Software Engineering & 3 & 3 & 2012\\
		\bottomrule
	\end{tabular}
\end{table}

\item ColumnFamily: Columns can be grouped together into a ColumnFamily which is
a structure with several rows containing key-value pairs. The key is the row key
and the value is a map of column names. This map of columns contains key value
pairs, of column keys and columns.

\begin{table}[h]
	\centering
	\begin{tabular}{ c c c c c }
		\toprule
		\multirow{2}{*}{SuperColumnKey} & \multicolumn{4}{c}{Values}\\

		& CourseName & Trimester & Level & Year\\ 
		\midrule
		SWEN301 & Software Engineering & 3 & 3 & 2012\\
		COMP301 & Computer Engineering & 2 & 3 & 2012\\
		COMP401 & Compiler Engineering & 1 & 4 & 2012\\
		\bottomrule
	\end{tabular}
\end{table}
A ColumnFamily is analogous to a table in an \ac{RDB}. Key value databases are
schema-less and thus no structure is enforced on the tables. Hence, the rows in
the ColumnFamily do not have any constraints with the number of columns it can
have (Sarkissian, 2009). This means that it is valid even if one of the rows in
the above example has more columns, for example, \texttt{SWEN301} could have an
additional column of \texttt{CRN} which no other rows have values for. Or
one row can have \texttt{100} columns and another row could have only \texttt{2}.

\item SuperColumnFamily- A SuperColumnFamily contains rows where each row of a
SuperColumnFamily contains a map of SuperColumns. In this map, the key is the
SuperColumn key and the value is the SuperColumn itself. For example, a row in a
SuperColumnFamily to show the enrolment of students would contain the
SuperColumns for the course and the students.

\begin{table}[h]
	\centering
	\begin{tabular}{ c c c c }
		\toprule
		\multirow{2}{*}{SuperColumnFamilyKey} & \multicolumn{3}{c}{Values}\\
		& StudentId & CourseId & Type \\ 
		\midrule
		1001 & 101 & SWEN301 & Student \\
		1002 & 101 & COMP401 & Student \\
		\bottomrule
	\end{tabular}
\end{table}

Here, the SuperColumns containing data about the student \texttt{101} who
takes the course \texttt{Software  Engineering} is given in a single row identified by
the key or \texttt{RowId} \texttt{1001}. If a student is enrolled in more than
one course, then the SuperColumnFamily row would contain another SuperColumn of that course too.

\item  KeySpace: All the ColumnFamilies are grouped into a KeySpace, which is in
general named by the user. A KeySpace can have multiple ColumnFamilies but no
relationship between such ColumnFamilies is maintained. A KeySpace can be
considered similar to a database in traditional relational databases, minus any
relationships.

\end{itemize}
The way in which key value is implemented in Cassandra, to match the
scalability, data availability and other requirements on cloud, is explained
later.


\section{Challenges in Key-Value model}\label{s:challenges-key-value}
Fundamentally, the key value data model is different from the relational model
in many ways. While relational data model aims at giving data a structure,
providing data integrity and referential integrity, key value databases just
store data as blobs or string values and generally do not maintain many
relationships between data. The key-value association and the ColumnFamily
grouping etc can be considered as the minimum relationship that is maintained
within cloud \ac{NoSQL} databases. Unlike the popular traditional relational
database systems, cloud \ac{NoSQL} database systems do not support any schema
and hence do not maintain data dependencies or relationships between data entities.

According to Bell and Brockhausen (1995), ``Data dependencies are the most
common types of semantic constraints in relational databases which determine the
database design''. Data dependencies are the various relationships that may
exist between data entities in a database. For example, in a \texttt{Student}
database, a student can enrol into more than one course and each course would have its own
course name, course ID etc. This means that there is a many-to-many relationship
between student and courses. In the case of a relational database, the student
would have its own table and the course would have a different table. The
relationship between any student and various courses the student takes would be
stored in a third table called \texttt{Enrolment}. Such a distribution of data
is present due to the schema specified and normalisation by the relational
database. In the above example, this means that rows in \texttt{Enrolment} table
would contain the \texttt{studentID} and the \texttt{courseID} (as foreign
keys), which shows the dependency or relationship between a student and course.
Any attempt to delete a course from the \texttt{Course} table, is prevented by a
constraint, unless the dependency itself is removed first. In relational
database systems, this constraint is referential integrity, which ensures that references between data entities are valid, consistent and intact
(Blaha, n.d.). Normalisation as well as modelling real world data (along with
the real relationships) enforce such dependencies in the schema and this causes
integrity constraints like referential integrity constraints, to be imposed on
data entities.

If such constraints are not imposed, there could arise many dangling
dependencies in the database. In the above example, consider the case of foreign
key references between \texttt{Course} and \texttt{Enrolment}. If a course, with
a foreign key in the \texttt{Enrolment} table, is deleted without removing its
dependencies in \texttt{Enrolment}, there could be many active references to the
deleted course in \texttt{Enrolment}. Similarly, a dangling reference could
occur during insertion of data. For example, a new student is entered in the
\texttt{Enrolment} table, with a \texttt{CourseId} (foreign key), that does not
exist in the \texttt{Course} table (i.e., wrong \texttt{CourseId}). A dangling
reference occurs because this inserted student refers to a nonexistent course.
Another common cause for dangling references can be due to improper commits
performed by the application. For example, consider a practical scenario of an
online student registration/enrolment system, where every student enters their
address and foreign key references exist between \texttt{Student} and
\texttt{Address} tables.
If, due to any reason, the commit into the \texttt{Student} table fails and the
commit to the \texttt{Address} table proceeds without any problem, this can
cause a dangling reference in the \texttt{Address} table. Such problems cause inconsistent data to be stored and
affect data integrity. To ensure that users get consistent and valid
information, applications would have to implement mechanisms to check or prevent
dangling references. But if referential integrity constraints are applied, like
in a relational database system, such deletions or insertion of data would not
be permitted and dangling references will not occur.

As previously mentioned, \ac{NoSQL} database systems are schema-less and thus do
not normalise data and nor are any relationships maintained. But relationships or dependencies between data
are common when real world data is stored in databases. For example, in real
world, a course could be taught by more than one lecturer or a student with an
Art major is restricted entry into Chemistry courses etc. These relationships
and constraints have to be preserved when such real world data is stored in
\ac{NoSQL} database systems too. In \ac{NoSQL} database systems, data is saved as a
document or a key-value pair etc, and data would be replicated across several
nodes in a data centre. This effectively means that the relationship of student
and course will not be strictly enforced and deleting a course in the \ac{NoSQL} is
allowed as a result of the absence of constraints. As mentioned above, this
could mean that a student could still be enrolled in a deleted course, since
there are no constraints to prevent such deletions or changes in \ac{NoSQL} database
systems.

Commonly, developers impose such constraints and reference checks on \ac{NoSQL} data
at the application side. Another way to implement such checks is to give these
constraints at the persistence layer of the application server. Persistence
layer is any software layer that makes it easier for a program to persist its
state (Wikipedia, 2011a), for example, Entity Java Beans, Java Persistence
architecture -JPA, Hibernate, DataNucleus in Google App Engine for Java, etc
(Google, 2011). Most persistence layers achieve persistence by using an
underlying \ac{DBMS}. Both these ways would eventually have to
handle all the processing and managing of these constraint checks for all the
widely spread data in \ac{NoSQL} databases. This could mean immense workload on the
application or the application server, especially if the data volume is large in
the \ac{NoSQL} database or if it is has many replicas that have to be checked as well
for the constraints.

This is a serious problem today, especially when data is interconnected and
dependant on other data entities in today's world. For example, Facebook and
many other social networking sites have to pull out large amounts of
interconnected data frequently and such data could naturally have many
dependencies on other data entities (Hoff, 2009). A simple example of this
problem is found in Facebook itself. Every user has a newsfeed page that shows
all the latest news about other users in their friend list. If one user posted a
message on Facebook, this would appear as news to all other users who are
friends with this user. Now, if this user deletes this wall post, sporadically
it has been observed that the deleted wall post would still appear on other
user's newsfeed page for some time, irrespective of page reloads. This shows how
deletions or changes in user data fail to be reflected everywhere in the
database system immediately. Eventual consistency, where every replica of the
data will see the update eventually after a time delay, also causes a delay in
having all the various instances of the data to get updated with the correct
data. But in spite of eventual consistency data dependency should be correctly
handled and recorded. Had this been in a relational database, the user would
have been prevented from deleting his wall post as other users have access to it
or data would be correctly updated so that users do not see the wall post even
after deletion.

Although such problems affect most \ac{NoSQL} database system users, the impact of
such a problem could be different for different users. For example, a banking
system using a \ac{NoSQL} database system could be gravely affected because of
dangling references or as seen in the Facebook example, user privacy could be
comprised, while in a simple game application such problems could be trivial.
Motivated by such problems of data dependencies, this research studies the
existing modelling of data dependencies in \ac{NoSQL} database systems and aims to
contribute by suggesting four solutions so that such dependencies are
effectively maintained, while also not limiting the benefits of not having a
schema in \ac{NoSQL} database systems. Such a result would eventually reduce the
workload of the applications or the persistence layers of the application
servers. This would give users of \ac{NoSQL} database systems better consistency in
data, along with assuring better data integrity even when data is widely
replicated or spread on different data-centers.



\section{Referential Integrity in Key-Value
Model}\label{s:referential-integrity}
Addressing the aformentioned challenge, would translate to having referential
integrity constraints introduced in \ac{NoSQL} database systems.

Referential integrity is a fundamental property of data within a database, which
ensures that certain data dependencies between tables are maintained correctly
in the database (\todo{cite oracle}). These dependencies could be a part of the
business rule and need to be enforced for proper data integrity. It is possible
to define conditions or rules on the database so that data integrity is ensured
at all times. These conditions are called integrity constraints and need to be
mandatorily satisfied at all times in order to ensure that users or applications
do not enter incorrect or inconsistent data into the databases.

The Referential Integrity Constraint is just one amongst other constraints, and
ensures that the value of one column of a table matches the values of a column
in another table. Generally in \acp{RDBMS}, a referential integrity constraint
identifies a foreign key in a table which has a direct relationship with the
primary key (or a unique key) of another table. The table with the referencing
key (foreign key) in it is the referencing table (a.k.a. child table), while the
table with the referenced key (primary or unique key) is the referenced table
(a.k.a.
parent table). For example, in the \texttt{University} database,
\texttt{Enrolment} is the referencing table while the \texttt{Student} and
\texttt{Course} tables are the referenced tables.


It has been defined by many researchers that referential integrity is enforced
by the combination of a primary (or unique) key and a foreign key, and that
every foreign key has to match the referenced key (\todo{cite}). In the
University example, every student foreign key in the \texttt{Enrolment} table
must match one of the referenced keys in the \texttt{Student} table. Similarly, every \texttt{Course} foreign
key in \texttt{Enrolment} must match a referenced key in the \texttt{Course} table. The
referential integrity constraint also states that the referenced key must exist
in the referenced table. If any foreign key refers to a non-existing primary
key, it violates the referential integrity constraint.  For example, if
\texttt{StudID100} is a student Foreign key in the \texttt{Enrolment} table, but
\texttt{StudID100} does not exist as a primary key in the \texttt{Student}
table, then it is a violation of referential integrity.

 
Referential integrity constraints also describe the data manipulation that is
allowed on the referenced values. Some of the widely associated rules are:

\texttt{Restrict} or \texttt{No delete}: Prevents any update or deletion of data
that has references.

\texttt{Set to NULL}: On updating or deleting the referenced key, all foreign
keys are set to NULL values.

\texttt{Set to Default}: On updating or deleting the referenced key, all foreign
keys are set to a default value.

\texttt{Cascade}: When the referenced data is updated or deleted, all the
associated dependent values are updated or deleted accordingly.

All existing \acp{DBMS} may not always support all of the above rules. Some \acp{DBMS} may
have the \texttt{Cascade} rule by default, while some may have the
\texttt{Restrict} rule by default. On the other hand some may not support
\texttt{Set to NULL} or may have other rules like the \texttt{No Action} rule,
which is similar to \texttt{Restrict} but performs checks only at the end of a
statement.

Referential integrity constraints have been a relational feature in traditional
\acp{RDBMS} and is imposed due to the way the \acp{RDBMS} enforce normalisation.
To improve the data dependency in \ac{NoSQL} database systems, this
thesis porposed referential integrity constraints using different
approaches. These are explained in Chapter (\todo{cite}).

\section{Summary}
	This Chapter presented the backgrounds about the underlying concepts in Cloud
	Computing and Cloud Databases. Cloud computing is gaining prevalence due to its
	many benefits as discussed in the chapter. With an increase in the number of
	users migrating to cloud computing, cloud data storage is gaining prominence as
	well, for easy and simple data storage. This has paved the way for the
	existence of many diffrent data models and databases on the cloud. Amongst the
	many data models, key-value data model has been the most widely used one on the
	cloud as it is more adapted to the cloud environment. This is mainly because
	of the data replication and 
	
	The next Chapter presents Cassandra as a Key-Value Cloud database as well as
	the solutions proposed to deal with referential integrity in cloud.





