% ब
\chapter{Background} \label{c:background}

This chapter  presents an overview of some of the significant topics relevant to
this thesis. Section~\ref{s:cloudComputing} describes cloud computing and the
various services it provides. Section~\ref{s:cloud-databases} presents cloud
databases as one of the key services provided in the cloud.
Section~\ref{s:cloud-data-models} describes the prevailing data models
prominently used by cloud databases. Section~\ref{s:key-value-data-model} gives a detailed
description of the column-oriented key-value data model, which is one of the
popular and widely used data models in cloud databases.
Section~\ref{s:challenges-key-value} presents some of the challenges existing in
this key-value model.
Section~\ref{s:referential-integrity} addresses one of these crucial challenges
of referential integrity in column-oriented key-value cloud databases.
Section\ref{s:Cassandra} introduces the architectural concepts of Cassandra, the
column-oriented key-value cloud \ac{DBMS} used in this thesis.

\section{Cloud Computing} \label{s:cloudComputing}
Cloud computing is a major paradigm that is rapidly shifting the way \ac{IT}
services and tools are being used in the industry.  It is perceived that cloud
computing would help extend the capabilities of many \ac{IT} and online services
without the need for costly infrastructure. 

Similar to remote computing where other machines or computers are accessed from
the local machine through a network,   cloud computing leverages network
connections to provide various services to the users.  It also brings with it the
virtualization of applications and services,   where it appears to users as if the
applications are running on the user's machine rather than a remote cloud
machine (Cloud Computing Defined,   2010).  This removes the need for installing
the actual software by the users.  Thus,   both expert and naive users need not
worry about the technical details and configurations to use these cloud
services. 

Cloud computing is generally based on a subscription model where users pay as
per their usage,   which is very similar to utility services like electricity,   gas
or water etc.  The coalescence of virtualization,   where applications are
separated from the infrastructure is what makes cloud computing easy to use. 
Users do not have to invest in software applications as they can access such
applications on the cloud.  Users pay only for the services they use.  For
example,   they pay only for the amount of storage their cloud database uses or
pay only for the bandwidth consumed by the servers they rent from the cloud
providers.  Applications and databases are stored in large server farms or data
centres owned by companies like Google,  IBM etc. 

The architecture of cloud computing services has users who avail cloud services
as the front-end.  The back end of the architecture includes the cloud servers,  
databases,   and computers etc. ,   which are abstracted from users.  All the
components like the servers,   applications,   the data storages work together
through a web service to provide the users with the cloud services. 

The overall structure of cloud computing and its various services have been
generalised into layers (ZakiSabbagh,   2010,   Bime,   2008). 

\begin{description}

	\item [User:] is any hardware or software application that relies on cloud
	computing to perform its work.  Generally,   'client' refers to any software applications or
	\acp{API}  that are used to perform cloud computing,  
	while 'users' represents the end-users,   like  database administrators or
	programmers or anyone who benefits from cloud computing services. 
	
	\item [\acf{SaaS}] is the service provided by the cloud
	providers where users do not have to install the software applications. 
	
	\item [\acf{PaaS}] is the service where a hardware or
	software platform is provided to users.  A platform could be an operating system,  
	programming environment,   hardware,   run-time libraries etc. 
	
	\item [\acf{IaaS}] is the service where users can use
	the expensive hardware like network equipments,   servers etc. 
	
	\item [\acf{DaaS}] is a cloud storage service  that represents
	the storage facilities,   like  \acp{DBMS} which are provided
	as cloud services for which users pay only for the storage space they use (Wu et
	al.,  2010). 

\end{description}


\ac{DaaS} involves hosting cloud databases in the cloud which offer data
management,   data retrieval,   and other database services.  Due to the
increasing number of users deploying and using web applications on cloud,  
cloud databases form a crucial part to store the increasing amounts of data
on the cloud.  Many companies like Amazon,   Google,   IBM,   and Microsoft
provide \ac{DaaS} and offer varying levels of services (Mateljan et al. ,  
2010). The next section gives a description about cloud databases and its key
features.
% More details about \ac{DaaS} and cloud \ac{DBMS} are provided in Literature
% Review.

% In the remainder of this chapter,   Section~\ref{s:cloud-databases} presents cloud
% databases.  Section~\ref{s:cloud-data-models} presents the prevailing cloud data
% models.  Section~\ref{s:key-value-data-model} presents the Key Value Data Model. 
% Section~\ref{s:challenges-key-value} presents the challenges in the Key Value
% Data model and Section~\ref{s:referential-integrity} discusses the Referential
% Integrity Constraint in Key Value data model,   which is the focus of this
% research. 


\section{Cloud Databases}\label{s:cloud-databases}

Most cloud applications store,   process and provide large amounts of data like
the user information,   application data or some stored data which maybe accessed
by the users.  Storage of such data during all times is essential for the cloud
applications to operate correctly(Kennedy,   2009). Traditionally,   users store data
in files or databases residing on dedicated database servers or on local disks,  
but the requirements for data storage on the cloud are very different and need a
distributed approach in data storage,   where data is spread across several
machines.  Cloud \acp{DBMS} require more features than the traditional \acp{DBMS}
for an efficient data management on the cloud (\todo{cite G-Store}).  Unlike
traditional \acp{DBMS},   cloud \acp{DBMS} are simple in their structure with
minimum querying support and have a simple API for database
administration.  These have been made scalable to support the diverse
and large number of users who store structured data and to support various
applications that users use.  Today,   cloud \acp{DBMS} are replicated,   distributed,  
simplified and often specialised (Cooper,   2010).  Databases on the cloud
(also known as cloud databases) are replicated so that multiple copies of data
are available to cater to many users who access the same data at the same time.  This also helps in cases of server crashes or
network failures,   as copies of the data are available.  These databases are
distributed as data is replicated on several machines.  Most cloud \acp{DBMS} are
simplified for ease of use and specialised to address certain cloud related
problems.  For example,   some cloud \acp{DBMS} are built to provide high
scalability while others are built to store huge amounts of interconnected data. 

Most cloud \acp{DBMS} are deployed in data centers owned by hosting companies
like Google,   Amazon etc.  Data centres house many servers,   computers and
telecommunication infrastructure,   including back up and security facilities and
users can rent or buy the storage space they need.  Within data centres,   data
is stored on remote machines,   which could be any server within the data
centre or in a different data centre.  When users connect to cloud databases
through the Internet,   they remain unaware of the exact location of their stored
data.   Users are guided to their databases by \acp{API} of the cloud \ac{DBMS}
(Wu et al. ,   2010). 

Cloud databases have to be scalable across these servers so that data is
available to any user at any given point in time.  Scalability in the context of
cloud storage refers to the ability of dynamically incorporating changes to the
number of users or storage space,   without affecting the functioning of the
databases or the availability of data to the users.  In other words,   when more
machines are added to increase storage capacity,   or when more users access the
same data,   cloud databases should cope with the increased workload and yet
maintain the same throughput. 

In general,   cloud \acp{DBMS} are found to be less efficient than traditional
\acp{DBMS} because of this dynamic scalability required to
support a changing user-base (Hogan,   2008).  Hogan (2008) claims that data
partitioning in cloud databases increases complexity as a database is spread
across several servers and querying the database would involve complex Joins and
more time.  This moves the databases and the user applications farther apart,  
increasing latency (Murphy,   2010).  Hence,   data is split into distinct individual
parts and saved on different nodes in the data centre across several databases. 
Thus,   nodes could have a subset of data or rows from each table in the database
(DeWitt et al. ,   n. d. ). This eventually means that querying would take longer time
as the data is spread across several databases,   possibly on different servers
and would include multiple joins on the datasets. 

Most traditional \acp{DBMS} are relational and give data a structure,   that
adheres to a schema.  This is mainly achieved through the process of
normalisation where each table in a database is evaluated according to its functional dependencies
and primary keys,   to reduce redundancy and minimise integrity anomalies
(\todo{cite Navathe}).  Normalisation causes databases to have smaller and
structured tables by removing duplicate data from large and badly organised
tables and by imposing constraints on the data.  Tables are normalised to
at-least \ac{1-NF} ensuring data is organised and less redundant. Redundancy is
reduced by bringing the database schema into at-least \ac{3-NF} (or \ac{BCNF}).
Throughout the chapters normalization refers to making databases at least in \ac{1-NF}.

Cloud \acp{DBMS} are mostly non relational and follow a different data model,  
which is explained in Section~\ref{s:cloud-data-models}.  Such Cloud \acp{DBMS}
are loosely termed as cloud \ac{NoSQL} \acp{DBMS}.  Unlike \acp{RDBMS},   cloud
\ac{NoSQL} \acp{DBMS} do not aim to be ACID-compliant. 
ACID stands for the properties Atomicity,   Consistency,   Isolation and Durability,  
which ensure the completeness and reliability of a database operation.  In
general,   unless operations are not ACID compliant in \acp{RDBMS},   it would not
be considered valid. 
But ACID compatibility in cloud \ac{NoSQL} \acp{DBMS} is a bottleneck as it does
not suit the distributed nature of the cloud environment (Wada et al.  2011). 
Cloud \ac{NoSQL} \acp{DBMS} require to have high throughput,   high availability
and also require to be elastically scalable to increasing resources or users. 
This requires cloud \ac{NoSQL} \acp{DBMS} to part with some traditional
\ac{RDBMS} functionalities (like JOINS) and ACID operations (Wada et al.  2011)
mainly because of its distributed nature.  The distributed nature,   across
different environments,   sometimes make cloud \acp{NoSQL} \acp{DBMS} to be prone
to node failures.  Node failures and the elasticity prevailing in cloud
environments affect consistency of data,   which adversely affect the '\texttt{C}'
of ACID properties,   i. e. ,   Consistency.  Network and data partitioning also play a
major role in affecting consistency and availability of data.   A partition takes
place when a node fails or there is a network failure at some point in the
network.  In cloud \ac{NoSQL} \acp{DBMS} this is a problem as cloud databases
rely on more than one server,   unlike traditional \acp{DBMS} sometime ago.  When
there is only a single server involved there is no issue of partitioning,   but
when there is a failure,   i. e.  this single node fails,   no data is available
during that time.  But cloud \ac{NoSQL} \acp{DBMS} distribute data across several
servers and nodes,   mainly to replicate data for high availability and fault
tolerance. 
These \acp{DBMS} also aim for partition tolerance,   which is the ability to
continue their operations despite node failures and partitions.  To achieve these
features it is commonly found that cloud \ac{NoSQL} \acp{DBMS} sacrifice data
consistency.  Commonly,   most web applications aim to have their data available at
anytime,   as many users could access the data at the same time and in a business
model,   applications lose valuable customers if they are not kept satisfied with
the services in terms of speed,   availability and consistency.  Instead of ACID
properties,   cloud \ac{NoSQL} \acp{DBMS} aim to achieve different
properties,   namely Consistency,   Availability and Partition-tolerance (CAP)
properties stated in the CAP theorem. 
These properties and the CAP theorem are explained below. 

\subsection{CAP Theorem} \label{ss:cap}
In distributed environments or web-based applications it is noted that the
three main system requirements necessary for designing and deploying
applications are: Consistency,   Availability and Partition tolerance.  CAP
theorem, proposed by Eric Brewer, claims that it is not possible for a
distributed system to achieve all these three properties at the same given time
(\todo{cite}).

\begin{itemize}
  
	  \item Consistency: When a request is made to access data,   a system is called
	  consistent if it provides the correct and latest version of the data (\todo
	  {cite HP n. d. }).  For example,   in the case of an online shopping website,
	  consistency would ensure that the stock of items is always correct.  When a
	  user attempts to buy an item and the same item is being bought by another
	  user,   the system will have to ensure that both the users get the most recent
	  stock details available.  So if there is only a single item left,   then the
	second user is informed that his request can not be completed as there is no
	stock available.  This means that the data is consistent and users do not get
	stale data.
	
	  \item Availability (\todo{cite Browne 2009; HP n. d. }): A system is
	  considered highly available when all parts of the system are always available,
	    despite any failures or problems.  It is expected that all requests would be
	  addressed any given point of time.  In the previous example,   this means that
	  even when the
	website is busy,   with many users accessing it,   it is expected to have all
	user requests addressed and to be up and running always.
	
	\item Partition-tolerance: Generally, distributed services are run on several
	machines across different networks and these services are prone to network
	partitions. Network partitions happen when there is a failure of a segment or
	component of a network such that nodes cannot communicate with each other.
	%As previously mentioned, this means that data is
% 	partitioned across the distributed network on several machines and a partition
% 	takes place when there is a failure in the network.  
	A system is considered  partition-tolerant when despite such partitions,   it
	continues to provide its services and address user requests.
\end{itemize}

The CAP theorem states that at a given point of time,   only two of these
properties can be achieved or satisfied by any application.  This means that an
application that is distributed like cloud \ac{NoSQL} \acp{DBMS} has to make
trade-offs on one of the properties always.  Trade-offs are common in the real world,   where
some features are sacrificed for other features,   that may suit the
business or operation model better.  In most distributed applications trade-offs are always
considered from the design stages.   Similarly most cloud \ac{NoSQL} \acp{DBMS} have chosen
priorities and trade-offs too.  For example,   Cassandra focuses on '\texttt{A}'
and '\texttt{P}' while Bigtable focuses on '\texttt{C}' and '\texttt{A}'
(\todo{cite}). 

What such trade-offs mean in relation to the CAP theorem is examined below
(\todo{cite HP n. d. }):

\begin{description}
	  \item [Case 1: Achieving '\texttt{C}' and '\texttt{A}' properties:] This
	  means that when an application aims to achieve consistency and availability,  
	  it will be less partition tolerant. 
	When data is partitioned,   there is more time involved in accessing the data from
	the various points in the distributed network.  Moreover,   failures mean more time
	delays.  So to achieve high consistency and availability it is required
	that the application depends on fewer nodes.  Having all data stored on a single
	machine means there is no partition of data and this data will be always
	consistent and available as long as this machine is up and running.   On the
	other hand,   when systems that are not partition tolerant face a partition,  
	it becomes unreliable.  It could either give inconsistent data or become
	unavailable or both (HP n. d. ).
	
	\item [Case 2: Achieving '\texttt{A}' and '\texttt{P}' properties:]
	Commonly,   this is what most cloud \ac{NoSQL} \acp{DBMS} aim to achieve and these
	prefer to pay less attention to consistency of data (Wada et al.  2011).  A system
	lacking consistency is thus mostly available even during partitions,   but
	may give stale data or incorrect data occasionally to the users.  This suits
	most business models as this ensures that users are always able to
	access their data.  For example,   in the previous online shopping example,   when
	data is not available or the request fails,   users could get anxious whether
	they lost their money during the transaction.  To avoid such cases,   users
	are presented with data as soon as possible,   despite being stale,   since users
	will be able to see the data rather than being left unsure. 
	Since a cloud \ac{NoSQL} \ac{DBMS} that has poor consistency is unrealistic,  
	these \acp{DBMS} tend to provide eventual consistency
	
	\item [Case 3: Achieving '\texttt{C}' and '\texttt{P}' properties:] This means
	that while a system is consistent and tolerant to partitions or
	failures,   it may not always be available and running.  Such a system 
	provides correct data while tolerating network failures but may not be
	accessible during failures preventing operations to be performed on data.  This
	leads to a less reliable system,   where data is correct but unavailable and
	inaccessible during network failures. 
\end{description}

Interestingly,   while cloud \ac{NoSQL} \acp{DBMS} do not comply with ACID
properties,   the CAP system has lead to a new set of properties called BASE and
is considered as an alternative of ACID properties in distributed and scalable
systems(Pritchett 2008). 
BASE refers to the properties Basically Available,   Soft-state and Eventually
consistent.  This means that data is basically available although at some point
not all data will be available.  Soft-state indicates that data could be lost if
not properly maintained,   i. e. ,   data has to be refreshed and version-checked for
it to remain saved.  Eventually consistent,   as mentioned previously,   is a weak
form of consistency where in a cluster of nodes,   every node would get the
updates eventually at some point in time. 
BASE could be understood as being closer to  \texttt{Case 2} mentioned above,  
where consistency takes a back seat.  But this leads to conflicts where a new update or a new read
request could be made before all nodes get the latest update.  To resolve such
conflicts there are some types of repairs used by cloud \ac{NoSQL} \acp{DBMS},  
like read-repairs,   write-repairs and asynchronous repairs (Terry et al.  1995). 
When a read or write operation takes place,   such repairs check for inconsistency
in data before correctly updating the data.  Some cloud \ac{NoSQL} \acp{DBMS} also rely
on APIs to work around such issues.  It is often considered that with good
design,   \acp{API} can work around these problems and provide better consistency. 
% Unlike such traditional \acp{DBMS},   cloud \acp{DBMS} are simple in their
% structure with minimum querying support and have a simple API for users for
% database administration.  Cloud databases have been made scalable to support
% the diverse and large number of users who store structured data and to support
% various applications that users use.  Today,   cloud databases are replicated,  
% distributed,   simplified and often specialised (Cooper,   2010).  Cloud databases
% are replicated so that multiple copies of data are available to cater to many
% users who access the same data at the same time.  This also helps in cases of
% server crashes or network failures,   as copies of the data are available.  The
% cloud database are distributed as data is replicated on several machines.  Most
% cloud databases are simplified for ease of use and specialised to address
% certain cloud related problems.  For example,   some cloud databases are built to
% provide high scalability while others are built to store huge amounts of
% interconnected data. 

All these characteristics make cloud \ac{NoSQL} \acp{DBMS} very different from traditional
\acp{DBMS} that are used outside cloud networks.  As mentioned previously,  
the underlying data model of the cloud
\ac{NoSQL} \acp{DBMS} is fundamentally different from the relational data model
of \acp{RDBMS} and this is explored in the following sections. 

\section{Cloud Data Models}\label{s:cloud-data-models}
Data models describe the structure of a database and give the users information
on how a database can be used or implemented.  On the cloud,   different types of
data models exist.  The selection of a data model for a cloud database depends on
the problem the cloud database is specialised to address or a feature it is
incorporating.  Some of the current popular data models on the cloud are:

\begin{itemize}
\item Key Value data model 

\item Document data model 

\item Relational data model
\end{itemize}

In general,   cloud \acp{DBMS} are non-relational and most cloud \acp{DBMS}
adopt the key-value data model to maintain the data replication,   consistency and
scalability that are part of cloud data storage (\todo{cite 440}). The key-value
databases,   document databases and other databases that support non-relational data models
on the cloud are loosely termed as \ac{NoSQL} databases.  \ac{NoSQL} \acp{DBMS}
are considered the next generation cloud \acp{DBMS} that aim to provide
non-relational distributed \acp{DBMS} with open-source content and development
for the cloud (\ac{NoSQL},   n. d. ) Many \ac{NoSQL} \acp{DBMS},   that are inherently
key-value \acp{DBMS},   have evolved by adopting various features from other
popular cloud \ac{NoSQL} \acp{DBMS}.  For example,   Cassandra adopts the column
oriented data model of Google's Bigtable(\todo{cite }) while Riak (\todo{cite 440}) is influenced by
Amazon's Dynamo (\todo{cite 440}).  This thesis focuses on the 
column-oriented key-value data model and is explained in
Section~\ref{s:key-value-data-model}. 

Although \acp{RDBMS} on the cloud are not widely used,   there exist some cloud
capable \acp{RDBMS} like Amazon Relational Data Service,   Microsoft SQL Azure
etc.  Just like the traditional relational model,   relational model on the cloud
also supports relations or tables with rows and columns to store structured data
and adheres to a schema.  These \acp{RDBMS} provide users with database
administration facilities and APIs to scale relational databases and to perform
operations on stored data,   like updating,   inserting,   deleting data etc.  The
cloud \acp{RDBMS} offered today vary according to the vendors and each of the vendors
propose alternative solutions to problems like scalability and latency.  But,   the
replication of data is restrained due to the relational nature of \acp{RDBMS} and this reduced
replication affects the scalability and performance as well.  Most cloud
\ac{RDBMS} are outperformed by \ac{NoSQL} \acp{DBMS} (\todo{cite}). 

\section{Key Value Data Model}\label{s:key-value-data-model}
In basic terms,   the key-value data model represents data as a key-value tuple
consisting of a key,   a value and a timestamp.  A key is a unique string commonly
encoded as UTF-8.  A value is the actual data that has to be saved and it is
associated with a key that is used to retrieve the value from a key-value
database.  The value is commonly of the string data type. 
This is similar to the way data is stored in a map.  A timestamp is a 64-bit
integer that records the time at which the value was inserted or updated in
any way. 

Generally,   the key-value data model on cloud implements the column-oriented
approach,   which is adopted from Bigtable,   Google's cloud \acp{DBMS} (\todo{cite
}).  The data model explored in this section is the column-oriented key-value
data model adopted by Cassandra.  This type of data model is fundamentally
different from the relational data model.  It sacrifices ACID properties as well
as normalisation in order to achieve high scalability,   fault tolerance,   data
partitioning among others.  To understand this new type of data model and cloud
\acp{DBMS} that adopt this model,   comparisons are drawn to \acp{RDBMS} that
adopt the relational model.  For this purpose a simple
example of a University database is used throughout the chapters,   where it
is assumed that students enrol into different courses.  This example is
illustrated below. 

When the University database is saved in an \ac{RDBMS},   a schema will be
applied. This example assumes that the details of the students are saved in a
table called \texttt{Student} and the course details in the \texttt{Course}
table. The Student-Course relationship is maintained in a separate table called
\texttt{Enrolment} which has foreign keys for both \texttt{Student} and
\texttt{Course} tables.  This can be seen in Figure~\ref{f:RDB}. 


\begin{figure}[h]
	\centering
	%\includegraphics[width=5cm,   height=5cm]{. /figure/random. jpg}
	\includegraphics[width=.8\textwidth]{./figure/Example/Relational-DB.png}
	\caption{University example as a Relational database}\label{f:RDB}
\end{figure}

This shows how the University database example is deployed as a
\ac{RDB}.  When data in the University example is modelled using the
column-oriented key-value data model,   the way it is stored is different. 
Although key-value \acp{DBMS} are schema-less,   column-oriented key-value
\acp{DBMS} are not entirely schema-less and hold fewer information about the
constraints within the databases as metadata ,   as seen in Cassandra
(\todo{cite DataStax}).  This data model allows
applications to model the way data is organised in a traditional RDBMS whilst
bringing more flexibility by denormalising data and imposing no rigid structures
or schema requirements (\todo{cite DataStax}).  Therefore,   it allows
applications to add data in the way they want and change their schema (if
needed),   without adhering to a rigid
schema unlike the traditional \acp{RDBMS}.

The building blocks of column-oriented key-value databases are the columns,  
the Super Columns,   the Column Family and the Key Space.  Using the
University example,   these terminologies are explained below. 
Appropriate analogies are drawn with the \ac{RDB} University,   as
seen in Figure~\ref{f:RDB},   to better understand these column-oriented key-value
concepts.  Since the focus is on Cassandra's data model,   these concepts
are explained in the way Cassandra deploys them.  The example used
to describe the Cassandra data model adopts a simple and flexible schema that
allows some structure in the way data is stored. 

\begin{description}
\item[Columns:]  A column is the basic unit of data in this data model.  It is a
tuple containing a column name,   a value and a timestamp (Figure~\ref{f:column}). 

\begin{figure}[h]
	\centering
	%\includegraphics[width=5cm,   height=5cm]{. /figure/random. jpg}
	\includegraphics[width=.4\textwidth]{./figure/Example/Column.png}
	\caption{Random Pic}\label{f:column}
\end{figure}

The column names are labels  and it is mandatory that a column has a
name.  Column names and values are stored as Bytes Type,   Long Type,  
Ascii Type,   binary values Lexical UUID Type,   Time UUID Type or as UTF8
serialized strings (\todo{cite }).  Timstamps are used to store the time of the
latest update made to the column and are thus used for conflict resolutions.  The
timestamp values are commonly stored as microseconds,   but could be in any format
that the application chooses.  However,   timestamp formats have to be consistent
across the database so that is the same format across all columns. 

Cassandra allows indexes to be created on column names.  These are called
Secondary indexes and are of type \texttt{Keys} in Cassandra.  When such secondary indexes
are used,   efficient queries can be specified using equality predicates,   and
can be made on ranges of columns too.  The latter ones are called range queries. 

A column name can be considered analogous to a column
in a table in any traditional \ac{RDBMS}.  To illustrate this analogy,
Figures~\ref{f:column-FirstName} and~\ref{f:RDB-User} show the differences
between the representation of values in \texttt{Student} in Cassandra and in an
\ac{RDBMS}.
It can be seen from these figures that a column in the column-oriented key-value
data model is similar to a single value in a row of a relational table.  For
example,   the data '\texttt{John}' in the relational table \texttt{Student} can
be considered equivalent to a single column in Cassandra.

\begin{figure}[H]
	\newcommand{\W}{.4\textwidth}
	\centering
	%\includegraphics[width=5cm,   height=5cm]{. /figure/random. jpg}
	\includegraphics[width=\W]{./figure/Example/Column_FirstName.png}
	\includegraphics[width=\W]{./figure/Example/Column_LastName.png}
	\includegraphics[width=\W]{./figure/Example/Column_Email.png}
	\includegraphics[width=\W]{./figure/Example/Column_Age.png}
	\caption{Columns in Cassandra}\label{f:column-FirstName}
\end{figure}

\begin{figure}[h]
	\centering
	%\includegraphics[width=5cm,   height=5cm]{. /figure/random. jpg}
	\includegraphics[width=.8\textwidth]{./figure/Example/RelationalTable_User.png}
	\caption{Relational Table - Student}\label{f:RDB-User}
\end{figure}

The JSON notation for  columns in Cassandra is shown in Figure~\ref{f:column-JSON}. 

\begin{figure}[H]
	\centering
	%\includegraphics[width=5cm,   height=5cm]{. /figure/random. jpg}
	\includegraphics[width=.4\textwidth]{./figure/Example/Column_JSON.png}
	\caption{JSON notation for a column}\label{f:column-JSON}
\end{figure}
% 
% Alternatively,   applications can also use column names to store values.  This is
% possible since it is not required that columns always have values and
% since column names are byte arrays,   applications can store any kind of
% values in it. 

\item [SuperColumns:] A super column is a different kind of a column where the
values are an array of regular columns (Figure~\ref{f:supercolumn}).  It consists of a super
column name and an ordered map of columns.  The columns within the values of a
super column are grouped together using a common look-up value,   which is
commonly referred to as the \texttt{RowKey}.  In other words,   a super column is a
nested key-value pair of columns.  The outer key-value pair forms the super column while the inner
nested key-value pairs are the columns.  Unlike regular columns,   super columns do
not have timestamps for its key-value pairs.  

\begin{figure}[H]
	\centering
	%\includegraphics[width=5cm,   height=5cm]{. /figure/random. jpg}
	\includegraphics[width=.8\textwidth]{./figure/Example/SuperColumn.png}
	\caption{A Super Column }\label{f:supercolumn}
\end{figure}

A super column can be considered roughly similar to a whole record in a
relational table in an \ac{RDB}. For example,   the super column for a
student,   as seen in Figure~\ref{f:supercolumn-John},   is analogous to a single
record in the relational table \texttt{Student} (Figure~\ref{f:RDB-User}). 

\begin{figure}[H]
	\centering
	%\includegraphics[width=5cm,   height=5cm]{. /figure/random. jpg}
	\includegraphics[width=.8\textwidth]{./figure/Example/SuperColumn_John.png}
	\caption{A Super Column for Student '\texttt{John}' in
	Cassandra}\label{f:supercolumn-John}
\end{figure}

The JSON notation for a super column is shown in Figure~\ref{f:supercolumn-JSON}. 

\begin{figure}[H]
	\centering
	%\includegraphics[width=5cm,   height=5cm]{. /figure/random. jpg}
	\includegraphics[width=.7\textwidth]{./figure/Example/JSON_SuperColumn_John.png}
	\caption{JSON notation for a super column}\label{f:supercolumn-JSON}
\end{figure}

\item [ColumnFamily:] A column family contains columns or super columns that are
grouped together using a unique row key.  It is a set of key-value
pairs,   where the key is the row key and the value is a map of column names
(Figure~\ref{f:columnfamily}).  The row key groups the columns together,   just as
in super columns. 

\begin{figure}[H]
	\centering
	%\includegraphics[width=5cm,   height=5cm]{. /figure/random. jpg}
	\includegraphics[width=.8\textwidth]{./figure/Example/ColumnFamily.png}
	\caption{Column Family in Cassandra}\label{f:columnfamily}
\end{figure}

Applications can define column families and metadata about the columns. 
It is commonly practised to have columns that are related or accessed
together to be grouped in the same column family.  Column families require that
some attributes are always defined,   like name,   column type and others.  It
also has optional attributes that can be defined if the application requires so.
 Some of the optional attributes are number of keys cached,   comments,   read
repairs,   column metadata among others.

Column families can have rows %to have relatively a definite number of columns. 
that are identified by their unique row keys.  This is similar to a table,   as
seen for table \texttt{Student} in Figure~\ref{f:RDB-User},   where every row in
the table has the same number of columns and primary keys are used to identify a
row.  An example of a column family is shown in Figure~\ref{f:columnfamilyUSER}.
Unlike relational tables in an \ac{RDB},   column families do not require all
the rows to define the same number of columns (\todo{cite Sarkissian}).

\begin{figure}[H]
	\centering
	%\includegraphics[width=5cm,   height=5cm]{. /figure/random. jpg}
	\includegraphics[width=.8\textwidth]{./figure/Example/ColumnFamily-User-DiffColumns.png}
	\caption{Column Family \texttt{User} in Cassandra}\label{f:columnfamilyUSER}
\end{figure}

The JSON notation for a single row of a column family in Cassandra is
shown in Figure~\ref{f:columnfamilyJSON} 

\begin{figure}[H]
	\centering
	%\includegraphics[width=5cm,   height=5cm]{. /figure/random. jpg}
	\includegraphics[width=.8\textwidth]{./figure/Example/JSON_ColumnFamily_1row.png}
	\caption{JSON notation for a column family in
	Cassandra}\label{f:columnfamilyJSON}
\end{figure}

\item [KeySpace:] A keyspace is a container to hold the data that the
application uses.  Keyspaces have one or more column families,   although it is not strictly
required that a keyspace should always have column families.  Any relationships
existing between column families in a keyspace are not preserved. 

A keyspace can be considered similar to a database in traditional relational
databases,   without any relationships.  An example of the keyspace
University is shown in Figure~\ref{f:keyspace}. 

Keyspaces require that some attributes are defined,   like a user defined name,  
replication strategy and others.  Some optional elements that can be defined are
the details of the column families in the keyspace and other options
for replication of data. 
\end{description}

%\newpage

\begin{figure}[H]
	\centering
	%\includegraphics[width=5cm,   height=5cm]{. /figure/random. jpg}
	\includegraphics[width=.9\textwidth]{./figure/Example/KEYSPACE.png}
	\caption{A keyspace in
	Cassandra}\label{f:keyspace}
\end{figure}

As previously mentioned,  cloud \ac{NoSQL}
\acp{DBMS} are generally specialised to address specific problems like
partition-tolerance,  high availability among others and for this some
trade-off are made when these are developed.
Some of the challenges and problems present in such \acp{DBMS} are discussed in the following section.

\section{Challenges in Key-Value model}\label{s:challenges-key-value}
Fundamentally,   the key-value data model is different from the relational model
in many ways.  While the relational data model aims at giving data a structure,  
providing data integrity,   the key-value data model just
store data as \acp{blob} or string values and generally do not maintain
relationships between data.  In the column-oriented key-value model,   the
key-value association and the grouping of columns in column families can be
considered as the minimum relationship that is maintained.  

According to Bell and Brockhausen (1995),   data dependencies are the most
common types of semantic constraints in relational databases and these determine
the database design.  Data dependencies are the various relationships that may
exist between data entities in a database.  For example,   in the
University database,   a student can enrol into more than one course and this
means that there is a many-to-many relationship between \texttt{Student} and
\texttt{Course} since   one course can have many students enrolled in it.  

As seen in Section~\ref{s:key-value-data-model},   the \texttt{Enrolment} table 
contains the \texttt{StudentID} and the \texttt{CourseID} as foreign keys,  
thus showing the dependency or relationship between students and courses
(Figure~\ref{f:RDB}). 
In the \texttt{University} \ac{RDB} any attempt to delete a course
from the \texttt{Course} table,   is prevented by a constraint,   unless the
dependency itself is removed first.  In \acp{RDBMS} ,   this constraint is referential
integrity,   which ensures that references between data entities are valid,  
consistent and intact (\todo(cite Blaha,   n. d. )). 
Normalisation,   as well as modelling real world data and
relationships enforce such dependencies in the schema and this causes integrity
constraints like referential integrity constraints,   to be imposed on data
entities. 

If such constraints are not imposed,   there could arise many dangling
dependencies in the database.  For instance,   consider the case of foreign key
references between \texttt{Course} and \texttt{Enrolment} in the  University
database.  If a course is deleted from the \texttt{Course} table without removing
its dependencies in \texttt{Enrolment},   the latter would contain active references
to the deleted course.  Another example of a dangling reference occurs during
insertion of data,   where a new student is entered in the
\texttt{Enrolment} table,   with a \texttt{CourseID},   that does not
exist in the \texttt{Course} table (i. e. ,   wrong \texttt{CourseID}).  A dangling
reference occurs because this inserted student refers to a nonexistent course. 
Such problems cause inconsistent data to be stored in databases and violate data
integrity.  To ensure that users get consistent and valid information,  
applications would have to implement mechanisms to check or prevent dangling references.  But if
referential integrity constraints are applied as in an \ac{RDB},   operations
on data that  adversely affects referential integrity   would not be
permitted.

As previously mentioned,   \ac{NoSQL} database systems do not normalise data and
nor are any relationships maintained.  But relationships or dependencies
between data are common when real world data is stored in databases.  For example,   in the
real world,   a course could be taught by more than one lecturer or a student with
an Art major is restricted entry into Chemistry courses etc.  These relationships
and constraints have to be preserved upon storage in
cloud \ac{NoSQL} database systems too.  As mentioned in
Section~\ref{s:cloud-databases},   cloud databases,   whether relational or
\ac{NoSQL},   have to replicate data across several machines and need to be
scalable to match the needs of the users.  The replicated and distributed nature
makes maintaining data dependencies complex and unfeasible in terms of speed and
efficiency.  In cloud \ac{NoSQL} databases,   this effectively means that the
relationship between \texttt{Enrolment}, \texttt{Student} and \texttt{Course}
tables will not be strictly enforced and deleting a course in cloud \ac{NoSQL} databases is allowed because
of the absence of constraints.  As mentioned before,   this means that students
could still be enrolled in deleted courses,   since there are no constraints to
prevent such deletions or changes in cloud \ac{NoSQL} databases. 

Commonly,   developers impose such constraints and reference checks on \ac{NoSQL}
data at the application side.  Another way to implement such checks is to give
these constraints at the persistence layer of the application server.  Both these
ways would eventually have to handle all the processing and managing of these
constraint checks for all the widely spread data in \ac{NoSQL} databases. 
However,   this could mean immense workload on the application or the application
server,   especially if the data volume is large in the \ac{NoSQL} database or if it is
has many replicas that have to be checked as well for the constraints. 


This is a serious problem when data is interconnected and dependant on other
data entities as is commonly the case.  For example,   consider a banking
application that uses cloud \ac{NoSQL} \acp{DBMS} where its data is spread
across several nodes and is interconnected.  Any debit or credit
transactions made to a users account will have to be replicated across all the
nodes and correctly persisted.  Many constraints will exist for transfer of funds
between user accounts and such constraints need to be validated correctly.  If a
user has multiple accounts,   the relationship between the accounts have to be
maintained too.  When such constraints are not validated correctly,   it will lead
to incorrect account balances and wrong updates in the user accounts.  On the
other hand,   when such applications use an \ac{RDBMS},   referential integrity
constraints would be imposed to maintain the relationships between the
accounts and such constraints would be defined while tables are created and
their validation would be triggered whenever any operations are performed on the
data. 

Updates may not be correctly reflected across all the nodes of the database due
to  eventual consistency too,   which is discussed in Section~\ref{s:Cassandra}. 
However, in spite of eventual consistency,   data dependencies should be correctly handled and
recorded. 

Although such problems mostly affect most cloud \ac{NoSQL} \ac{DBMS} users,   it
could be different for different users.  For example,   a banking system as
mentioned above could be gravely affected because of dangling references while
in a simple game application such problems could be trivial. 

% Motivated by such problems of data dependencies,   this thesis studies the
% existing modelling of data dependencies in cloud \ac{NoSQL} database systems and
% aims to contribute by suggesting four solutions so that referential integrity
% is effectively maintained,   while also not limiting the benefits of not having
% a rigid schema in \ac{NoSQL} database systems.  Such a result would reduce the
% workload of the applications or the persistence layers of the application
% servers.  Additionally it would give users of \ac{NoSQL} database systems
% better consistency in data,   along with ensuring better data integrity even when
% it is widely replicated or spread on different data-centers. 



\section{Referential Integrity in Key-Value
Model}\label{s:referential-integrity}
Addressing the aforementioned challenges,   implies to introduce referential
integrity constraints in cloud \ac{NoSQL} database systems. 

Referential integrity is a fundamental property of data within databases,   which
ensures that data dependencies between tables are maintained correctly in the
database (\todo{cite oracle}).  These dependencies could be a part of the
business rule and need to be enforced for proper data integrity.  Users define
conditions or rules on the tables in a database so that data integrity is
ensured at all times.  These conditions are called integrity constraints and need
to be mandatorily satisfied at all times in order to ensure that users or
applications do not enter incorrect or inconsistent data into the databases. 

The Referential Integrity Constraint is just one amongst other constraints,   and
generally in \acp{RDBMS},   these constraints ensure that the value of foreign
keys in a table matches the values of primary keys in another table.  The table
containing the foreign key is the referencing table (or child table),  
while the table with the primary or unique key is the referenced table (or parent table). 
For example,   in the University database,   \texttt{Enrolment} is the
referencing table while \texttt{Student} and \texttt{Course} are the
referenced tables.  Foreign keys are also known as the referencing key and
the primary keys as the referenced keys. 

It has been defined by many researchers that referential integrity is enforced
by the combination of a primary (or unique) key and a foreign key,   and that
every foreign key has to match the primary key (\todo{cite}).  In the
University example,   every foreign key in the \texttt{Enrolment} table must match
one of the primary keys in the \texttt{Student} and \texttt{Course} tables. 
Hence,   if any foreign key refers to a non-existing primary key,   the
referential integrity constraint is violated.   For example,   if
'\texttt{StudID100}' is a foreign key for a student in the \texttt{Enrolment}
table,   but '\texttt{StudID100}' does not exist as a primary key in the \texttt{Student}
table,   then it is a violation of referential integrity. 
 
Referential integrity constraints also describe the data manipulation that is
allowed on the referenced values.  Some of the widely associated rules are:

\begin{itemize}
  \item \texttt{Restrict} or \texttt{No delete}: which prevents any update or
  deletion of data that has references. 
\item \texttt{Set to NULL}: which sets all foreign keys to NULL values,   on
updating or deleting the referenced key. 
\item \texttt{Set to Default}: which sets all the foreign
keys to a default value,   on updating or deleting the referenced key. 
\item \texttt{Cascade}: which updates or deletes all the
associated dependant values accordingly,   when the referenced data is updated or
deleted. 
\item \texttt{No Action}: which performs checks only at the end of a
statement and is similar to \texttt{Restrict}
\end{itemize}

Existing \acp{DBMS} may not always support all of the above rules.  Some \acp{DBMS} may
have the \texttt{Cascade} rule by default like Oracle,   while some may have the
\texttt{Restrict} rule by default.  

Generally,   in \acp{RDBMS} the database manager enforces a set of rules to
prevent nay data operation,   like insert,   update or delete,   to change data in
such a way that referential integrity is not violated as seen in
Figure~\ref{f:RI}. 

\begin{figure}[H]
	\centering
	%\includegraphics[width=5cm,   height=5cm]{. /figure/random. jpg}
	\includegraphics[width=.8\textwidth]{./figure/Example/RI-Figure.png}
	\caption{Referential Integrity Rules}\label{f:RI}
\end{figure}

These rules are explained below:

\begin{itemize}
  \item Insert rule: An insert operation triggers a referential integrity
validation when data is being inserted into a referencing table,   i. e. ,   the
child table.  In such an event,   prior to entering the values in the referencing
table,   it is checked if the foreign keys exist in the referenced table.  For
example,   in the University \ac{RDB},   when a row is inserted in the
\texttt{Enrolment} table with foreign key values for \texttt{StudentID} and
\texttt{CourseID},   a check is triggered to verify whether these foreign keys
exist in the \texttt{Course} and \texttt{Student} tables as primary keys.  If
the foreign keys do not exist in the referenced tables,   then the insert operation is not allowed. 

\item Update rule: When data is updated either in the referencing table or
the referenced table,   a referential integrity validation is needed.  When any
primary key is updated in the referenced table,  
then it is verified whether this key is a foreign key in any of the
referencing tables.  If a dependency is found to exist,   then the applicable
data manipulation rule is checked.  For instance,   if it is a \texttt{Cascade}
rule,   then the associated foreign keys in the referencing table are updated
prior to updating the key in the referenced table.  In the University \ac{RDB},  
if the primary key '\texttt{SWEN100}' for a course is updated to
'\texttt{SWEN101}',   then all the records in \texttt{Enrolment} that have
'\texttt{SWEN100}' as a foreign key have to be updated to '\texttt{SWEN101}',  
if it has a \texttt{Cascade} rule. 

When any foreign key is being updated in a referencing table,   then a referential
integrity validation has to be performed.  It is ensured that the new
updated value exists as a primary key in the referenced table.  For example,   in
the \texttt{Enrolment} table,   if \texttt{CourseID} in a row is updated to a new
value,   then it is verified that the new value is an existing primary key in the
\texttt{Course} table.  If the new value does not exist,   the
update is not allowed generally. 

\item Delete rule: A delete operation triggers a referential integrity
validation when data is deleted from the referenced table.  When data that is marked
for deletion is found to have dependencies in other referencing tables,   the
data manipulation rule applicable for this operation has to be checked.  This
means that if the rule allows \texttt{Cascade},   then the depending values in the
referencing table have to be removed prior to deleting values from the
referenced tables.  For example,   when a student record is deleted from the
\texttt{Student} table,   a check is performed to see if the '\texttt{StudentID}'
is a foreign key in any other table.  Therefore,   \texttt{Enrolment} would be
checked and when the '\texttt{StudentID}' is found as a foreign key,   the
appropriate action is performed depending on the data manipulation rule.  If it
is '\texttt{Cascade}',   the enrolment details for the '\texttt{StudentID}' are
removed from \texttt{Enrolment} and then the student record is deleted from
\texttt{Student}. 

\end{itemize}

Referential integrity constraints have been a relational feature in traditional
\acp{RDBMS} and is imposed due to the way the \acp{RDBMS} enforce normalisation. 
To improve the data dependency in cloud \ac{NoSQL} \acp{DBMS},   this
thesis proposes solutions that implement referential integrity constraints using
different approaches. These solutions implement and
validate referential integrity constraints based on these fundamental data manipulation and
referential integrity rules and are discussed in Chapter 3.

% These proposed solutions are deployed and analysed in Cassandra,  which is a
% column-oriented key-value \ac{DBMS}. To implement any solution it is necessary
% to understand the architecture and the key operations allowed in a \ac{DBMS}.
% For this purpose,  the architecture of Cassandra is discussed in the following section.



\section{Summary}



This chapter presented the background about the underlying concepts in cloud
computing and cloud databases.  It is clear that cloud computing is gaining
prevalence due to its many benefits like high data availability and cheap
storage etc.  With an increase in the number of users
migrating to cloud computing,   cloud data storage is gaining prominence as well,  
for easy and simple data storage.  This has paved the way for the existence
of many different data models and databases on the cloud.  Amongst the many data
models,   the key-value data model has been  most widely used  on the cloud
as it is more adapted to the cloud environment due to its support for
replication and scalability and other cloud related features (\todo{cite}). 



The next chapter describes the four solutions proposed to enforce referential
integrity constraints in cloud \ac{NoSQL} \acp{DBMS},   particularly in
Cassandra which is based on the column-oriented key-value data model. 





