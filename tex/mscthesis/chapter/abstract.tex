% ब

% \chapter*{Abstract}

\abstract{
Cloud computing delivers on-demand access to essential computing services 
providing benefits such as reduced maintenance, lower costs, global access,
and others. One of its important and prominent services  is
\ac{DaaS} which includes cloud \acp{DBMS}. 
Cloud  \acp{DBMS} commonly adopt the key-value data model and  are  called
\ac{NoSQL} \acp{DBMS}. These provide cloud suitable features like
scalability, flexibility and robustness, but in order to provide
these, features  such as r% \maketitle
eferential integrity are often sacrificed. 
In such cases, referential integrity is left to  be  dealt with by the
applications instead of being handled by the cloud \acp{DBMS}. Thus, 
 applications are required to  either deal with inconsistency in the data (e.g.
 dangling references) or to incorporate the necessary logic to ensure that
referential integrity is maintained.

This thesis presents an \ac{API} that serves as a middle layer between the
applications and the cloud \ac{DBMS} in order to maintain referential integrity.
The \ac{API} provides the necessary \ac{CRUD} operations  to be performed on
the \ac{DBMS} while ensuring that the referential integrity constraints are
satisfied. These constraints are represented as metadata and four different
approaches are provided to store it.
Furthermore, the performance of these approaches is measured with different
referential integrity constraints  and  evaluated upon a set of experiments in
Apache Cassandra, a prominent cloud \ac{NoSQL} \ac{DBMS}.
The results showed significant differences between the approaches in terms of
performance. However, the final word on which one is better depends on the
application demands as  each approach presents different trade-offs.
}


 
