% ब
\section{Metadata}\label{s:design-Metadata}
Metadata in \acp{DBMS} provide information about the data stored within the
databases.
It may contain details related to schemas, constraints,  primary and foreign
keys, and so on.   As previously mentioned,  most traditional \acp{RDBMS}
maintain such metadata within their \texttt{System}  tables or data
dictionaries.
In Apache Cassandra, the \ac{DBMS} of interest, metadata is stored in a keyspace
named \texttt{System} and it contains information about the cluster and its
nodes along with information related to the keyspaces, column families, and so
on~\citep{BOOK}.
 Even when Cassandra has a  \texttt{System} keyspace to store metadata, it is
 read-only and therefore it cannot be modified to store additional metadata
 about referential integrity constraints.
% As previously mentioned ,  most traditional \acp{RDBMS} maintain such metadata
% within their \texttt{System} tables or data dictionaries.
% Such metadata is decoupled from the actual data and its operations,  so that
% retrieving the metadata is faster since it does not involve handling the
% actual data(\todo{cite Duval}).
% It has been studied that the \ac{DaaS} is moving towards maintaining metadata
% in the cloud \acp{DBMS},  where commonly this metadata stores information
% about the nodes in a distributed cluster (\todo{cite Bin(2010)}).  For 
% maintaining the scalability required in such cloud \acp{DBMS}, metadata is
% often decoupled from the actual data so that accessing metadata does not cause
% a bottleneck in performance.  Cassandra maintains  metadata about the nodes in
% a cluster  in a separate keyspace named \texttt{System}, which stores the
% properties of every node, for example the node tokens,  the name of the
% cluster to which  nodes belongs to, information about the stored keyspaces and
% column families and so on(\todo{cite BOOK}).
% As per the design of Cassandra,  the \texttt{System} keyspace cannot be
% modified and thus  the metadata for the   solutions cannot be incorporated in
% this \texttt{System} keyspace.  Hence,  for preserving the metadata in each
% solution implements a  different strategy In other words, metadata is
% associated with actual data in different ways.  Associations can be classified
% as (\todo{cite Duval}):
Hence,  for preserving the metadata, each of the solutions implement a 
different strategy in which metadata is associated with actual data. Solutions~1
and~2 use embedded metadata, that is, metadata is created with the actual
data; while solutions~3 and~4 associate metadata separately from
the actual data.  Notice that, the structure of the metadata is kept the same across all the solutions even when  the way of
 storing and associating this metadata is different in each.

The role of metadata in  the solutions is primarily to hold the necessary
 information required to maintain referential integrity. The metadata contains
 information about primary keys,   foreign keys,  referenced and referencing
 column family details, constraints, and others.  The constraints considered in
 the solutions can be either \ac{PK} or \ac{FK} constraints. \ac{PK} constraints
 specify which column is the primary key of a column family. \ac{FK} constraints
 (or referential integrity constraints) determine the foreign key relationship
 between two column families, that is, the column of a column family which  is
 dependent on the primary key  column of another column family.  Hence, for each
 column family with a primary key,  the metadata  contains one \ac{PK}
 constraint  and  as many \ac{FK} constraints as foreign key relationships the
 column family has.
% Notice that, throughout the solutions, the structure of metadata containing
% these constraints is consistent despite.

The structure of the metadata is shown in Table~\ref{fd:Metadata-Constraints}.
This structure contains information about a University keyspace example in which
 a simple schema is applied for the keyspace. In this example,  the details of
the students are saved in  the \texttt{Student} column family and the course
 details in the \texttt{Course} column family.  The enrolment details of
 students are saved in the \texttt{Enrolment} column family by associating
 students to courses and hence having foreign key relationships to both
 \texttt{Student} and \texttt{Course} column families. All the column families
 have unique primary keys and their \ac{PK} constraints are saved in the
 metadata as presented in Table~\ref{fd:Metadata-Constraints} while the foreign
 key relationships between \texttt{Enrolment}, \texttt{Student} and \texttt{Course} are saved as \ac{FK} constraints.  For instance, consider in
 Table~\ref{fd:Metadata-Constraints} the \ac{PK} constraint \texttt{CONST100},
 for the \texttt{Student} column family, and the \ac{FK} constraint
 \texttt{CONST400} for the foreign key relationship between
\texttt{Enrolment} and \texttt{Student}. Notice that only
 single column primary keys are considered in the solutions.

\begin{table}[h] 
	\centering
	
	\newcolumntype{C}{@{\hspace{3pt}}>{\scriptsize}c@{\hspace{3pt}}}
	\begin{tabular}{CCC CCC CC}
		\toprule
		\bfseries Constraint & \bfseries Keyspace & \bfseries Constraint &
		\bfseries Column & \bfseries RKeyspace & \bfseries RConstraint &
		\bfseries RColumn & \bfseries DeleteRule\\[-6pt]
		\bfseries Name & & \bfseries Type & \bfseries Family & & \bfseries Name \\
		\midrule
		CONST100 & University & P & Student & University & & StudentId &\\
		\rc CONST200 & University & P & Course & University & & CourseId &\\
		CONST300 & University & P & Enrolment & University & & RowId &\\
% 		\hline
% 		\hline
		\rc CONST400 & University & R & Enrolment & University & CONST100 & StudentId
		& CASCADE\\
		CONST500 & University & R & Enrolment & University & CONST200 & CourseId &
		NODELETE\\
		\rc CONST600 & University & F & Course & University & CONST500 & CourseId &
		NODELETE\\
		CONST700 & University & F & Student & University & CONST400 & StudentId &
		CASCADE\\
		\bottomrule
	\end{tabular}
	\caption{Metadata for the Solutions}\label{fd:Metadata-Constraints}
\end{table}


Specifically, the structure of the metadata contains:

\begin{itemize}
  
  \item \texttt{ConstraintName:} is the name assigned for
  every constraint and it uniquely identifies an
  existing \ac{PK} or \ac{FK} constraint in the metadata. 
   For example,  \texttt{CONST100} and \texttt{CONST400} are 
  \texttt{ConstraintNames}.
  
  \item \texttt{Keyspace:}represents the name of the Keyspace the constraint
  belongs to. 
  
  \item \texttt{ConstraintType:} denotes the type of the constraint and the
  possible values are '\texttt{P}', '\texttt{R}' and '\texttt{F}'.
% The former referes to  a \ac{PK} constraint while the latter represents  a
% \ac{FK} constraint.
	A \ac{PK} constraint is referred by '\texttt{P}', while '\texttt{R}' and
	'\texttt{F}' are two representations of \ac{FK} constraints.
	'\texttt{R}' represents the referential integrity constraint (or  \ac{FK}
	constraint) a child entity has on a parent primary key, and '\texttt{F}'
	represents the  existing dependencies on a parent entity . For example,
	\texttt{CONST400} shows that the  parent entity for \texttt{Enrolment} is
	\texttt{Student} by looking up \texttt{RConstraintName}.
	\texttt{CONST700} shows that parent entity \texttt{Student} has child
	dependencies on it by following \texttt{CONST400}. Notice that, the constraint
	type '\texttt{F}' is primarily used to locate the child dependencies for a
	parent when it is deleted or updated.
% \begin{itemize} \item  '\texttt{P}' represents a \ac{PK} constraint \item
% '\texttt{R}' represents a

   
  \item \texttt{ColumnFamily:} refers to the column family this constraint
  applies to. For example,  the \ac{PK} constraint
  \texttt{CONST100}  applies on column family \texttt{Student}, and the \ac{FK}
  constraint \texttt{CONST400} applies on  \texttt{Enrolment}.
  
  \item \texttt{RKeyspace:} is the name of the keyspace on which this constraint
  is applied.  In the example, all the constraints  are applied in  the keyspace
  \texttt{University}.
  
    % In other words, it indicates which primary key
% is referenced or which .
  \item \texttt{RConstraintName:} represents the constraint that is referenced.
% For the constraint type '\texttt{R}' this field represents the referenced
% parent column families for a column family and for the constraint type
% '\texttt{F}' it shows the child dependencies for a column family.
  For the constraint type '\texttt{R}', this represents the referenced \ac{PK}
  constraint; and for the constraint type '\texttt{F}', it shows the child
  dependencies for a parent entity. In the example, the \ac{FK} constraint
  \texttt{CONST400} references the \ac{PK} constraint \texttt{CONST100},  which
  means that \texttt{Enrolment} has a foreign key relationship with
  \texttt{Student}. In \texttt{CONST700} this field indicates that \ac{FK}
  constraint \texttt{CONST400} exists for \texttt{Student}. Notice that this
  field is left blank in a \ac{PK} constraint since it has no references to
  other keys.
  
  \item \texttt{RColumn:}  indicates the primary key column on which this
  constraint is applicable.  For \ac{PK} constraints,  this holds the name of
  the primary key column. For \ac{FK} constraints, this field denotes
  the referenced column.  This example shows that the \ac{PK} constraint
  \texttt{CONST100} is applied on the primary key column \texttt{StudentId} of
  \texttt{Student} column family . The \ac{FK} constraint \texttt{CONST400}
  shows that the referenced column is \texttt{StudentId},  indicating that
  \texttt{Enrolment} references  primary key column \texttt{StudentId} of
  \texttt{Student}.
  
  \item \texttt{DeleteRule:} stores the type of data manipulation rule
  applicable on this constraint for a delete operation. Notice that for the
  sake of simplicity, this rule is also used for update operations. The possible
  rules considered in this thesis are \texttt{Cascade} and \texttt{NoDelete},
  other rules such as \texttt{Null} or \texttt{Default} are out of the scope of
  this thesis.  
%   values are \texttt{Cascade} and \texttt{NoDelete}.
%   For the sake of clarity and conciseness, other values like \texttt{Null} or \texttt{Default} are not
%   considered in the solutions. 
  Notice that, this field is not applicable for
  \ac{PK} constraints since data manipulation rules are associated with
  constraints that hold dependency information like the \ac{FK} constraints.
  

  
\end{itemize} 

In the solutions, metadata  is accessed whenever referential integrity
validations are triggered by
% to extract information about \ac{FK} constraints.
% % Specific methods are designed in all the solutions to retrieve and process
% the % metadata in order to validate referential integrity.
% Referential integrity validations are triggered when
\ac{CRUD}\footnote{Notice that \texttt{Read} does not trigger any referential
integrity validations, but the \ac{CRUD} acronym is still used all throughout
this thesis for the sake of familiarity.} operations performed on a column
family. Thus, the relevant \ac{FK} constraints  to perform such validations are
accessed from the metadata. Notice that, each solution stores metadata in a
distinct way and provides specific methods to access and process the metadata to
support the validation. The logic for validating the referential integrity is
consistent across all the solutions. The way these solutions store metadata and
the motivation behind the design of its metadata storage is presented in the
following sections.










